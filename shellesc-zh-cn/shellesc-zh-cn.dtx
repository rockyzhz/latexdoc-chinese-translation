% \iffalse
%% Source File: shellesc.dtx
%% Copyright (C) 2015-2023
%%
%% The LaTeX Project and any individual authors listed elsewhere
%% in this file.
%%
%% This file may be distributed under the terms of the LPPL.
%% See README for details.
%
%<*dtx>
          \ProvidesFile{shellesc.dtx}
%</dtx>
%<package>\ifx\ProvidesPackage\undefined
%<package>\def\next#1#2[#3]{\wlog{#2 #3}}
%<package>\ifx\PackageInfo\undefined\def\PackageInfo#1#2{\wlog{#1: #2}}\fi
%<package>\ifx\PackageWarning\undefined\def\PackageWarning#1#2{\wlog{#1: #2}}\fi
%<package>\expandafter\next\fi
%<package>\ProvidesPackage{shellesc}
%<driver> \ProvidesFile{shellesc.drv}
% \fi
%         \ProvidesFile{shellesc.dtx}
       [2023/07/08 v1.0d unified shell escape interface for LaTeX]
%
% \iffalse
%<*driver>
\documentclass{ltxdoc}
% ^^A =================== 中文版特别设置 =========================
\usepackage[fontset=source]{ctex}
\usepackage{xcolor}
\definecolor{spot}{HTML}{003399}
\definecolor{code}{HTML}{a25e26}
\definecolor{verb}{HTML}{007f00}
\usepackage{etoolbox}
\let\OriginalMF\MacroFont
\def\MacroFont{\color{spot}\OriginalMF}
\makeatletter
\pretocmd{\macrocode}{\def\macro@font{\color{code}\OriginalMF}\def\MacroFont{\color{code}\OriginalMF}}{}{}
\makeatother
\AtBeginEnvironment{quote}{\color{verb}}
\AtBeginEnvironment{flushleft}{\tt\color{verb}}
\AtBeginEnvironment{verbatim}{\color{verb}}
\usepackage{hyperref}
\hypersetup{%
  colorlinks=true,
  linkcolor=spot,
  urlcolor=spot,
  citecolor=spot,
  bookmarksopen=false,
  bookmarksnumbered=false,
  plainpages=false}
% ^^A =================== 中文版设置结束 =========================
\begin{document}
\DocInput{shellesc-zh-cn.dtx}
\end{document}
%</driver>
% \fi
%
% \GetFileInfo{shellesc.dtx}
%
% \ExplSyntaxOn
% \def\zhfiledate{\exp_args:No \zhdate {\filedate}}
% \ExplSyntaxOff
% \title{{\bfseries\huge\color{spot}\textsf{shellesc} 宏包}\thanks{本文件
%        版本号为 \fileversion, 最后
%        修改于 \zhfiledate 。}}
% \author{
%   \begin{tabular}{r@{\hspace*{\ccwd}}l}
%       \LaTeX\ 项目组 & 【著】\\
%       张泓知 & 【译】
%   \end{tabular}}
% \date{\zhfiledate}
%
% \maketitle
%
% \vspace*{-\baselineskip}
% \enlargethispage{\baselineskip}
%
% \changes{v0.1b}{2016/02/02}{文档错字修正(JB)}
% \changes{v0.2a}{2016/06/07}{改进与 plain TeX 的使用}
% \section{引言}
%
%
% 多年来,基于 web2c 的 \TeX\ 实现一直使用 \verb|\write| 命令的语法来访问系统命令,
% 它使用了特殊的 18 号流(15 号及以上的流不能分配给经典 \TeX\ 中的文件,所以 18 号流
% 本应只是输出到终端)。
%
% 这是一个有用的扩展,没有违反经典 \TeX\ 中关于扩展的严格规则。该宏包提供了一个简单的
% 宏级接口来隐藏 \verb|write18| 的实现,因此可以使用类似 \verb|\ShellEscape{rm file.txt}|
% 的命令在类 Unix 系统中指定删除文件(在 Windows 中使用 \verb|del|)。请注意,默认情况下
% 禁止访问系统,并且通常需要使用命令行选项 \verb|--shell-escape| 调用 \LaTeX。
%
% 该包可以与标准的 \texttt{latex}、\texttt{pdflatex} 或 \texttt{xetex} 一起使用,
% 但它主要是为了 \texttt{lualatex} 而设计的,因为从 Lua\TeX\ 0.87 开始,Lua\TeX\
% \emph{不再}支持 \verb|\write18| 语法来访问系统命令:它具有 256 个写入流,流 18
% 可以关联到文件(没有此包时)并且没有特殊意义。这个包在 Lua\LaTeX\ 中定义了相同的
% \verb|\ShellEscape| 语法,但实现是通过 Lua 和 \verb|os.execute| 函数完成的。
%
% 实际上,\verb|\ShellEscape| 对应于 \verb|\immediate\write18|(或 \verb|\directlua|)。
% 极少数情况下,您可能需要延迟系统命令直到输出当前页面(当页码已知时),对于这种情况,
% 您可以经典地使用 \verb|\write18|(或 \verb|\latelua|)。此包提供了
% \verb|\DelayedShellEscape| 作为这种用法的常用语法。
%
% 可以通过检查整数(chardef)命令 \verb|\ShellEscapeStatus| 查询 shell escape 状态,
% 0(禁用)、1(启用)、2(受限)。
%
% 为了帮助将现有文档移植到 Lua\TeX\ 0.87,该包重载了 \verb|\write| 命令,以便
% \verb|\write18{rm file.txt}| 在 Lua\TeX\ 中起作用。请注意,\verb|\write| 的重新定义
% 无法检测出是否已使用 \verb|\immediate|,当写入文件流或终端时,\verb|\immediate| 会
% 正常工作,但流 18 的特殊情况,它被定义为始终使用 \verb|os.execute|,始终使用
% \verb|\directlua|(因此对应于 \verb|\immediate\write18|)。在需要将非即时
% \verb|\write18| 转换为当前 Lua\TeX\ 的文档中的极少情况下,您需要更改为使用
% \verb|\DelayedShellEscape| 命令。
%
% \section{代码实现}
%
%    \begin{macrocode}
%<*package>
%    \end{macrocode}
%
% \changes{v1.0b}{2019/10/17}{分类码 (Catcode) 保护}
%    \begin{macrocode}
\chardef\shellesc@quotecat\catcode`\"
\chardef\shellesc@underscorecat\catcode`\_
\@makeother\"
\@makeother\_
%    \end{macrocode}
%
% \subsection{状态检查}
%
%
% \subsection{shellesc 宏包接口}
%
% \begin{macro}{\ShellEscapeStatus}
% \changes{v1.0a}{2019/10/13}{引入命令}
% 整数值的含义分别为 0(禁用 shell escape)、1(允许 shell escape)、2(受限制的 shell escape)。
%
%    \begin{macrocode}
\chardef\ShellEscapeStatus
  \ifx\pdfshellescape\@undefined
    \ifx\shellescape\@undefined
      \ifx\directlua\@undefined
        \z@
      \else
        \directlua{%
          tex.sprint((status.shell_escape or os.execute()) .. " ")}
      \fi
    \else
      \shellescape
    \fi
  \else
    \pdfshellescape
  \fi
%    \end{macrocode}
% \end{macro}
%
% \changes{v0.2a}{2016/06/07}{消息中的拼写错误}
%    \begin{macrocode}
\ifcase\ShellEscapeStatus
  \PackageWarning{shellesc}{Shell escape disabled}
\or
  \PackageInfo   {shellesc}{Unrestricted shell escape enabled}
\else
  \PackageInfo   {shellesc}{Restricted shell escape enabled}
\fi
%    \end{macrocode}
%
%
% \begin{macro}{\ShellEscape}
% \changes{v1.0a}{2019/10/13}{Lua 日志记录用于 gh/195}
% 执行提供的记号作为一个系统相关命令,假设允许这样的执行。
%    \begin{macrocode}
\ifx\lastsavedimageresourcepages\@undefined
  \protected\def\ShellEscape{\immediate\write18 }
%    \end{macrocode}
%
%    \begin{macrocode}
\else
  \protected\def\ShellEscape{\directlua\ShellEscape@Lua}
\fi
%    \end{macrocode}
% \end{macro}
%
% \begin{macro}{\DelayedShellEscape}
% \changes{v0.1c}{2016/04/29}{定义 \cs{DelayedShellEscape} 而不是 \cs{ShellEscape}(UF)}
% \changes{v1.0a}{2019/10/13}{Lua 日志记录用于 gh/195}
% 当此节点随着完成的页面输出时,将提供的记号作为一个系统相关命令执行,假设允许这样的执行。
%    \begin{macrocode}
\ifx\lastsavedimageresourcepages\@undefined
  \protected\def\DelayedShellEscape{\relax\write18 }
%    \end{macrocode}
%
%    \begin{macrocode}
\else
  \protected\def\DelayedShellEscape{\latelua\ShellEscape@Lua}
\fi
%    \end{macrocode}
% \end{macro}
%
%
%
% \begin{macro}{\ShellEscape@Lua}
% \changes{v1.0a}{2019/10/13}{gh/195 的日志记录}
% \changes{v1.0d}{2023/04/15}{在消息 gh/1008 中添加省略号}
% \verb|\DelayedShellEscape| 和 \verb|\ShellEscape| 的共享 Lua 代码。
%    \begin{macrocode}
\ifx\directlua\@undefined\else
\protected\def\ShellEscape@Lua#1{{%
local status, msg = os.execute("\luaescapestring{#1}")%
if status == nil then
    texio.write_nl("log",%
      "runsystem(" .. "\luaescapestring{#1}"%
       .. ")...(" .. msg .. ").\string\n")
  elseif status == 0 then
    texio.write_nl("log",%
      "runsystem(" .. "\luaescapestring{#1}"%
      .. ")...executed.\string\n")
  else
    texio.write_nl("log",%
      "runsystem(" .. "\luaescapestring{#1}"%
       .. ")...failed. " .. (msg or "") .. "\string\n")
  end
}}
\fi
%    \end{macrocode}
% \end{macro}
%
% \subsection{write18 宏包接口}
%
% 在除 Lua\TeX\ 外的基于 web2c 的引擎中,可以直接使用 |\write18|。
% 在较旧的 Lua\TeX\ 中也是如此,但从版本 0.85 开始,这不再可用。
%
% 上述的 |shellesc| 宏包接口推荐用于新代码,然而,为了方便将现有文档和包移植到更新的
% Lua\TeX\ 版本,这里通过调用 Lua 的 |os.execute| 提供了一个 |\write18| 接口。
%
% 请注意,目前的写法总是对系统进行\emph{立即}调用。
%
% |\immediate| 是被支持但被忽略的,|\immediate\write18| 和 |\write18| 都会立即执行。
% 要在下一个输出时延迟执行,请使用上面定义的\hfil\break |\DelayedShellEscape| 命令。
%
% 请注意,可以很容易地使此处定义的 |\write18| 使用延迟执行,只需在下面的定义中使用
% |\DelayedShellEscape| 而不是 |ShellEscape|。然而,检测 |\immediate| 是有技巧性的,
% 因此这里的选择是始终使用即时形式,这在与 |\write18| 一起使用时是绝大多数情况下使用的形式。
%
% 如果不是最近的 Lua\TeX,请在此处停止。
%    \begin{macrocode}
\ifx\lastsavedimageresourcepages\@undefined
 \catcode`\"\shellesc@quotecat
 \catcode`\_\shellesc@underscorecat
 \expandafter\endinput
\fi
%    \end{macrocode}
%
%    \begin{macrocode}
\directlua{%
%    \end{macrocode}
%
%    \begin{macrocode}
shellesc = shellesc or {}
%    \end{macrocode}
%
% 使用记号扫描器获取下一个 \TeX\ 数字的 Lua 函数,并测试是否正在使用流 18,
% 然后在每种情况下插入适当的 \TeX\ 命令来处理随后的大括号组。
%    \begin{macrocode}
local function write_or_execute()
  local s = token.scan_int()
  if (s==18) then
     tex.sprint(\the\numexpr\catcodetable@atletter\relax,
                "\string\\ShellEscape ")
  else
     tex.sprint(\the\numexpr\catcodetable@atletter\relax,
                "\string\\shellesc@write " .. s)
  end
end
%    \end{macrocode}
%
%    \begin{macrocode}
shellesc.write_or_execute=write_or_execute
%    \end{macrocode}
%
%    \begin{macrocode}
}
%    \end{macrocode}
%
%    \begin{macrocode}
\let\shellesc@write\write
%    \end{macrocode}
%
%    \begin{macrocode}
\protected\def\write{\directlua{shellesc.write_or_execute()}}
%    \end{macrocode}
%
%    \begin{macrocode}
\catcode`\"\shellesc@quotecat
\catcode`\_\shellesc@underscorecat
%    \end{macrocode}
%
%    \begin{macrocode}
%</package>
%    \end{macrocode}
%
% \Finale
%
