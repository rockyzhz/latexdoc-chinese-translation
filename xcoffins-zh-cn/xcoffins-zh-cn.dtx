% \iffalse meta-comment
%
%% File: xcoffins.dtx
%
% Copyright (C) 2010-2024 The LaTeX Project
%
% It may be distributed and/or modified under the conditions of the
% LaTeX Project Public License (LPPL), either version 1.3c of this
% license or (at your option) any later version.  The latest version
% of this license is in the file
%
%    https://www.latex-project.org/lppl.txt
%
% This file is part of the "l3experimental bundle" (The Work in LPPL)
% and all files in that bundle must be distributed together.
%
% -----------------------------------------------------------------------
%
% The development version of the bundle can be found at
%
%    https://github.com/latex3/latex3
%
% for those people who are interested.
%
%<*driver>
\documentclass[full]{l3doc}
\usepackage{xcoffins} % Not yet part of expl3, so not in l3doc
\NewCoffin \ExampleCoffin
\NewCoffin \SmallCoffin
\NewCoffin \OutputCoffin
\NewCoffin \RedCoffin
\NewCoffin \BlueCoffin
\NewCoffin \GreenCoffin
\NewCoffin \YellowCoffin
\NewCoffin \OrangeCoffin


\usepackage{float} %%%% 防止表格等浮动
%%%%%%%%%%%%% 以下设置中文字体 %%%%%%%%%%%%%%%%%%%%%%%%%%%%%%%%%%%%%%%%%
\usepackage{xeCJK}  %%

\setCJKfamilyfont{Heiti}{Source Han Sans SC Normal} %%%% 自定义\Heiti命令,显示思源黑体,用于标题页标题的中文部分
\newcommand{\Heiti}{\CJKfamily{Heiti}} %%%% 自定义\Heiti命令,显示思源黑体,用于标题页标题的中文部分

\setCJKfamilyfont{heiti}{Source Han Sans SC Light} %%自定义\heiti命令,显示思源黑体,用于正文的章节标题
\newcommand{\heiti}{\CJKfamily{heiti}} %%自定义\heiti命令,显示思源黑体,用于正文的章节标题

\setCJKfamilyfont{songti}{Source Han Serif SC Light}  %%%% 自定义\songti命令,显示思源宋体,用于正文
\newcommand{\songti}{\CJKfamily{songti}} %%%% 自定义\songti命令,显示思源宋体,用于正文

\setCJKfamilyfont{heitixt}{思源黑体_CN_LightItalic.otf}  %%%% 自定义\heitixt命令,显示思源黑体斜体
\newcommand{\heitixt}{\CJKfamily{heitixt}} %%%% 自定义\heitixt命令,显示思源黑体斜体

\setCJKmainfont{Source Han Serif SC Light} %%%% 设置中文的主字体为思源宋体 CN Light
\setmainfont{SourceSerifPro-Regular.otf} %%%% 设置英文的主字体为Source Serif Pro,最好看
%%%%\setmainfont{Source Han Serif SC} %%%% 设置英文的主字体为Source Han Serif SC
%%%%%%\setmainfont{Times New Roman} %%%% 设置英文的主字体为Times New Roman

\setCJKfamilyfont{kaiti}{KaiTi} %%设置中文字体楷体,用于强调
\newcommand{\kaiti}{\CJKfamily{kaiti}} %%设置中文字体楷体,用于强调
%%%%%%%%%%%%% 以上设置中文字体 %%%%%%%%%%%%%%%%%%%%%%%%%%%%%%%%%%%%%%%%%
\usepackage{fontspec}



%%% 以下输入带圈的数字,调用时的命令:如 \char"2469 生成 ⑩ %%%%
%%详参目录中的“latex 如何添加圆圈数字? - Tsingke - 博客园.mhtml”%%%%
\xeCJKDeclareCharClass{CJK}{%
  "24EA,        % ⓪
  "2460->"2473, % ①–⑳
  "3251->"32BF, % ㉑–㊿
  "24FF,        % ⓿
  "2776->"277F, % ❶–❿
  "24EB->"24F4  % ⓫–⓴
}
%%% 以上输入带圈的数字,调用时的命令:如 \char"2469 生成 ⑩ %%%%

%%%%%%%%%%%%% 以下设置中文版式 %%%%%%%%%%%%%%%%%%%%%%%%%%%%%%%%%%%%%%%%%
\usepackage{indentfirst} %%% 首行缩进
\setlength{\parindent}{2em} %%% 缩进2个字符(中文为2个字)
\linespread{1.5} %%% 设置行间距
%%%%%%%%%%%%% 以上设置中文版式 %%%%%%%%%%%%%%%%%%%%%%%%%%%%%%%%%%%%%%%%%
\usepackage{changepage} %%%用于整体缩进,\begin{adjustwidth}{2cm}{1cm}

%%%%%%% 以下在 tabular 表格中定制 横线如\hlinew{1.2pt} %%%%%%
\makeatletter
\def\hlinew#1{%
\noalign{\ifnum0=`}\fi\hrule \@height #1 \futurelet
\reserved@a\@xhline}
\makeatother%
%%%%%%% 以上在 tabular 表格中定制 横线如\hlinew{1.2pt} %%%%%%

%%%%%%% 以下自定义脚注 %%%%%%%%%%%%%%%%%%%%%%%%%%%%%%%%%%%%
\setlength{\footnotesep}{0.5cm} %%%设置几第脚注之间的距离
\setlength{\skip\footins}{2.0em} %%%设置脚注与正文之间的距离
%%\renewcommand\footnoterule{} %%%定义脚注线为空
\renewcommand\footnoterule{
     \kern -3pt                         % This -3 is negative
     \hrule width 0.6\textwidth height 0.6pt % of the sum of this 1
     \kern 2pt} %%%
%%%%%%% 以上自定义脚注 %%%%%%%%%%%%%%%%%%%%%%%%%%%%%%%%%%%%

\renewcommand{\contentsname}{\centerline {\heiti {\Large 目\ \ \ 录}}}   %%% 在{document}后面加入该命令,将"contents"变成“目  录”
%%%\renewcommand{\thepart}{第{\Roman{part}}部分}
\renewcommand{\refname}{\heiti 参考文献}
\renewcommand{\figurename}{\Heiti 图}
\renewcommand{\tablename}{\Heiti 表}
\renewcommand{\abstractname}{\heiti {\Large 摘\ 要}}
\renewcommand{\listfigurename}{\centerline{\heiti {\large 图形目录}}}
\renewcommand{\listtablename}{\centerline{\heiti {\large 表格目录}}}
\renewcommand{\indexname}{\heiti 索引} %%%%让最后生成的PDF文件的书签的索引显示“索引”而不是“Index”

%%%%%%%%%% 以下将正文中的“Part Ⅰ ”中文化成“第Ⅰ部分” %%%%%%%%%%%%%%%%%%%%%%%%%%%%%%%%%%%%%%%%%
%%%\usepackage[center]{titlesec}  %%标题居中
\usepackage{titlesec}
\titleformat{\part}{\centering\Large\bfseries}{{\heiti 第} \thepart {\heiti 部分}}{1.2em}{}
%%%%%%%%%% 以上将正文中的“Part Ⅰ ”中文化成“第Ⅰ部分” %%%%%%%%%%%%%%%%%%%%%%%%%%%%%%%%%%%%%%%%%

%%%%%%% 以下调整目录条目之间的间距 %%%%%%%%%%%%%%%%%%%%%%%%%%%%%%%%%%%%
\usepackage{tocloft}
\setlength{\cftbeforetoctitleskip}{20pt} %%% “目录”二字的段前间距为20pt
\setlength{\cftaftertoctitleskip}{20pt}  %%% “目录”二字的段后间距为50pt
\setlength{\cftbeforepartskip}{25pt} %%% 部分(part)之前的空白为25pt
\renewcommand{\cftpartafterpnum}{\vspace{6pt}} %%% 部分(part)之后的空白为6pt
\setlength{\cftbeforesecskip}{7pt} %%% 节(sec)之前的空白为14pt
\renewcommand{\cftsecafterpnum}{\vspace{3pt}} %%% 节(sec)之后的空白为3pt
\renewcommand{\cftsubsecafterpnum}{\vspace{3pt}} %%% 小节(subsec)之后的空白为3pt
%%%%%%% 以上调整目录条目之间的间距 %%%%%%%%%%%%%%%%%%%%%%%%%%%%%%%%%%%%

%%%%%%%%%%%% 以下设置书签和目录的颜色、链接%%%%%%%%%%%%%%%%%%%%%%%%%%%%%%
\usepackage[svgnames]{xcolor}
\definecolor{myurlcolor}{rgb}{0,0,0.7}
%%%%\definecolor{mylinkcolor}{rgb}{0.7,0,0}
\definecolor{mylinkcolor}{RGB}{178,0,0}
\definecolor{codecolor}{rgb}{0,0.4,0.2}
\definecolor{overviewcolor}{rgb}{0,0.2,0.4}
\definecolor{Mylightgreen}{RGB}{216,233,236} %定义名为Mylightgreen的颜色(RGB分别为216,233,236)
\usepackage{hyperref}
\hypersetup{xetex,bookmarks=true,hidelinks,%
colorlinks,linkcolor=mylinkcolor,urlcolor=myurlcolor,%
pageanchor=true,hyperindex=true,
}
%%%%%%%%%%%% 以上设置书签和目录的颜色、链接%%%%%%%%%%%%%%%%%%%%%%%%%%%%%%

%%%%%%%%%%%% 以下设置网址太长自动换行 %%%%%%%%%%%%%%%%%%%%%%%%%%%%%%
\usepackage{url}
\def\UrlBreaks{\do\A\do\B\do\C\do\D\do\E\do\F\do\G\do\H\do\I\do\J
\do\K\do\L\do\M\do\N\do\O\do\P\do\Q\do\R\do\S\do\T\do\U\do\V
\do\W\do\X\do\Y\do\Z\do\[\do\\\do\]\do\^\do\_\do\`\do\a\do\b
\do\c\do\d\do\e\do\f\do\g\do\h\do\i\do\j\do\k\do\l\do\m\do\n
\do\o\do\p\do\q\do\r\do\s\do\t\do\u\do\v\do\w\do\x\do\y\do\z
\do\.\do\@\do\\\do\/\do\!\do\_\do\|\do\;\do\>\do\]\do\)\do\,
\do\?\do\'\do+\do\=\do\#\do\&}
%%%%%%%%%%%% 以上设置网址太长自动换行 %%%%%%%%%%%%%%%%%%%%%%%%%%%%%%

%%%%%%%%%%%%%%%%%%%% 其它设置 %%%%%%%%%%%%%%%%%%%%%%%%%%%%%%%%%
\EnableCrossrefs
\CodelineIndex
%%%%%%%%%%%%%%%%%%%%%%%%%%%%%%%%%%%%%%%%%%%%%%%%%%%%%%%%%%%%%%

\begin{document}
\DocInput{xcoffins-zh-cn.dtx}
\end{document}
%</driver>
% \fi
%
% \title{^^A
% {\Huge \pkg{xcoffins}}\ {\huge \Heiti 宏包}\\[4pt] {\huge \Heiti 设计级别的排版盒子}^^A
% }
%
% \author{^^A
%  \LaTeX{}\ 项目\thanks
%    {^^A
%      E-mail:
%        \href{mailto:latex-team@latex-project.org}
%          {latex-team@latex-project.org}^^A
%    }\ \ \ 【著】^^A
%    \\[4pt]黄\ 旭\ 华\ \thanks{一名业余 \LaTeX\ 爱好者。}\ \ \ \ \ \ 【译】
% }
%
% \date{分发于\ 2024-01-04}
%
% \maketitle
%
% \newpage
% \tableofcontents
%
% \newpage

% \begin{documentation}
%
% \section{{\heiti 介绍}:coffin {\heiti 的概念}}
%
% 在 \LaTeX3\ 术语(terminology)中,“coffin (排版盒子)”是一个装有排版材料
% (typeset material)的盒子(box)。除了盒子本身,coffin 的结构(structure)还包括盒子
% 尺寸(size)和形状(shape)的信息,这使得可以轻松地对齐两个或多个 coffins。
% 这是通过为每个 coffin 提供一系列“杆(pole)”来实现的。这些杆(pole)是在
% 确定的位置(defined positions)如顶部或水平中心(horizontal centre)穿过 coffin 的
% 水平线(horizontal lines)和垂直线(vertical lines),
% 这些杆(pole)相交(intersect)的点称为“handles (手柄)”。然后,
% 通过描述一个 coffin 上的手柄与另一个 coffin 的手柄之间的关系,就可以将两个 coffin 对齐。
% 用文字表达(in words),一个例子可能是这样的:
% \begin{quote}
%   Align the top-left handle of coffin A with the bottom-right
%   handle of coffin B.
%
%  {\kaiti 即:将 coffin A 的左上手柄(top-left handle)与 coffin B 的右下手柄(bottom-right handle)对齐。}
% \end{quote}
%
% coffin 手柄的位置从视觉上很容易理解。图~\ref{fgr:handles}~显示了
% 水平模式(horizontal mode)(左)和垂直模式(vertical mode)(右)中 coffin 排版的
% 标准手柄位置(standard handle positions)。请注意,后一种情况会有更多的手柄可用。
% 如图所示,每个手柄都是两个杆(pole)相交的结果。例如,coffin 的中心标记为|(hc,vc)|,
% {\kaiti 即}水平中心杆(horizontal centre pole)与垂直中心杆(vertical centre pole)的交点。
% 当杆添加到 coffin 中时,会自动生成新的手柄:手柄是“动态的(dynamic)”实体(entities)。
%
% \vspace{4em}
% \begin{figure}[H]
%   \hfil
%   \begin{minipage}{0.4\textwidth}
%     \SetHorizontalCoffin\ExampleCoffin
%       {\color{black!10!white}\rule{1 in}{1 in}}
%     \DisplayCoffinHandles\ExampleCoffin{blue}
%   \end{minipage}
%   \hfil
%   \begin{minipage}{0.4\textwidth}
%     \SetVerticalCoffin\ExampleCoffin{1 in}
%       {\color{black!10!white}\rule{1 in}{1 in}}
%     \DisplayCoffinHandles\ExampleCoffin{blue}
%   \end{minipage}
%   \vspace{4em}
%   \caption{标准的 coffin 手柄。左图,水平 coffin;右图,垂直 coffin。}
%   \label{fgr:handles}
% \end{figure}
%
% \newpage
%
%\section{{\heiti 创建和设置} coffins}
%
% 在进行任何对齐(alignment)之前,必须创建 coffins 并创建其内容。除了 coffins 制作以外,
% 所有的 coffins 操作(operations)都是当前 \TeX{}\ 组(group)的局部操作。
% \begin{function}{\NewCoffin}
%   \begin{syntax}
%     \cs{NewCoffin} \meta{coffin}
%   \end{syntax}
%   在使用 \meta{coffin}\ 之前,它必须使用 \cs{NewCoffin}\ 进行分配(allocated)。
%   \meta{coffin}\ 的名称应该是一个控制序列(control sequence)
%   (通常以转义字符 |\| 开始),例如:
%   \begin{verbatim}
%     \NewCoffin\MyCoffin
%   \end{verbatim}
%   coffins 是全局分配的(allocated globally),如果 \meta{coffin}\ 的名称不是
%   全局唯一的(globally-unique),就会引起错误。
% \end{function}
%
% \begin{function}{\SetHorizontalCoffin}
%   \begin{syntax}
%     \cs{SetHorizontalCoffin} \meta{coffin} \Arg{material}
%   \end{syntax}
%   在水平模式(horizontal mode)中排版 \meta{material},将结果存储在 \meta{coffin}\ 中。
%   然后根据排版材料(typeset material)的尺寸(size)建立 \meta{coffin}\ 的标准杆(standard poles)。
% \end{function}
%
% \begin{function}{\SetVerticalCoffin}
%   \begin{syntax}
%     \cs{SetVerticalCoffin} \meta{coffin} \Arg{width} \Arg{material}
%   \end{syntax}
%   在给定 \meta{width}\ 的垂直模式(vertical mode)中排版 \meta{material},并将结果
%   存储在 \meta{coffin}\ 中。然后根据排版材料(typeset material)的尺寸(size)建
%   立 \meta{coffin}\ 的标准杆(standard poles)。
% \end{function}
%
%
% \section{{\heiti 控制} coffin {\heiti 的杆}}
%
% 设置或对齐 coffin 时,会自动生成多个标准杆(standard poles)。coffins 的所有标准杆如下:
% \begin{itemize}[font = \ttfamily]
%   \item[l] 沿着 coffin 边框左侧边缘(left-hand edge)延伸的杆;
%   \item[hc] 垂直穿过 coffin 中心的杆,位于边框的左右边缘之间,{\kaiti 即}“水平中心”;
%   \item[r] 沿着 coffin 边框右侧边缘(right-hand edge)延伸的杆;
%   \item[b] 沿着 coffin 边框底部边缘(bottom edge)延伸的杆;
%   \item[vc] 水平穿过 coffin 中心的杆,位于边框的顶底边缘之间,{\kaiti 即}“垂直中心”;
%   \item[t] 沿着 coffin 边框顶部边缘(bottom edge)延伸的杆;
%   \item[H] 沿着 coffin 中排版材料的基线(baseline)延伸的杆。
% \end{itemize}
% 此外,装有垂直模式材料(vertical-mode material)的 coffins 还具有反映这些系统更丰富性质(nature)的杆:
% \begin{itemize}
%   \item[B] 沿着 coffin 底部材料的基线延伸的杆。
%   \item[T] 沿着 coffin 顶部材料的基线延伸的杆。
% \end{itemize}
%
% \begin{function}{\SetHorizontalPole}
%   \begin{syntax}
%     \cs{SetHorizontalPole} \meta{coffin} \Arg{pole} \Arg{offset}
%   \end{syntax}
%   设置横向穿过 \meta{coffin}\ 的 \meta{pole}。\meta{pole}\ 位于 \meta{coffin}\ 基
%   线的 \meta{offset}\ 上。这个 \meta{offset}\ 应作为尺寸表达式(dimension expression)给出,
%   这可能包括术语 \cs{TotalHeight}、\cs{Height}、\cs{Depth}\ 和 \cs{Width},它们
%   将计算 \meta{coffin}\ 的适当尺寸(appropriate dimensions)。例如,要在 coffin 底部
%   到顶部距离的三分之一处创建一根横穿 coffin 的杆,适当的指令(instruction)应该是:
%   \begin{verbatim}
%     \SetHorizontalPole \MyCoffin {height/3} {\TotalHeight/3}
%   \end{verbatim}
%   注意,{\kaiti 水平}(\emph{horizontally})延伸的杆是根据其在 coffin 中的
%   {\kaiti 垂直}(\emph{vertical})位置来描述的。还请注意,coffin 的总高度(total height)
%   由 \cs{Height}\ 和 \cs{Depth}\ 之和表示:这两个值都是从 coffin 中材料的
%   水平基线(horizontal baseline)来测量的。
% \end{function}
%
% \begin{function}{\SetVerticalPole}
%   \begin{syntax}
%     \cs{SetVerticalPole} \meta{coffin} \Arg{pole} \Arg{offset}
%   \end{syntax}
%   设置垂直穿过 \meta{coffin}\ 的 \meta{pole}。\meta{pole}\ 位于 \meta{coffin}\ 盒子
%   边框左侧边缘(left-hand edge)的 \meta{offset}\ 上。这个 \meta{offset}\ 应作为
%   尺寸表达式(dimension expression)给出,这可能包括术语 \cs{TotalHeight}、\cs{Height}、
%   \cs{Depth}\ 和 \cs{Width},它们将计算 \meta{coffin}\ 的适当尺寸(appropriate dimensions)。
%   例如,要在 coffin 的左侧边缘三分之一处创建一根垂直穿过 coffin 的杆,适当的指令(instruction)应该是:
%   \begin{verbatim}
%     \SetVerticalPole \MyCoffin {width/3} {\Width/3}
%   \end{verbatim}
%   注意,{\kaiti 垂直}(\emph{vertically})延伸的杆是根据其在 coffin 中的
%   {\kaiti 水平}(\emph{horizontal})位置来描述的。
% \end{function}
%
% \begin{function}{\TotalHeight}
%   \begin{syntax}
%     \cs{TotalHeight}
%   \end{syntax}
%   在 \cs{SetHorizontalPole}\ 和 \cs{SetVerticalPole}\ 的 \meta{offset}\ 参数中,
%   \cs{TotalHeight}\ 将给出从底部(base)到相关 coffin 的边界框(bounding box)顶部(top)的距离。
%
% \end{function}
%
% \begin{function}{\Height}
%   \begin{syntax}
%     \cs{Height}
%   \end{syntax}
%   在 \cs{SetHorizontalPole}\ 和 \cs{SetVerticalPole}\ 的 \meta{offset}\ 参数中,
%   \cs{Height}\ 将给出从基线(baseline)到相关 coffin 的边界框(bounding box)顶部(top)的距离。
% \end{function}
%
% \begin{function}{\Depth}
%   \begin{syntax}
%     \cs{Depth}
%   \end{syntax}
%   在 \cs{SetHorizontalPole}\ 和 \cs{SetVerticalPole}\ 的 \meta{offset}\ 参数中,
%   \cs{Depth}\ 将给出从基线(baseline)到相关 coffin 的边界框(bounding box)底部(bottom)的距离。
% \end{function}
%
% \begin{function}{\Width}
%   \begin{syntax}
%     \cs{Width}
%   \end{syntax}
%   在 \cs{SetHorizontalPole}\ 和 \cs{SetVerticalPole}\ 的 \meta{offset}\ 参数中,
%   \cs{Width}\ 将给出相关 coffin 的边界框(bounding box)从右边缘(right edge)到
%   左边缘(left edge)的距离。
% \end{function}
%
%
% \section{{\heiti 旋转} coffins}
%
% \begin{function}{\RotateCoffin}
%   \begin{syntax}
%     \cs{RotateCoffin} \meta{coffin} \Arg{angle}
%   \end{syntax}
%   用给定的关于其参考点的 \meta{angle}\ (逆时针方向给定的角度)旋转 \meta{coffin}。
%   这个过程将旋转 coffin 的内容(content)和杆(poles)。多次旋转不会导致 coffin 的
%   边界框(bounding box)不必要地增长(growing)。
%
%   旋转对 coffin 的影响如图~\ref{fgr:rotation}~所示。如图所示,coffin 的手柄(handles)将
%   保持相对于 coffin 内容物的正确位置。旋转后 coffin 的“顶(top)”当然可能
%   不再是最接近物理页面顶部(top of the physical page)的边缘。
% \end{function}
% \vspace{4em}
%   \begin{figure}[H]
%     \hfil
%     \SetHorizontalCoffin\ExampleCoffin
%       {^^A
%         \color{black!10!white}\rule{0.5 in}{1 in}^^A
%         \color{black!20!white}\rule{0.5 in}{1 in}^^A
%       }
%     \begin{minipage}{0.4\textwidth}
%       \DisplayCoffinHandles\ExampleCoffin{blue}
%     \end{minipage}
%     \hfil
%     \begin{minipage}{0.4\textwidth}
%       \RotateCoffin\ExampleCoffin{45}
%       \DisplayCoffinHandles\ExampleCoffin{blue}
%     \end{minipage}
%     \vspace{3em}
%     \caption{旋转 Coffin:左图,未旋转;右图,已旋转$45$\textdegree。}
%     \label{fgr:rotation}
%   \end{figure}
%
% \newpage
%
% \section{{\heiti 调整} coffins {\heiti 的尺寸}}
%
% \begin{function}{\ResizeCoffin}
%   \begin{syntax}
%     \cs{ResizeCoffin} \meta{coffin} \Arg{width} \Arg{total-height}
%   \end{syntax}
%   将 \meta{coffin}\ 重新调整(resized)到 \meta{width} 和 \meta{total-height},
%   这两者都应作为尺寸表达式(dimension expressions)给出。
% \end{function}
%
% \begin{function}{\ScaleCoffin}
%   \begin{syntax}
%     \cs{ScaleCoffin} \meta{coffin} \Arg{x-scale} \Arg{y-scale}
%   \end{syntax}
%   在水平方向(horizontal directions)和垂直方向(vertical directions)分别
%   按 \meta{x-scale}\ 和 \meta{y-scale}\ 因子(factors)缩放(scale) \meta{coffin}。
%   这两个缩放因子(scale factors)应该作为实数(real numbers)给出。
% \end{function}
%
% \cs{ResizeCoffin}\ 和 \cs{ScaleCoffin}\ 可以互换使用:调整尺寸的最佳形式是
% 比例因子(scale factors)还是绝对值(absolute values)将取决于上下文(context)(图~\ref{fgr:resizing})。
% \vspace{1em}
% \begin{figure}[H]
%   \hfil
%   \SetHorizontalCoffin\ExampleCoffin
%     {^^A
%       \color{black!10!white}\rule{0.5 in}{1 in}^^A
%       \color{black!20!white}\rule{0.5 in}{1 in}^^A
%     }
%   \begin{minipage}{0.4\textwidth}
%     \ResizeCoffin \ExampleCoffin {4 cm} {3 cm}
%     \DisplayCoffinHandles \ExampleCoffin {blue}
%   \end{minipage}
%   \hfil
%   \begin{minipage}{0.4\textwidth}
%     \ScaleCoffin \ExampleCoffin {2.0} {0.5}
%     \DisplayCoffinHandles \ExampleCoffin {blue}
%   \end{minipage}
%   \vspace{2em}
%   \caption{调整 coffin 的尺寸:左图,调整尺寸为$4$\,cm乘以$6$\,cm;
%   右图,$x$ 和 $y$的比例因子分别为 $2$ 和 $0.5$(如图~\ref{fgr:rotation}~所示的 coffin 示例)。}
%   \label{fgr:resizing}
% \end{figure}
%
%
% \section{{\heiti 连接} coffins}
%
% coffins 的关键操作(key operation)是将 coffins 相互连接。执行此操作时,
% 第一个 coffin 始终是“父 coffin”,并通过对齐(alignment)进行更新。
% 第二个“子 coffin”不会因对齐过程(alignment process)而改变。
%
% \begin{function}{\JoinCoffins}
%   \begin{syntax}
%     \cs{JoinCoffins} *
%     ~~\meta{coffin1} [ \meta{coffin1-pole1} , \meta{coffin1-pole2} ]
%     ~~\meta{coffin2} [ \meta{coffin2-pole1} , \meta{coffin2-pole2} ]
%     ~~( \meta{x-offset} , \meta{y-offset} )
%   \end{syntax}
%
%   两个 coffins 的连接由 \cs{JoinCoffins}\ 函数执行,该函数接受两个强制(mandatory)参数:
%   “父”\meta{coffin1}\ 和“子”\meta{coffin2}。显示的所有其他参数都是可选的(optional)。
% \end{function}
%
%   标准的 \cs{JoinCoffins}\ 函数将 \meta{coffin2}\ 连接到 \meta{coffin1},
%   使得过程结束后的 \meta{coffin1}\ 的边界框(bounding box)扩大。
%   新的边界框将是覆盖两个输入 coffins 的边界框的最小矩形(rectangle)。
%   当使用 \cs{JoinCoffins}\ 的星号变体(starred variant)时,\meta{coffin1}\ 的
%   边界框不会改变,{\kaiti 即}~\meta{coffin2}\ 可能会突出于更新
%   的 \meta{coffin1}\ 的边界框之外。这两种对齐形式之间的区别最好用一个
%   可视化示例(visual example)来说明。在图~\ref{fgr:alignment}~中,
%   对两个过程(processes)进行了对比。在这两种情况下,红色的小 coffin 都与
%   灰色的大 coffin 对齐。在左边的插图中,使用了 \cs{JoinCoffins}\ 函数,
%   从而产生了一个扩展的边界框(expanded bounding box)。相比之下,
%   右边的插图使用了 \cs{AttachCoffin},这意味着边界框不包括较小 coffin 的区域。
% \vspace{1em}
%   \begin{figure}[H]
%     \fboxsep 0 pt\relax
%     \SetHorizontalCoffin\ExampleCoffin
%       {\color{black!20!white}\rule{1 in}{1 in}}
%     \SetHorizontalCoffin \SmallCoffin
%       {\color{red!20!white}\rule{0.1 in}{0.1 in}}
%     \hfil
%     \begin{minipage}{0.4\textwidth}
%       \centering
%       \JoinCoffins\ExampleCoffin[vc,r]\SmallCoffin[vc,l]
%       \fbox{\TypesetCoffin\ExampleCoffin}
%     \end{minipage}
%     \hfil
%     \begin{minipage}{0.4\textwidth}
%       \centering
%        \JoinCoffins*\ExampleCoffin[vc,r]\SmallCoffin[vc,l]
%       \fbox{\TypesetCoffin\ExampleCoffin}%
%     \end{minipage}
%     \vspace{2em}
%     \caption{\cs{JoinCoffins} (左图)和 \cs{JoinCoffins*} (右图)之间的对比;
%     coffin 的边界框(bounding box)以黑色显示。}
%     \label{fgr:alignment}
%   \end{figure}
%
%   通过首先计算 \meta{handle1},即 \meta{coffin1-pole1}\ 和 \meta{coffin1-pole2}\ 的交叉点,
%   以及 \meta{handle2},即 \meta{coffin2-pole1}\ 和 \meta{coffin2-pole2}\ 的交叉点,
%   来执行对齐(alignment)。如果这两个 \meta{poles}\ 未指定,\cs{JoinCoffins}\ 将
%   使用默认值 |(H,l)|,{\kaiti 即} \TeX\ 用于底层框(underlying box)的参考点(reference point)。
%   一旦找到了两个 \meta{handles},\meta{coffin2}\ 就会附着到 \meta{coffin1},
%   使得 \meta{handle1}\ 和 \meta{handle2}\ 之间的关系
%   由 \meta{x-offset}\ 和 \meta{y-offset}\ 来描述。这个 \meta{offset}\ 是可选参数,
%   如果未给出 \meta{offset},则使用 |(0 pt, 0 pt)|。
%
% 注意,当使用 \cs{JoinCoffins}\ 时,新的边界框(bounding box)是包含两个
% 输入 coffins 的边界框的最小矩形。因此,它将包括额外的空白(additional white space),
% 除非一个 coffin 完全重叠其他(图~\ref{fgr:bounding},左图)。在重新计算边界框时,
% coffins 的旋转将考虑旋转后材料的范围(extent)。这意味着在旋转时不会添加
% {\kaiti 不必要}(\emph{unnecessary})的空白(图~\ref{fgr:bounding},右图)。
%
% \vspace{1em}
% \begin{figure}[H]
%   \fboxsep 0 pt\relax
%   \SetHorizontalCoffin\ExampleCoffin
%     {\color{black!20!white}\rule{1 in}{1 in}}
%   \SetHorizontalCoffin\SmallCoffin
%     {\color{red!20!white}\rule{0.1 in}{0.1 in}}
%   \JoinCoffins\ExampleCoffin[vc,r]\SmallCoffin[vc,l]
%   \hfil
%   \begin{minipage}{0.4\textwidth}
%     \centering
%     \fbox{\copy\ExampleCoffin}
%   \end{minipage}
%   \hfil
%   \begin{minipage}{0.4\textwidth}
%     \centering
%     \RotateCoffin\ExampleCoffin{135}
%     \fbox{\copy\ExampleCoffin}
%   \end{minipage}
%   \vspace{2em}
%   \caption{已连接的 coffin 的旋转效果:黑线显示 coffin 的边界框。}
%   \label{fgr:bounding}
% \end{figure}
%
% 作为连接过程(joining procedure)的一部分,两个输入 coffins 的杆(poles)被保存在
% 更新 coffin 的结构(structure)内。这样就有可能执行复杂的对齐程序(alignment procedures)。
% 因此,对齐后的 coffin 的杆可分为三组(groups):
% \begin{enumerate}
%   \item 更新 coffin 的“原始(native)”杆,如 \texttt{l}、\texttt{r}、\texttt{hc}\ {\kaiti 等}。
%   \item 源自 \meta{coffin1}\ 的杆,如 \texttt{\meta{coffin1}-l}、\texttt{\meta{coffin1}-r}、
%   \texttt{\meta{coffin1}-hc}\ {\kaiti 等}。
%   \item 源自 \meta{coffin2}\ 的杆,如 \texttt{\meta{coffin2}-l}、\texttt{\meta{coffin2}-r}、
%   \texttt{\meta{coffin2}-hc}\ {\kaiti 等}。
% \end{enumerate}
%
% 应用此功能可以进行一系列连接操作(joining operations),如图~\ref{fgr:nested}~所示。
% 在本例中,用于对齐的方案(scheme)如下:
% \begin{verbatim}
%   \SetHorizontalCoffin\OutputCoffin{}
%   \SetHorizontalCoffin\RedCoffin
%     {\color{red!20!white}\rule{0.2 in}{0.2 in}}
%   \JoinCoffins\OutputCoffin[vc,hc]\RedCoffin[vc,hc]
%   \SetHorizontalCoffin\BlueCoffin
%     {\color{blue!20!white}\rule{0.2 in}{0.2 in}}
%   \JoinCoffins\OutputCoffin[\RedCoffin-vc,\RedCoffin-hc]
%     \BlueCoffin[b,l]
%   \SetHorizontalCoffin\GreenCoffin
%     {\color{green!20!white}\rule{0.2 in}{0.2 in}}
%   \JoinCoffins\OutputCoffin[\BlueCoffin-vc,\BlueCoffin-hc]
%     \GreenCoffin[b,l]
%   \SetHorizontalCoffin\YellowCoffin
%     {\color{yellow!20!white}\rule{0.2 in}{0.2 in}}
%   \JoinCoffins\OutputCoffin[\GreenCoffin-vc,\GreenCoffin-hc]
%     \YellowCoffin[b,l]
%   \SetHorizontalCoffin \OrangeCoffin
%     {\color{orange!20!white}\rule{0.2 in}{0.2 in}}
%   \JoinCoffins\OutputCoffin[\BlueCoffin-t,\BlueCoffin-l]
%     \OrangeCoffin[b,r]
%   \TypesetCoffin\OutputCoffin
% \end{verbatim}
% 此过程首先设置 \cs{OutputCoffin}\ 以保存连接的输出(joined output)。然后,
% 每个连接都会发生相对于前一个连接放置新添加项(new addition)的过程。
% 由于每个连接的 coffin 都有一个唯一的名称,因此可以相对于组件(assembly)的
% 每一个组成部分(component parts)进行对齐。这可以通过基于 \cs{BlueCoffin}\ 的
% 早期位置(earlier placement)添加最终的 \cs{OrangeCoffin}\ 来说明。
%
% \vspace{1em}
% \begin{figure}[H]
%   \centering
%   \SetHorizontalCoffin\OutputCoffin{}
%   \SetHorizontalCoffin\RedCoffin
%     {\color{red!20!white}\rule{0.2 in}{0.2 in}}
%   \JoinCoffins\OutputCoffin[vc,hc]\RedCoffin[vc,hc]
%   \SetHorizontalCoffin\BlueCoffin
%     {\color{blue!20!white}\rule{0.2 in}{0.2 in}}
%   \JoinCoffins\OutputCoffin[\RedCoffin-vc,\RedCoffin-hc]
%     \BlueCoffin[b,l]
%   \SetHorizontalCoffin\GreenCoffin
%     {\color{green!20!white}\rule{0.2 in}{0.2 in}}
%   \JoinCoffins\OutputCoffin[\BlueCoffin-vc,\BlueCoffin-hc]
%     \GreenCoffin[b,l]
%   \SetHorizontalCoffin\YellowCoffin
%     {\color{yellow!20!white}\rule{0.2 in}{0.2 in}}
%   \JoinCoffins\OutputCoffin[\GreenCoffin-vc,\GreenCoffin-hc]
%     \YellowCoffin[b,l]
%   \SetHorizontalCoffin \OrangeCoffin
%     {\color{orange!20!white}\rule{0.2 in}{0.2 in}}
%   \JoinCoffins\OutputCoffin[\BlueCoffin-t,\BlueCoffin-l]
%     \OrangeCoffin[b,r]
%   \TypesetCoffin\OutputCoffin
%   \vspace{1em}
%   \caption{使用以前操作中的杆(poles)对齐 coffins。}
%   \label{fgr:nested}
% \end{figure}
%
%
% \section{{\heiti 排版} coffins}
%
% \begin{function}{\TypesetCoffin}
%   \begin{syntax}
%     \cs{TypesetCoffin}
%     ~~\meta{coffin} [ \meta{pole1} , \meta{pole2} ]
%     ~~( \meta{x-offset} , \meta{y-offset} )
%   \end{syntax}
%   通过首先计算 \meta{handle},即 \meta{pole1}\ 和 \meta{pole2}\ 的交叉点来排版。
%   这是一个可选参数,如果未给出,则使用 |(H,l)|,即
%   底层框(underlying box)的 \TeX{}\ 参考点(reference point)。然后对 coffin 进行排版,
%   以使文档中的当前参考点与 \meta{handle}\ 由 \meta{x-offset}\ 和 \meta{y-offset}\ 描述。
%   这 \meta{offset}\ 是可选的,如果未给出 \meta{offset},则使用 |(0pt, 0pt)|。
%   因此,排版 coffin 类似于执行一个对齐(alignment),
%   其中“父(parent)” coffin 是当前的插入点(insertion point)。
% \end{function}
%
%
% \section{{\heiti 测量} coffins}
%
% 在设计过程(design process)中,有些地方可以在杆设置程序(pole-setting procedures)
% 之外测量 coffins。
%
% \begin{function}{\CoffinDepth}
%   \begin{syntax}
%     \cs{CoffinDepth} \meta{coffin}
%   \end{syntax}
%   以适合在 \meta{dimension expression}\ 中使用的形式计算 \meta{coffin}\ 的深度(depth,低于基线),
%   例如,|\setlength{\mylength}{\CoffinDepth\ExampleCoffin}|。
% \end{function}
%
% \begin{function}{\CoffinHeight}
%   \begin{syntax}
%     \cs{CoffinHeight} \meta{coffin}
%   \end{syntax}
%   以适合在 \meta{dimension expression}\ 中使用的形式计算 \meta{coffin}\ 的高度(height,高于基线),
%   例如,|\setlength{\mylength}{\CoffinHeight\ExampleCoffin}|。
% \end{function}
%
% \begin{function}{\CoffinTotalHeight}
%   \begin{syntax}
%     \cs{CoffinTotalHeight} \meta{coffin}
%   \end{syntax}
%   以适合在 \meta{dimension expression}\ 中使用的形式计算 \meta{coffin}\ 的总高度(total height),
%   例如,|\setlength{\mylength}{\CoffinTotalHeight\ExampleCoffin}|。
% \end{function}
%
% \begin{function}{\CoffinWidth}
%   \begin{syntax}
%     \cs{CoffinWidth} \meta{coffin}
%   \end{syntax}
%   以适合在 \meta{dimension expression}\ 中使用的形式计算 \meta{coffin}\ 的宽度(width),
%   例如,|\setlength{\mylength}{\CoffinWidth\ExampleCoffin}|。
% \end{function}
%
%
% \section{\heiti 诊断函数}
%
% 用于跟踪(following) coffin 建造过程(coffin-building process)的诊断数据(diagnostic data)
% 可通过图形(graphically)和终端(terminal)获得。这反映了 coffins 是视觉构造(visual constructs)的事实。
%
% \begin{function}{\DisplayCoffinHandles}
%   \begin{syntax}
%     \cs{DisplayCoffinHandles} \meta{coffin} \Arg{color}
%   \end{syntax}
%   这个函数首先计算 \meta{coffin}\ 的所有 \meta{poles}\ 之间的交点,给出一组 \meta{handles}。
%   然后,在源的当前位置(current location in the source)打印 \meta{coffin},
%   并将 \meta{handles}\ 的位置(position) 标记在 coffin 上。
%   \meta{handles}\ 将被标记为这个过程(process)的一部分: \meta{handles}\ 的
%   位置(locations)和标签(labels)都以指定的 \meta{color}\ 被打印。
% \end{function}
%
% \begin{function}{\MarkCoffinHandle}
%   \begin{syntax}
%     \cs{MarkCoffinHandle} \meta{coffin}
%     ~~[ \meta{pole1} , \meta{pole2} ] \Arg{color}
%   \end{syntax}
%   这个函数首先计算由 \meta{pole1}\ 和 \meta{pole2}\ 的交点所定义的 \meta{coffin}\ 的 \meta{handle}。
%   然后标记出 \meta{handle}\ 在 \meta{coffin}\ 上的位置。\meta{handle}\ 将被标记为此过程的一部分:
%   \meta{handle}\ 的位置(location)和标签(label)都以指定的 \meta{color}\ 被打印。
%   如果没有给出 \meta{poles},则使用缺省值 |(H,l)|。
% \end{function}
%
% \begin{function}{\ShowCoffinStructure}
%   \begin{syntax}
%     \cs{ShowCoffinStructure} \meta{coffin}
%   \end{syntax}
%   这个函数在终端(terminal)显示了 \meta{coffin}\ 的结构信息(structural information)。
%   给出了排版材料(typeset material)的宽度(width)、高度(height)和深度(depth),
%   以及 coffin 所有杆(location)的位置。例如,对于图~\ref{fgr:rotation}~中的
%   已旋转的(rotated) coffin,\cs{ShowCoffinStructure}\ 的输出是:
%   \begin{verbatim}
%     Size of coffin \ExampleCoffin:
%     > ht = 72.26999pt
%     > dp = 0.0pt
%     > wd = 72.26999pt
%     Poles of coffin \ExampleCoffin:
%     >  l  =>  {0pt}{0pt}{0pt}{1000pt}
%     >  B  =>  {0pt}{0pt}{1000pt}{0pt}
%     >  H  =>  {0pt}{0pt}{1000pt}{0pt}
%     >  T  =>  {0pt}{0pt}{1000pt}{0pt}
%     >  hc  =>  {36.135pt}{0pt}{0pt}{1000pt}
%     >  r  =>  {72.26999pt}{0pt}{0pt}{1000pt}
%     >  vc  =>  {0pt}{36.135pt}{1000pt}{0pt}
%     >  t  =>  {0pt}{72.26999pt}{1000pt}{0pt}
%     >  b  =>  {0pt}{0.0pt}{1000pt}{0pt}.
%     <recently read> }
%   \end{verbatim}
%   请注意 coffin 的杆(poles)由四个值定义:杆通过的点的 $x$ 和 $y$ 坐标(co-ordinates)
%   以及表示杆方向的向量(vector)的 $x$-分量和$y$-分量。决定杆方向的是后者之间的比值(ratio),
%   而不是绝对值(absolute values)。
% \end{function}
%
% \end{documentation}
%
% \begin{implementation}
%
% \newpage
%
% \section{\heiti 实现}
%
%    \begin{macrocode}
%<*package>
%    \end{macrocode}
%
%    \begin{macrocode}
%<@@=coffin>
%    \end{macrocode}
%
%    \begin{macrocode}
\ProvidesExplPackage{xcoffins}{2024-01-04}{}
  {L3 Experimental design level coffins}
%    \end{macrocode}
%
% \begin{variable}
%   {
%     \l_@@_A_hpole_tl          ,
%     \l_@@_A_vpole_tl          ,
%     \l_@@_B_hpole_tl          ,
%     \l_@@_B_vpole_tl          ,
%     \l_@@_bound_box_grow_bool ,
%     \l_@@_hoffset_dim         ,
%     \l_@@_voffset_dim
%   }
%   对齐系统(alignment system)的键值定义(Key--value definitions)。除了 \texttt{grow-bounding-box}\ 之外,
%   所有这些都必须给定一个值。
%    \begin{macrocode}
\keys_define:nn { coffin }
{
    coffin1-hpole     .tl_set:N         = \l_@@_A_hpole_tl          ,
    coffin1-hpole     .value_required:n = true                      ,
    coffin1-vpole     .tl_set:N         = \l_@@_A_vpole_tl          ,
    coffin1-vpole     .value_required:n = true                      ,
    coffin2-hpole     .tl_set:N         = \l_@@_B_hpole_tl          ,
    coffin2-hpole     .value_required:n = true                      ,
    coffin2-vpole     .tl_set:N         = \l_@@_B_vpole_tl          ,
    coffin2-vpole     .value_required:n = true                      ,
    grow-bounding-box .bool_set:N       = \l_@@_bound_box_grow_bool ,
    grow-bounding-box .default:n        = true                      ,
    hoffset           .dim_set:N        = \l_@@_hoffset_dim         ,
    hoffset           .value_required:n = true                      ,
    voffset           .dim_set:N        = \l_@@_voffset_dim         ,
    voffset           .value_required:n = true
}
\keys_set:nn { coffin }
{
    coffin1-hpole     = H    ,
    coffin1-vpole     = l    ,
    coffin2-hpole     = H    ,
    coffin2-vpole     = l    ,
    grow-bounding-box = true ,
    hoffset           = 0 pt ,
    voffset           = 0 pt
}
%    \end{macrocode}
% \end{variable}
%
% \begin{macro}{\@@_design_names:N}
% \begin{variable}{\Height, \Depth, \Width, \TotalHeight}
% \begin{variable}
%   {\l_@@_height_dim, \l_@@_depth_dim, \l_@@_width_dim, \l_@@_totalheight_dim}
%   为各种 coffin 尺寸(dimensions)设置设计级别的名称(design-level names)。
%   它们不是在这个范围(scope)之外定义的,而是尺寸,
%   以便它们在例如 \cs{fp_eval:n}\ 这样的内部能正确工作(work correctly)。
%    \begin{macrocode}
\cs_new_protected:Npn \@@_design_names:N #1
{
    \dim_set:Nn \l_@@_height_dim { \coffin_ht:N #1 }
    \dim_set:Nn \l_@@_depth_dim  { \coffin_dp:N #1 }
    \dim_set:Nn \l_@@_width_dim  { \coffin_wd:N #1 }
    \dim_set:Nn \l_@@_totalheight_dim
    { \l_@@_height_dim + \l_@@_depth_dim }
    \cs_set_eq:NN \Height \l_@@_height_dim
    \cs_set_eq:NN \Depth \l_@@_depth_dim
    \cs_set_eq:NN \Width \l_@@_width_dim
    \cs_set_eq:NN \TotalHeight \l_@@_totalheight_dim
}
\dim_new:N \l_@@_height_dim
\dim_new:N \l_@@_depth_dim
\dim_new:N \l_@@_width_dim
\dim_new:N \l_@@_totalheight_dim
%    \end{macrocode}
% \end{variable}
% \end{variable}
% \end{macro}
%
% 其中大部分或多或少只是直接传递数据(passing data straight through)。
%
% \begin{macro}{\NewCoffin}
% 这是一个非常简单的转换(conversion)。
%    \begin{macrocode}
\NewDocumentCommand \NewCoffin { m }
{ \coffin_new:N #1 }
%    \end{macrocode}
% \end{macro}
%
% \begin{macro}{\SetHorizontalCoffin}
% \begin{macro}{\SetVerticalCoffin}
%   这些都是直接的翻译(straight-forward translations)。
%    \begin{macrocode}
\NewDocumentCommand \SetHorizontalCoffin { m +m }
{ \hcoffin_set:Nn #1 {#2} }
\NewDocumentCommand \SetVerticalCoffin { m m +m }
{ \vcoffin_set:Nnn #1 {#2} {#3} }
%    \end{macrocode}
% \end{macro}
% \end{macro}
%
% \begin{macro}{\SetHorizontalPole}
% \begin{macro}{\SetVerticalPole}
%   这里,需要为 coffin 尺寸设置设计级别名称(design-level names)。这需要分组(grouping),
%   但 coffin 工作(work)必须在组之外(outside of the group)进行。因此,
%   这里有一点展开技巧(expansion trickery)。
%    \begin{macrocode}
\NewDocumentCommand \SetHorizontalPole { m m m }
  {
    \group_begin:
      \@@_design_names:N #1
    \use:e
      {
        \group_end:
        \coffin_set_horizontal_pole:Nnn #1
          { \exp_not:n {#2} } { \dim_eval:n {#3} }
      }
  }
\NewDocumentCommand \SetVerticalPole { m m m }
  {
    \group_begin:
      \@@_design_names:N #1
    \use:e
      {
        \group_end:
        \coffin_set_vertical_pole:Nnn #1
          { \exp_not:n {#2} } { \dim_eval:n {#3} }
      }
  }
%    \end{macrocode}
% \end{macro}
% \end{macro}
%
% \begin{macro}{\JoinCoffins}
%   \cs{JoinCoffins}\ 函数需要对输入语法(input syntax)做一些工作,
%   因为有许多可选参数(optional arguments)需要考虑。这里的想法是 \cs{JoinCoffins}\ 既
%   可以用来扩展 \meta{coffin1}\ 的边界框,也可以在不扩展边界框的情况下添加 \meta{coffin2}。
%   还有两个手柄位置(handle positions)和偏移量(offset)要排序。
%    \begin{macrocode}
\NewDocumentCommand \JoinCoffins
  {
    o
    s
    m
    > { \SplitArgument { 1 } { , } } O { H , l }
    m
    > { \SplitArgument { 1 } { , } } O { H , l }
    > { \SplitArgument { 1 } { , } } D ( ) { 0 pt , 0 pt }
  }
  {
    \IfNoValueTF {#1}
      {
        \IfBooleanTF #2
          { \coffin_attach:NnnNnnnn #3 #4 #5 #6 #7 }
          { \coffin_join:NnnNnnnn #3 #4 #5 #6 #7 }
      }
      {
        \group_begin:
          \keys_set:nn { coffin } {#1}
          \tl_set:Ne \l_@@_tmp_tl
            {
              \group_end:
              \bool_if:NTF \l_@@_bound_box_grow_bool
                { \coffin_join:NnnNnnnn }
                { \coffin_attach:NnnNnnnn }
              \exp_not:N #3
              { \exp_not:o { \l_@@_A_hpole_tl } }
              { \exp_not:o { \l_@@_A_vpole_tl } }
              \exp_not:N #5
              { \exp_not:o { \l_@@_B_hpole_tl } }
              { \exp_not:o { \l_@@_B_vpole_tl } }
              { \dim_use:N \l_@@_hoffset_dim }
              { \dim_use:N \l_@@_voffset_dim }
            }
        \l_@@_tmp_tl
      }
  }
%    \end{macrocode}
% \end{macro}
%
% \begin{macro}{\TypesetCoffin}
%   对于 coffins 的排版,有两个可选参数,这两个参数都需要拆分(split)。
%   这是 \cs{JoinCoffins}\ 所需代码的一个更简单的例子。
%    \begin{macrocode}
\NewDocumentCommand \TypesetCoffin
{
    m
    > { \SplitArgument { 1 } { , } } O { H , l }
    > { \SplitArgument { 1 } { , } } D ( ) { 0 pt , 0 pt }
}
{ \coffin_typeset:Nnnnn #1 #2 #3 }
%    \end{macrocode}
% \end{macro}
%
% \begin{macro}{\RotateCoffin}
% \begin{macro}{\ResizeCoffin}
% \begin{macro}{\ScaleCoffin}
%   更多直接拷贝(straight-forward copies)。
%    \begin{macrocode}
\NewDocumentCommand \RotateCoffin  { m m }
{ \coffin_rotate:Nn #1 {#2} }
\NewDocumentCommand \ResizeCoffin  { m m m }
{ \coffin_resize:Nnn #1 {#2} {#3} }
\NewDocumentCommand \ScaleCoffin  { m m m }
{ \coffin_scale:Nnn #1 {#2} {#3} }
%    \end{macrocode}
% \end{macro}
% \end{macro}
% \end{macro}
%
% \begin{macro}{\CoffinDepth, \CoffinHeight, \CoffinTotalHeigth, \CoffinWidth}
%   没有什么太复杂的,除了将总高度(total height)设置为表达式(expression),
%   以便在前缀(prefixed)为负号(negative sign){\kaiti 等}时能正确操作。
%    \begin{macrocode}
\NewDocumentCommand \CoffinDepth { m }
{ \dim_eval:n { \coffin_dp:N #1 } }
\NewDocumentCommand \CoffinHeight { m }
{ \dim_eval:n { \coffin_ht:N #1 } }
\NewDocumentCommand \CoffinTotalHeight { m }
{ \dim_eval:n { \coffin_ht:N #1 + \coffin_dp:N #1 } }
\NewDocumentCommand \CoffinWidth { m }
{ \dim_eval:n { \coffin_wd:N #1 } }
%    \end{macrocode}
% \end{macro}
%
% \begin{macro}{\DisplayCoffinHandles}
%   显示所有手柄(handles)稍微容易一点,因为不必担心手柄。
%    \begin{macrocode}
\NewDocumentCommand \DisplayCoffinHandles { m m }
{ \coffin_if_exist:NT #1 { \coffin_display_handles:Nn #1 {#2} } }
%    \end{macrocode}
% \end{macro}
%
% \begin{macro}{\MarkCoffinHandle}
%   标记手柄(marking a handle)需要对输入(input)进行一些处理,
%   这样设计级别的接口(design-level interface)就“漂亮(nice)”了。
%    \begin{macrocode}
\NewDocumentCommand \MarkCoffinHandle
{ m > { \SplitArgument { 1 } { , } } O { H , l } m }
{  \coffin_if_exist:NT #1 { \coffin_mark_handle:Nnnn #1 #2 {#3} } }
%    \end{macrocode}
% \end{macro}
%
% \begin{macro}{\ShowCoffinStructure}
%   再次回到易于实现的函数(easy-to-implement functions)。
%    \begin{macrocode}
\NewDocumentCommand \ShowCoffinStructure { m }
{ \coffin_show_structure:N #1 }
%    \end{macrocode}
% \end{macro}
%
%    \begin{macrocode}
%</package>
%    \end{macrocode}
%
% \end{implementation}
% \newpage
% \PrintIndex
\endinput