% \iffalse meta-comment
%
% Copyright (C) 1993-2023
% The LaTeX Project and any individual authors listed elsewhere
% in this file.
%
% This file is part of the LaTeX base system.
% -------------------------------------------
%
% It may be distributed and/or modified under the
% conditions of the LaTeX Project Public License, either version 1.3c
% of this license or (at your option) any later version.
% The latest version of this license is in
%    https://www.latex-project.org/lppl.txt
% and version 1.3c or later is part of all distributions of LaTeX
% version 2008 or later.
%
% This file has the LPPL maintenance status "maintained".
%
% The list of all files belonging to the LaTeX base distribution is
% given in the file `manifest.txt'. See also `legal.txt' for additional
% information.
%
% The list of derived (unpacked) files belonging to the distribution
% and covered by LPPL is defined by the unpacking scripts (with
% extension .ins) which are part of the distribution.
%
% \fi
%
% \iffalse
% \section{Identification}
%
%    These document classes can only be used with \LaTeXe, so we make
%    sure that an appropriate message is displayed when another \TeX{}
%    format is used.
% \changes{v1.3p}{1995/11/30}{Added date of \LaTeX\ format to argument
%    of \cs{NeedsTeXFormat}}
%    \begin{macrocode}
%<article|report|book>\NeedsTeXFormat{LaTeX2e}[1995/12/01]
%    \end{macrocode}
%
%    Announce the Class name and its version:
%    \begin{macrocode}
%<article>\ProvidesClass{article}
%<report>\ProvidesClass{report}
%<book>\ProvidesClass{book}
%<10pt&!bk>\ProvidesFile{size10.clo}
%<11pt&!bk>\ProvidesFile{size11.clo}
%<12pt&!bk>\ProvidesFile{size12.clo}
%<10pt&bk>\ProvidesFile{bk10.clo}
%<11pt&bk>\ProvidesFile{bk11.clo}
%<12pt&bk>\ProvidesFile{bk12.clo}
%<*driver>
\ProvidesFile{classes.drv}
%</driver>
              [2023/05/17 v1.4n
%<article|report|book> Standard LaTeX document class]
%<10pt|11pt|12pt>      Standard LaTeX file (size option)]
%    \end{macrocode}
%
% \section{A driver for this document}
%
% The next bit of code contains the documentation driver file for
% \TeX{}, i.e., the file that will produce the documentation you are
% currently reading. It will be extracted from this file by the
% {\sc docstrip} program.
%
% \changes{1.0f}{1993/12/07}{Use class ltxdoc document class}
% \changes{1.0r}{1994/02/28}{Moved driver code in order not to need a
%    separate driver}
%    \begin{macrocode}
%<*driver>
]
\documentclass{ltxdoc}



\usepackage[T1]{fontenc}
\usepackage{tabularray}
\usepackage{caption}

%%%%%%%%%%%%% 以下设置中文字体 %%%%%%%%%%%%%%%%%%%%%%%%%%%%%%%%%%%%%%%%%
\usepackage[fontset=source]{ctex}

\setCJKfamilyfont{Heiti}{Source Han Sans Regular} %%%% 自定义\Heiti命令,显示思源黑体,用于标题页标题的中文部分
\newcommand{\Heiti}{\CJKfamily{Heiti}} %%%% 自定义\Heiti命令,显示思源黑体,用于标题页标题的中文部分

\setCJKfamilyfont{heiti}{Source Han Sans Light} %%自定义\heiti命令,显示思源黑体,用于正文的章节标题
\newcommand{\heiti}{\CJKfamily{heiti}} %%自定义\heiti命令,显示思源黑体,用于正文的章节标题

\setCJKfamilyfont{songti}{思源宋体 CN Light}  %%%% 自定义\songti命令,显示思源宋体,用于正文
\newcommand{\songti}{\CJKfamily{songti}} %%%% 自定义\songti命令,显示思源宋体,用于正文

\setCJKmainfont{思源宋体 CN Light} %%%% 设置中文的主字体为思源宋体 CN Light
%\setmainfont{Source Serif Pro} %%%% 设置英文的主字体为Source Serif Pro,也可以设为Times New Roman

\setCJKfamilyfont{kaiti}{KaiTi} %%设置中文字体楷体,用于强调
\newcommand{\kaiti}{\CJKfamily{kaiti}} %%设置中文字体楷体,用于强调
%%%%%%%%%%%%% 以上设置中文字体 %%%%%%%%%%%%%%%%%%%%%%%%%%%%%%%%%%%%%%%%%

%%%%%%%%%%%%% 以下设置中文版式 %%%%%%%%%%%%%%%%%%%%%%%%%%%%%%%%%%%%%%%%%
\usepackage{indentfirst} %%% 首行缩进
\setlength{\parindent}{2em} %%% 缩进2个字符(中文为2个字)
\linespread{1.35} %%% 设置行间距
%%%%%%%%%%%%% 以上设置中文版式 %%%%%%%%%%%%%%%%%%%%%%%%%%%%%%%%%%%%%%%%%


%%%%%%%%%%%% 以下设置书签、目录 %%%%%%%%%%%%%%%%%%%%%%%%%%%%%%%%%%%%%%%%
\usepackage{xcolor}
\usepackage[colorlinks=true,linkcolor=red]{hyperref}
%%%%%%%%%%%% 以上设置书签、目录 %%%%%%%%%%%%%%%%%%%%%%%%%%%%%%%%%%%%%%%%%


%%%%%%% 以下在 tabular 表格中定制 横线如\hlinew{1.2pt} %%%%%%
\makeatletter
\def\hlinew#1{%
\noalign{\ifnum0=`}\fi\hrule \@height #1 \futurelet
\reserved@a\@xhline}
\makeatother%
%%%%%%% 以上在 tabular 表格中定制 横线如\hlinew{1.2pt} %%%%%%

%%%%%%% 以下自定义脚注 %%%%%%%%%%%%%%%%%%%%%%%%%%%%%%%%%%%%
\setlength{\footnotesep}{0.5cm} %%%设置几第脚注之间的距离
\setlength{\skip\footins}{3em} %%%设置脚注与正文之间的距离
%%\renewcommand\footnoterule{} %%%定义脚注线为空
\renewcommand\footnoterule{
     \kern -3pt                         % This -3 is negative
     \hrule width 0.6\textwidth height 0.6pt % of the sum of this 1
     \kern 2pt} %%%
%%%%%%% 以上自定义脚注 %%%%%%%%%%%%%%%%%%%%%%%%%%%%%%%%%%%%

%%%%%%% 以下制作索引 %%%%%%%%%%%%%%%%%%%%%%%%%%%%%%%%%%%%
\usepackage {makeidx}
\makeindex
%%%%%%% 以上制作索引 %%%%%%%%%%%%%%%%%%%%%%%%%%%%%%%%%%%%


\renewcommand{\contentsname}{\heiti 目\ 录}   %%% 在{document}后面加入该命令,将"contents"变成“目  录”
\renewcommand{\refname}{\heiti 参考文献}
\renewcommand{\tablename}{表}
\renewcommand{\abstractname}{\heiti 摘\ 要}


%
%%    \end{macrocode}
%
%    We don't want everything to appear in the index
%    \begin{macrocode}
\DoNotIndex{\',\.,\@M,\@@input,\@Alph,\@alph,\@addtoreset,\@arabic}
\DoNotIndex{\@badmath,\@centercr,\@cite}
\DoNotIndex{\@dotsep,\@empty,\@float,\@gobble,\@gobbletwo,\@ignoretrue}
\DoNotIndex{\@input,\@ixpt,\@m,\@minus,\@mkboth}
\DoNotIndex{\@ne,\@nil,\@nomath,\@plus,\roman,\@set@topoint}
\DoNotIndex{\@tempboxa,\@tempcnta,\@tempdima,\@tempdimb}
\DoNotIndex{\@tempswafalse,\@tempswatrue,\@viipt,\@viiipt,\@vipt}
\DoNotIndex{\@vpt,\@warning,\@xiipt,\@xipt,\@xivpt,\@xpt,\@xviipt}
\DoNotIndex{\@xxpt,\@xxvpt,\\,\ ,\addpenalty,\addtolength,\addvspace}
\DoNotIndex{\advance,\ast,\begin,\begingroup,\bfseries,\bgroup,\box}
\DoNotIndex{\bullet}
\DoNotIndex{\cdot,\cite,\CodelineIndex,\cr,\day,\DeclareOption}
\DoNotIndex{\def,\DisableCrossrefs,\divide,\DocInput,\documentclass}
\DoNotIndex{\DoNotIndex,\egroup,\ifdim,\else,\fi,\em,\endtrivlist}
\DoNotIndex{\EnableCrossrefs,\end,\end@dblfloat,\end@float,\endgroup}
\DoNotIndex{\endlist,\everycr,\everypar,\ExecuteOptions,\expandafter}
\DoNotIndex{\fbox}
\DoNotIndex{\filedate,\filename,\fileversion,\fontsize,\framebox,\gdef}
\DoNotIndex{\global,\halign,\hangindent,\hbox,\hfil,\hfill,\hrule}
\DoNotIndex{\hsize,\hskip,\hspace,\hss,\if@tempswa,\ifcase,\or,\fi,\fi}
\DoNotIndex{\ifhmode,\ifvmode,\ifnum,\iftrue,\ifx,\fi,\fi,\fi,\fi,\fi}
\DoNotIndex{\input}
\DoNotIndex{\jobname,\kern,\leavevmode,\let,\leftmark}
\DoNotIndex{\list,\llap,\long,\m@ne,\m@th,\mark,\markboth,\markright}
\DoNotIndex{\month,\newcommand,\newcounter,\newenvironment}
\DoNotIndex{\NeedsTeXFormat,\newdimen}
\DoNotIndex{\newlength,\newpage,\nobreak,\noindent,\null,\number}
\DoNotIndex{\numberline,\OldMakeindex,\OnlyDescription,\p@}
\DoNotIndex{\pagestyle,\par,\paragraph,\paragraphmark,\parfillskip}
\DoNotIndex{\penalty,\PrintChanges,\PrintIndex,\ProcessOptions}
\DoNotIndex{\protect,\ProvidesClass,\raggedbottom,\raggedright}
\DoNotIndex{\refstepcounter,\relax,\renewcommand}
\DoNotIndex{\rightmargin,\rightmark,\rightskip,\rlap,\rmfamily}
\DoNotIndex{\secdef,\selectfont,\setbox,\setcounter,\setlength}
\DoNotIndex{\settowidth,\sfcode,\skip,\sloppy,\slshape,\space}
\DoNotIndex{\symbol,\the,\trivlist,\typeout,\tw@,\undefined,\uppercase}
\DoNotIndex{\usecounter,\usefont,\usepackage,\vfil,\vfill,\viiipt}
\DoNotIndex{\viipt,\vipt,\vskip,\vspace}
\DoNotIndex{\wd,\xiipt,\year,\z@}
%    \end{macrocode}
%    We do want an index, using line numbers
%    \begin{macrocode}
\EnableCrossrefs
\CodelineIndex
%    \end{macrocode}
%    We use so many \file{docstrip} modules that we set the
%    \texttt{StandardModuleDepth} counter to 1.
%    \begin{macrocode}
\setcounter{StandardModuleDepth}{1}
%    \end{macrocode}
%    The following command retrieves the date and version information
%    from the file.
%    \begin{macrocode}
\GetFileInfo{classes.drv}
%    \end{macrocode}
%    Some commonly used abbreviations
% \changes{v1.2w}{1994/12/01}{Use \cs{newcommand*}}
%    \begin{macrocode}
\newcommand*{\Lopt}[1]{\textsf {#1}}
\newcommand*{\file}[1]{\texttt {#1}}
\newcommand*{\Lcount}[1]{\textsl {\small#1}}
\newcommand*{\pstyle}[1]{\textsl {#1}}
%    \end{macrocode}
%    We also want the full details.
%    \begin{macrocode}
\begin{document}
\DocInput{classes-zh-cn.dtx}
\PrintIndex
% ^^A\PrintChanges
\end{document}
%</driver>
%    \end{macrocode}
%
% \fi
%
% \changes{v1.0d}{1993/11/30}{remove \cs{@in}, made option makeindex
%    a synonym for option makeidx}
% \changes{v1.0d}{1993/11/30}{removed \cs{@minus}, \cs{@plus},
%    \cs{@settopoint}, \cs{@setfontsize}; they are now in the
%    kernel}
% \changes{v1.0d}{1993/11/30}{Added use of \cs{NeedsTeXFormat}}
% \changes{v1.0d}{1993/11/30}{Replaced \cs{bf} with \cs{bfseries};
%    \cs{rm} with \cs{rmfamily}}
% \changes{v1.0d}{1993/11/30}{Made equation and eqnarray environments
%    in the fleqn option up to date with latex.dtx}
% \changes{v1.0f}{1993/12/08}{Made all lines shorter than 72 characters}
% \changes{v1.0g}{1993/12/08}{Made change in eqnarray for the fleqn
%    option, as suggested by Rainer.}
% \changes{v1.0h}{1993/12/18}{Made the definitions of the font- and
%    size-changing commands use \cs{renew} rather than \cs{new}.
%    Defined the float parameters with \cs{renewcommand} rather than
%    \cs{newcommand}.  Corrected some typos in the fleqn option.
%    Replaced two occurrences of -\cs{@secpenalty} by
%    \cs{@secpenalty}.  ASAJ.}
% \changes{v1.0j}{1993/12/20}{Added \cs{ProvidesFile} to size files}
% \changes{v1.0j}{1993/12/10}{Use \cs{cmd} in change entries}
% \changes{v1.0k}{1994/01/09}{Removed some typos/bugs}
% \changes{v1.0l}{1994/01/11}{add the extension to the names of the
%     files}
% \changes{v1.0l}{1994/01/10}{Changed version numbering; moved leqno
%    and fleqn options to an external file.}
% \changes{v1.0n}{1994/01/19}{Removed code for makeidx option and made
%    it a separate package; removed use of \cs{setlength} from list
%    parameters.}
% \changes{v1.0o}{1994/01/31}{Small documentation changes}
% \changes{v1.0q}{1994/02/16}{Small documentation changes}
% \changes{v1.1a}{1994/03/12}{Removed \cs{typeout} messages}
% \changes{v1.1f}{1994/04/15}{Inserted forgotten line break}
% \changes{v1.2a}{1994/03/17}{Added openright option. (LL)}
% \changes{v1.2b}{1994/03/17}{Added the \ldots{}matter commands. (LL)}
% \changes{v1.2c}{1994/03/17}{Fixed page numbering in titlepage
%    env. (LL)}
% \changes{v1.2d}{1994/04/11}{Checked the file for long lines and
%    wrapped them when necessary; made a slight implementation
%    modification to the openright and openany options.}
% \changes{v1.2i}{1994/04/28}{Use LaTeX instead of LaTeX2e in messages}
% \changes{v1.2j}{1994/05/01}{Removed the use of \cs{fileversion}
%    c.s.}
% \changes{v1.2l}{1994/05/11}{changed some \cs{changes} entries}
% \changes{v1.2m}{1994/05/12}{Forgot a few entries}
% \changes{v1.2o}{1994/05/24}{Changed file information}
% \changes{v1.2p}{1994/05/27}{Moved identification and driver to the
%    front of the file}
% \changes{v1.2t}{1994/06/22}{Rephrased a few sentences to prevent
%    overfull hboxes}
% \changes{v1.2v}{1994/12/01}{Made the oneside option work for the
%    book class}
% \changes{v1.2w}{1994/12/01}{Use \cs{newcommand*} for commands with
%    arguments}
% \changes{v1.2z}{1995/05/16}{Always use \cs{cs} in \cs{changes}
%    entries}
% \changes{v1.3a}{1995/05/17}{Replaced all \cs{hbox to} by \cs{hb@xt@}}
% \changes{v1.3d}{1995/06/05}{Replaced all \cs{uppercase} by
%    \cs{MakeUppercase}}
% \changes{v1.3l}{1995/10/20}{Disabled in compatibility mode all
%    options that are new in \LaTeXe.}
% \changes{v1.3v}{1997/06/16}{Documentation fixes.}
% \changes{v1.4j}{2019/08/27}{Various commands made robust}
% \changes{v1.4m}{2020/04/10}{(JLB) prettyprinting the code}
%
%
% \title{{\Huge \textbf{\LaTeXe}}\ {\Huge \Heiti 的标准文档类}\ \thanks{该文件的版本号为 \fileversion,最新修订日期为 \filedate。}}
%
% \author{%
% Copyright (C) 1992 by Leslie Lamport \and
% Copyright (C) 1994-2021 by Frank Mittelbach, \and Johannes Braams\ 和\
%                            \LaTeX\,项目团队\\[6pt]赣医一附院神经科\ \ \ 黄旭华\ \ \ \ \ \ \ \ 翻译
% }
% \date{\filedate}
% \MaintainedByLaTeXTeam{latex}
% \maketitle
% \tableofcontents
%
%\newpage
% \MaybeStop{}    ^^A
%
% \section{{\textbf{\sc docstrip}}\ {\heiti 模块}}
%
% 实现(implementation)中使用了以下模块(modules)来指导 {\sc docstrip} 生成外部文件(external files):
% \begin{center}
% \begin{tabular}{rl}
    \hlinew{1.2pt}
%   {\heiti 模块}& {\heiti 生成的外部文件}\\ \hlinew{0.7pt}
%   article & 生成 article 文档类\\
%   report  & 生成 report 文档类\\
%   size10  & 生成 10pt 的类选项\\
%   size11  & 生成 11pt 的类选项\\
%   size12  & 生成 12pt 的类选项\\
%   book    & 生成 book 文档类\\
%   bk10    & 生成 10pt 的 book 类选项\\
%   bk11    & 生成 11pt 的 book 类选项\\
%   bk12    & 生成 12pt 的 book 类选项\\
%   driver  & 生成文档驱动程序(documentation driver)文件 \\\hlinew{1.2pt}
% \end{tabular}
% \end{center}
%
%\newpage
%
% \section{\heiti 初始代码}
%
%    在本节中,我们定义了一些稍后使用的命令。
%
% \begin{macro}{\@ptsize}
%    这个控制序列(control sequence)用于存储我们正在排版的点大小(pointsize)的第二个数字(second digit)。因此,通常它的值是 0、1 或 2 中的一个。
%    \begin{macrocode}
%<*article|report|book>
\newcommand\@ptsize{}
%    \end{macrocode}
% \end{macro}
%
% \begin{macro}{\if@restonecol}
%    当文档必须打印成两栏(columns)时,我们有时不得不临时切换到一栏。这个开关用于记住切换回来。
%    \begin{macrocode}
\newif\if@restonecol
%    \end{macrocode}
% \end{macro}
%
% \begin{macro}{\if@titlepage}
%    用于指示是否必须制作标题页(titlepage)的开关。对于 article 文档类,默认设置是不创建单独的标题页(separate titlepage)。
%    \begin{macrocode}
\newif\if@titlepage
%<article>\@titlepagefalse
%<!article>\@titlepagetrue
%    \end{macrocode}
% \end{macro}
%
% \begin{macro}{\if@openright}
%    用于指示章(chapters)是否必须从右页(right-hand page)开始的开关。report 类的默认值为 no; book 类的默认值为 yes。
%    \begin{macrocode}
%<!article>\newif\if@openright
%    \end{macrocode}
% \end{macro}
%
% \changes{v1.3k}{1995/08/27}{Macro \cs{if@openbib} removed}
%
% \begin{macro}{\if@mainmatter}
% \changes{v1.2v}{1994/12/01}{Moved the allocation of
%    \cs{if@mainmatter} here}
%
%    |\if@mainmatter| 开关,仅在 book 文档类中可用,表示我们是否正在处理 book 中的主要材料(main material)。
%    \begin{macrocode}
%<book>\newif\if@mainmatter \@mainmattertrue
%    \end{macrocode}
%  \end{macro}
%
%\newpage
%
% \section{\heiti 选项的声明}
%
%
% \subsection[设定纸张尺寸]{\heiti 设定纸张尺寸}
%
%    变量 |\paperwidth|(纸宽度) 和 |\paperheight|(纸高度) 应反映剪裁(trimming)后的实际纸张尺寸(physical paper size)。
%    对于台式打印机(desk printer)输出,这通常是真正的纸张尺寸,因为没有后期处理(post-processing)。
%    真正的书籍制作(book production)类(glasses)可能会添加其他纸张尺寸,并额外制作裁剪标记(crop marks for trimming)。
%    在兼容模式(compatibility mode)中,这些(和一些后续)选项是禁用的,因为它们在 \LaTeX 2.09\ 中不存在。
% \changes{v1.0g}{1993/12/09}{Removed typo, A4 is not 279 mm high}
%    \begin{macrocode}
\if@compatibility\else
  \DeclareOption{a4paper}
     {\setlength\paperheight {297mm}%
      \setlength\paperwidth  {210mm}}
  \DeclareOption{a5paper}
     {\setlength\paperheight {210mm}%
      \setlength\paperwidth  {148mm}}
  \DeclareOption{b5paper}
     {\setlength\paperheight {250mm}%
      \setlength\paperwidth  {176mm}}
  \DeclareOption{letterpaper}
     {\setlength\paperheight {11in}%
      \setlength\paperwidth  {8.5in}}
  \DeclareOption{legalpaper}
     {\setlength\paperheight {14in}%
      \setlength\paperwidth  {8.5in}}
  \DeclareOption{executivepaper}
     {\setlength\paperheight {10.5in}%
      \setlength\paperwidth  {7.25in}}
%    \end{macrocode}
%
%    假设给出了肖像纸(portrait paper)的尺寸,选项 \Lopt{landscape}\ 切换 |\paperheight| 和 |\paperwidth| 的值。
%    \begin{macrocode}
  \DeclareOption{landscape}
     {\setlength\@tempdima   {\paperheight}%
      \setlength\paperheight {\paperwidth}%
      \setlength\paperwidth  {\@tempdima}}
\fi
%    \end{macrocode}
%
% \subsection[选择类型尺寸]{\heiti 选择类型尺寸}
%
%    通过定义 |\@ptsize| 来处理类型尺寸选项(type size options),它包含所讨论尺寸的最后一位数字,
%    并在 |\ifcase| 语句(statements)上进行分支(branching)。由于历史原因,这样做是为了与使
%    用 |\@ptsize| 变量选择特殊操作(special actions)的其他包保持兼容(compatible)。
%    这使得小于 10pt 的尺寸选项(size options)的声明变得困难,尽管可以使用 \texttt{9}\ 和 \texttt{8},
%    前提是类(class)不会同时定义 \Lopt{8pt}\ 和 \Lopt{18pt}\ 选项。
%
%    \begin{macrocode}
\if@compatibility
  \renewcommand\@ptsize{0}
\else
  \DeclareOption{10pt}{\renewcommand\@ptsize{0}}
\fi
\DeclareOption{11pt}{\renewcommand\@ptsize{1}}
\DeclareOption{12pt}{\renewcommand\@ptsize{2}}
%    \end{macrocode}
%
%
%  \subsection[双面或单面打印]{\heiti 双面或单面打印}
%
%    对于双面打印(two-sided printing),我们使用开关 |\if@twoside|。此外,我们还必须设置 |\if@mparswitch|,
%    以便将任何页边距段落(margin paragraphs)放入外部页边距(outside margin)中。
%    \begin{macrocode}
\if@compatibility\else
  \DeclareOption{oneside}{\@twosidefalse \@mparswitchfalse}
\fi
\DeclareOption{twoside}{\@twosidetrue  \@mparswitchtrue}
%    \end{macrocode}
%
%
%  \subsection[draft\ 选项]{\textbf{draft}\ {\heiti 选项}}
%
%    如果用户要求 \Lopt{draft}(草稿),我们会显示任何溢框(overfull boxes)。我们可能会
%    在这个选项中添加一些更有趣的东西。
%    \begin{macrocode}
\DeclareOption{draft}{\setlength\overfullrule{5pt}}
\if@compatibility\else
  \DeclareOption{final}{\setlength\overfullrule{0pt}}
\fi
%    \end{macrocode}
%
%  \subsection[titlepage\ 选项]{\textbf{titlepage}\ {\heiti 选项}}
%    一篇 article(文章)通常没有单独的标题页(separate titlepage),但是用户可以请求一个。
%    \begin{macrocode}
\DeclareOption{titlepage}{\@titlepagetrue}
\if@compatibility\else
  \DeclareOption{notitlepage}{\@titlepagefalse}
\fi
%    \end{macrocode}
%
%  \subsection[openright\ 选项]{\textbf{openright}\ {\heiti 选项}}
%    此选项确定章(chapter)是否必须从请求的右侧页面(right-hand page)开始。
%    \begin{macrocode}
%<!article>\if@compatibility
%<book>\@openrighttrue
%<!article>\else
%<!article>\DeclareOption{openright}{\@openrighttrue}
%<!article>\DeclareOption{openany}{\@openrightfalse}
%<!article>\fi
%    \end{macrocode}
%
%  \subsection[双栏打印]{\heiti 双栏打印}
%
%    通过该开关再次实现了双栏和单栏打印。
%    \begin{macrocode}
\if@compatibility\else
  \DeclareOption{onecolumn}{\@twocolumnfalse}
\fi
\DeclareOption{twocolumn}{\@twocolumntrue}
%    \end{macrocode}
%
%  \subsection[公式编号在左侧]{\heiti 公式编号在左侧}
%
%    \Lopt{leqno}\ 选项可用于获取公式左侧的公式编号(equation numbers)。它加载在构建
%    格式时从内核文件(kernel files)自动生成的代码。如果公式编号确实获得了特殊的格式(special formatting),
%    那么类将需要显式地(explicitly)提供代码,而不是使用内核文件(kernel file)。
%    \begin{macrocode}
\DeclareOption{leqno}{\input{leqno.clo}}
%    \end{macrocode}
%
%  \subsection[左对齐显示]{\heiti 左对齐显示}
%
%    \Lopt{fleqn}\ 选项重新定义了所显示的数学环境(math environments),使其显示为左对齐(flush left),
%    并从左边距(left margin)缩进 |\mathindent|。它加载在构建格式时从内核文件自动生成的代码。
% \changes{v1.0h}{1993/12/18}{Corrected some typos.  ASAJ.}
%    \begin{macrocode}
\DeclareOption{fleqn}{\input{fleqn.clo}}
%    \end{macrocode}
%
% \subsection[打开参考文献]{\heiti 打开参考文献}
%
%    \Lopt{openbib}\ 选项生成“open(打开)”参考文献样式(bibliography style),其中每个块(block)从一个新行开始,
%    块中的后续行缩进 |\bibindent|。
% \changes{v1.3k}{1995/08/27}{openbib option reimplemented}
%    \begin{macrocode}
\DeclareOption{openbib}{%
%    \end{macrocode}
%    首先,在参考文献环境(bibliography environment)中添加一些钩子(hook)。
%    \begin{macrocode}
  \AtEndOfPackage{%
    \renewcommand\@openbib@code{%
      \advance\leftmargin\bibindent
      \itemindent -\bibindent
      \listparindent \itemindent
      \parsep \z@
      }%
%    \end{macrocode}
%    另外,|\newblock| 的定义被覆盖。
%    \begin{macrocode}
    \renewcommand\newblock{\par}}%
}
%    \end{macrocode}
%
%
%\newpage
%
% \section{\heiti 执行选项}
%
%    在这里,我们执行(execute)默认选项来初始化(initialize)某些变量。注意,文档类“book”总是使用双面打印(two sided printing)。
%    \begin{macrocode}
%<*article>
\ExecuteOptions{letterpaper,10pt,oneside,onecolumn,final}
%</article>
%<*report>
\ExecuteOptions{letterpaper,10pt,oneside,onecolumn,final,openany}
%</report>
%<*book>
\ExecuteOptions{letterpaper,10pt,twoside,onecolumn,final,openright}
%</book>
%    \end{macrocode}
%
%    |\ProcessOptions| 命令会为每个声明的 \Lopt{FOO}\ 选项执行代码,用户在他的 |\documentclass| 命令中
%    为其键入了 \Lopt{FOO}\ 选项。对于他键入的每个未声明的 \Lopt{BAR}\ 选项,会被假定为全局选项(global option)。
%    所有选项都将作为文档选项(document options)传递给文档导言中的任何 |\usepackage| 命令。
%    \begin{macrocode}
\ProcessOptions
%    \end{macrocode}
%    既然所有选项都已执行,我们就可以加载所选的类选项文件(class option file),其中包含
%    所有尺寸相关的代码(size dependent code)。
%    \begin{macrocode}
%<!book>\input{size1\@ptsize.clo}
%<book>\input{bk1\@ptsize.clo}
%</article|report|book>
%    \end{macrocode}
%
% \newpage
%
%  \section{\heiti 加载宏包}
%
%  标准类文件(standard class files)不加载其他宏包(packages)。
%
%
% \section{\heiti 文档布局}
% \label{sec:classes:maincode}
%
%  在本节中,我们最终将讨论令人讨厌的排印细节(typographical details)。
%
% \subsection[字体]{\heiti 字体}
%
%    \LaTeX\ 提供了用户命令(user commands)来更改字体的尺寸(size),它是相对于“主(main)”尺寸的。
%    每个相对尺寸变化的(relative size changing)命令 |\size| 执行命令
%    |\@setfontsize||\size|\meta{font-size}\meta{baselineskip},其中的:
%
%    \begin{description}
%    \item[\meta{font-size}] 从现在开始使用的字体的绝对尺寸(absolute size)。
%
%    \item[\meta{baselineskip}] 所选字体尺寸的 |\baselineskip| 的正常值。
%    (实际值为 |\baselinestretch| * \meta{baselineskip})
%    \end{description}
%
%    \LaTeX{}\ 内核中定义的许多命令缩短了(shorten)以下定义,并一直在使用。他们是:
% \begin{center}
% \begin{tabular}{ll@{\qquad}ll@{\qquad}ll}
%  \verb=\@vpt= & 5 & \verb=\@vipt= & 6 & \verb=\@viipt= & 7 \\
%  \verb=\@viiipt= & 8 & \verb=\@ixpt= & 9 & \verb=\@xpt= & 10 \\
%  \verb=\@xipt= & 10.95 & \verb=\@xiipt= & 12 & \verb=\@xivpt= & 14.4\\
%  ...
%  \end{tabular}
%  \end{center}
%
% \begin{macro}{\normalsize}
% \begin{macro}{\@normalsize}
% \changes{v1.0o}{1994/01/31}{\cs{@normalsize} now defined in the
%    kernel}
%
%    主尺寸(main size)的用户级命令(user level command)为 |\normalsize|。在内部,
%    当 \LaTeX{}\ 引用主尺寸时,它使用 |\@normalsize|。
%    如果 |\normalsize| 从其默认定义(default definition)被重新定义(仅发出错误消息) ,
%    那么 |\@normalsize| 将被定义为 |\normalsize|。否则 |\@normalsize| 只会选择一个10pt/12pt 的尺寸。
%
%    |\normalsize| 宏还为 |\abovedisplayskip|、|\abovedisplayshortskip| 和 |\belowdisplayshortskip| 设置新值。
%
% \changes{v1.0e}{1993/12/07}{\cs{normalsize} doesn't exist, so use
%    \cs{newcommand}}
% \changes{v1.0h}{1993/12/18}{\cs{normalsize} is now defined in the
%    kernel, so use \cs{renewcommand}.  ASAJ.}
%    \begin{macrocode}
%<*10pt|11pt|12pt>
\renewcommand\normalsize{%
%<*10pt>
   \@setfontsize\normalsize\@xpt\@xiipt
   \abovedisplayskip 10\p@ \@plus2\p@ \@minus5\p@
   \abovedisplayshortskip \z@ \@plus3\p@
   \belowdisplayshortskip 6\p@ \@plus3\p@ \@minus3\p@
%</10pt>
%<*11pt>
   \@setfontsize\normalsize\@xipt{13.6}%
   \abovedisplayskip 11\p@ \@plus3\p@ \@minus6\p@
   \abovedisplayshortskip \z@ \@plus3\p@
   \belowdisplayshortskip 6.5\p@ \@plus3.5\p@ \@minus3\p@
%</11pt>
%<*12pt>
   \@setfontsize\normalsize\@xiipt{14.5}%
   \abovedisplayskip 12\p@ \@plus3\p@ \@minus7\p@
   \abovedisplayshortskip \z@ \@plus3\p@
   \belowdisplayshortskip 6.5\p@ \@plus3.5\p@ \@minus3\p@
%</12pt>
%    \end{macrocode}
%    |\belowdisplayskip| 始终等于 |\abovedisplayskip|。第一级列表( first level list)的
%    参数总是由 |\@listI| 给出。
%    \begin{macrocode}
   \belowdisplayskip \abovedisplayskip
   \let\@listi\@listI}
%    \end{macrocode}
%
%    We initially choose the normalsize font.
%    \begin{macrocode}
\normalsize
%    \end{macrocode}
%    我们使用 |\MakeRobust| 而不是上面的 |\DeclareRobustCommand| 来避免重新定义的
%    日志条目(log entry)。但如果我们在回滚情况(rollback situation)下运行(2015年之前),我们不会触及它。
% \changes{v1.4k}{2019/10/25}{Roll back handling (gh/201)}
%    \begin{macrocode}
\ifx\MakeRobust\@undefined \else
    \MakeRobust\normalsize
\fi
%    \end{macrocode}
% \end{macro}
% \end{macro}
%
% \begin{macro}{\small}
%    这类似于 |\normalsize|。
% \changes{v1.0h}{1993/12/18}{\cs{small} is now defined in the kernel,
%    so use \cs{renewcommand}.  ASAJ.}
% \changes{v1.2e}{1994/04/14}{\cs{small} is no longer defined in the
%    kernel; use \cs{newcommand}}
%    \begin{macrocode}
\DeclareRobustCommand\small{%
%<*10pt>
   \@setfontsize\small\@ixpt{11}%
   \abovedisplayskip 8.5\p@ \@plus3\p@ \@minus4\p@
   \abovedisplayshortskip \z@ \@plus2\p@
   \belowdisplayshortskip 4\p@ \@plus2\p@ \@minus2\p@
   \def\@listi{\leftmargin\leftmargini
               \topsep 4\p@ \@plus2\p@ \@minus2\p@
               \parsep 2\p@ \@plus\p@ \@minus\p@
               \itemsep \parsep}%
%</10pt>
%<*11pt>
   \@setfontsize\small\@xpt\@xiipt
   \abovedisplayskip 10\p@ \@plus2\p@ \@minus5\p@
   \abovedisplayshortskip \z@ \@plus3\p@
   \belowdisplayshortskip 6\p@ \@plus3\p@ \@minus3\p@
   \def\@listi{\leftmargin\leftmargini
               \topsep 6\p@ \@plus2\p@ \@minus2\p@
               \parsep 3\p@ \@plus2\p@ \@minus\p@
               \itemsep \parsep}%
%</11pt>
%<*12pt>
   \@setfontsize\small\@xipt{13.6}%
   \abovedisplayskip 11\p@ \@plus3\p@ \@minus6\p@
   \abovedisplayshortskip \z@ \@plus3\p@
   \belowdisplayshortskip 6.5\p@ \@plus3.5\p@ \@minus3\p@
   \def\@listi{\leftmargin\leftmargini
               \topsep 9\p@ \@plus3\p@ \@minus5\p@
               \parsep 4.5\p@ \@plus2\p@ \@minus\p@
               \itemsep \parsep}%
%</12pt>
   \belowdisplayskip \abovedisplayskip
}
%    \end{macrocode}
% \end{macro}
%
% \begin{macro}{\footnotesize}
%    这类似于 |\normalsize|。
% \changes{v1.0h}{1993/12/18}{\cs{footnotesize} is now defined in the
%    kernel, so use \cs{renewcommand}.  ASAJ.}
% \changes{v1.2e}{1994/04/14}{use \cs{newcommand} again}
%    \begin{macrocode}
\DeclareRobustCommand\footnotesize{%
%<*10pt>
   \@setfontsize\footnotesize\@viiipt{9.5}%
   \abovedisplayskip 6\p@ \@plus2\p@ \@minus4\p@
   \abovedisplayshortskip \z@ \@plus\p@
   \belowdisplayshortskip 3\p@ \@plus\p@ \@minus2\p@
   \def\@listi{\leftmargin\leftmargini
               \topsep 3\p@ \@plus\p@ \@minus\p@
               \parsep 2\p@ \@plus\p@ \@minus\p@
               \itemsep \parsep}%
%</10pt>
%<*11pt>
   \@setfontsize\footnotesize\@ixpt{11}%
   \abovedisplayskip 8\p@ \@plus2\p@ \@minus4\p@
   \abovedisplayshortskip \z@ \@plus\p@
   \belowdisplayshortskip 4\p@ \@plus2\p@ \@minus2\p@
   \def\@listi{\leftmargin\leftmargini
               \topsep 4\p@ \@plus2\p@ \@minus2\p@
               \parsep 2\p@ \@plus\p@ \@minus\p@
               \itemsep \parsep}%
%</11pt>
%<*12pt>
   \@setfontsize\footnotesize\@xpt\@xiipt
   \abovedisplayskip 10\p@ \@plus2\p@ \@minus5\p@
   \abovedisplayshortskip \z@ \@plus3\p@
   \belowdisplayshortskip 6\p@ \@plus3\p@ \@minus3\p@
   \def\@listi{\leftmargin\leftmargini
               \topsep 6\p@ \@plus2\p@ \@minus2\p@
               \parsep 3\p@ \@plus2\p@ \@minus\p@
               \itemsep \parsep}%
%</12pt>
   \belowdisplayskip \abovedisplayskip
}
%</10pt|11pt|12pt>
%    \end{macrocode}
% \end{macro}
%
% \begin{macro}{\scriptsize}
% \begin{macro}{\tiny}
% \begin{macro}{\large}
% \begin{macro}{\Large}
% \begin{macro}{\LARGE}
% \begin{macro}{\huge}
% \begin{macro}{\Huge}
%    这些宏(macros)都比以前的宏简单得多,它们只选择一个新的字体尺寸(fontsize),但只
%    保留显示(displays)和列表(lists)的参数。
% \changes{v1.0h}{1993/12/18}{These are now defined in the kernel,
%    so use \cs{renewcommand}.  ASAJ.}
% \changes{v1.2e}{1994/04/14}{use \cs{newcommand} again}
%    \begin{macrocode}
%<*10pt>
\DeclareRobustCommand\scriptsize{\@setfontsize\scriptsize\@viipt\@viiipt}
\DeclareRobustCommand\tiny{\@setfontsize\tiny\@vpt\@vipt}
\DeclareRobustCommand\large{\@setfontsize\large\@xiipt{14}}
\DeclareRobustCommand\Large{\@setfontsize\Large\@xivpt{18}}
\DeclareRobustCommand\LARGE{\@setfontsize\LARGE\@xviipt{22}}
\DeclareRobustCommand\huge{\@setfontsize\huge\@xxpt{25}}
\DeclareRobustCommand\Huge{\@setfontsize\Huge\@xxvpt{30}}
%</10pt>
%<*11pt>
\DeclareRobustCommand\scriptsize{\@setfontsize\scriptsize\@viiipt{9.5}}
\DeclareRobustCommand\tiny{\@setfontsize\tiny\@vipt\@viipt}
\DeclareRobustCommand\large{\@setfontsize\large\@xiipt{14}}
\DeclareRobustCommand\Large{\@setfontsize\Large\@xivpt{18}}
\DeclareRobustCommand\LARGE{\@setfontsize\LARGE\@xviipt{22}}
\DeclareRobustCommand\huge{\@setfontsize\huge\@xxpt{25}}
\DeclareRobustCommand\Huge{\@setfontsize\Huge\@xxvpt{30}}
%</11pt>
%<*12pt>
\DeclareRobustCommand\scriptsize{\@setfontsize\scriptsize\@viiipt{9.5}}
\DeclareRobustCommand\tiny{\@setfontsize\tiny\@vipt\@viipt}
\DeclareRobustCommand\large{\@setfontsize\large\@xivpt{18}}
\DeclareRobustCommand\Large{\@setfontsize\Large\@xviipt{22}}
\DeclareRobustCommand\LARGE{\@setfontsize\LARGE\@xxpt{25}}
\DeclareRobustCommand\huge{\@setfontsize\huge\@xxvpt{30}}
\let\Huge=\huge
%</12pt>
%    \end{macrocode}
% \end{macro}
% \end{macro}
% \end{macro}
% \end{macro}
% \end{macro}
% \end{macro}
% \end{macro}
%
%
% \subsection[分段]{\heiti 分段}
%
% \begin{macro}{\lineskip}
% \begin{macro}{\normallineskip}
%    当两行(lines)过于接近时,这些参数控制 \TeX\ 的行为。
%    \begin{macrocode}
%<*article|report|book>
\setlength\lineskip{1\p@}
\setlength\normallineskip{1\p@}
%    \end{macrocode}
% \end{macro}
% \end{macro}
%
% \begin{macro}{\baselinestretch}
%    这用作 |\baselineskip| 的乘数(multiplier)。默认设置是{\kaiti 不}拉伸(stretch)基线(baselines)。
%    请注意,如果此命令未解析为“空(empty)”,则会忽略 |\baselineskip| 规范中的任
%    何 \texttt{plus}\ 和 \texttt{minus}\ 部分。
%    \begin{macrocode}
\renewcommand\baselinestretch{}
%    \end{macrocode}
% \end{macro}
%
% \begin{macro}{\parskip}
% \begin{macro}{\parindent}
%    |\parskip| 在段落之间提供额外的垂直间距(vertical space),|\parindent| 是段落
%    缩进(paragraph indentation)的宽度。|\parindent| 的值取决于我们是否处于双栏模式(two-column mode)。
% \changes{v1.0m}{1994/01/12}{\cs{parindent} should be different,
%    depending on the pointsize}
%    \begin{macrocode}
\setlength\parskip{0\p@ \@plus \p@}
%</article|report|book>
%<*10pt|11pt|12pt>
\if@twocolumn
  \setlength\parindent{1em}
\else
%<10pt>  \setlength\parindent{15\p@}
%<11pt>  \setlength\parindent{17\p@}
%<12pt>  \setlength\parindent{1.5em}
\fi
%</10pt|11pt|12pt>
%    \end{macrocode}
% \end{macro}
% \end{macro}
%
%  \begin{macro}{\smallskipamount}
%  \begin{macro}{\medskipamount}
%  \begin{macro}{\bigskipamount}
%    这三个参数的值是在 \LaTeX\ 内核中设置的。根据指定的尺寸选项(size option),它们可能会有所不同。
%    但由于无论尺寸选项如何,它们始终具有相同的值,因此我们不会更改它们以保持
%    与 \LaTeX~2.09\ 和旧版本的 \LaTeXe\ 兼容。
% \changes{v1.3n}{1995/10/29}{Added setting the values of
%    \cs{...skipamount}}
%    \begin{macrocode}
%<*10pt|11pt|12pt>
\setlength\smallskipamount{3\p@ \@plus 1\p@ \@minus 1\p@}
\setlength\medskipamount{6\p@ \@plus 2\p@ \@minus 2\p@}
\setlength\bigskipamount{12\p@ \@plus 4\p@ \@minus 4\p@}
%</10pt|11pt|12pt>
%    \end{macrocode}
%  \end{macro}
%  \end{macro}
%  \end{macro}
%
% \begin{macro}{\@lowpenalty}
% \begin{macro}{\@medpenalty}
% \begin{macro}{\@highpenalty}%
%    命令 |\nopagebreak| 和 |\nolinebreak| 设置了惩罚(penalty),以阻止这些中断(breaks)在其放置点发生。
%    他们使用 |\@lowpenalty|、|\@medpenalty| 或 |\@highpenalty|,这取决于他们的参数。
%    \begin{macrocode}
%<*article|report|book>
\@lowpenalty   51
\@medpenalty  151
\@highpenalty 301
%    \end{macrocode}
% \end{macro}
% \end{macro}
% \end{macro}
%
% \begin{macro}{\clubpenalty}
% \begin{macro}{\widowpenalty}
%    这些惩罚(penalty)用于阻止段末短行(club lines)和寡行(widow lines)\footnote{[译者注]寡行(widow lines):指的是
%    在新页面顶部单独出现的段落的最后一行。(这一行来自于上一个页面的段落)}。因为我们使用它们的默认值,所以这里只显示它们,注释掉了。
%    \begin{macrocode}
% \clubpenalty  150
% \widowpenalty 150
%    \end{macrocode}
% \end{macro}
% \end{macro}
%
% \begin{macro}{\displaywidowpenalty}
% \begin{macro}{\predisplaypenalty}
% \begin{macro}{\postdisplaypenalty}
%    在数学显示(math display)前阻止寡行(widow lines)(但不要太多),禁止在显示前直接中断(breaking)。
%    允许在显示后中断而不受惩罚(penalty)。再次使用默认值,因此我们只在此处显示它们。
%    \begin{macrocode}
% \displaywidowpenalty 50
% \predisplaypenalty   10000
% \postdisplaypenalty  0
%    \end{macrocode}
% \end{macro}
% \end{macro}
% \end{macro}
%
% \begin{macro}{\interlinepenalty}
%    允许在段落中间分页(breaking of a page)。
%    \begin{macrocode}
% \interlinepenalty 0
%    \end{macrocode}
% \end{macro}
%
%
% \begin{macro}{\brokenpenalty}
%    我们允许在用连字符连接的行(hyphenated line)之后分页。
% \changes{v1.1a}{1994/03/12}{Show correct default which is 100}
%    \begin{macrocode}
% \brokenpenalty 100
%</article|report|book>
%    \end{macrocode}
% \end{macro}
%
%
% \subsection[页面布局]{\heiti 页面布局}
%
%    所有页边距尺寸(margin dimensions)均从距页面顶部和左侧(lefthand)一英寸的点开始测量。
%
% \subsubsection[垂直间距]{\heiti 垂直间距}
%
% \begin{macro}{\headheight}
% \begin{macro}{\headsep}
% \begin{macro}{\topskip}
%    |\headheight| 栏外标题(running head,即短标题)所在的盒子的高度。|\headsep| 是栏外标题的底部
%    和文本顶部之间的距离。|\topskip| 是页面第一行的 |\baselineskip|;如果值为 0pt,
%    那么 \LaTeX\ 的输出程序(output routine)将无法正常工作,因此不要这样做!
%    \begin{macrocode}
%<*10pt|11pt|12pt>
\setlength\headheight{12\p@}
%<!bk>\setlength\headsep   {25\p@}
%<10pt&bk>\setlength\headsep   {.25in}
%<11pt&bk>\setlength\headsep   {.275in}
%<12pt&bk>\setlength\headsep   {.275in}
%<10pt>\setlength\topskip   {10\p@}
%<11pt>\setlength\topskip   {11\p@}
%<12pt>\setlength\topskip   {12\p@}
%    \end{macrocode}
% \end{macro}
% \end{macro}
% \end{macro}
%
% \begin{macro}{\footskip}
%    栏外脚注(running footer)所在盒子的基线(baseline)到最后一行文本基线之间的距离由 |\footskip| 控制。
%    \begin{macrocode}
%<!bk>\setlength\footskip{30\p@}
%<10pt&bk>\setlength\footskip{.35in}
%<11pt&bk>\setlength\footskip{.38in}
%<12pt&bk>\setlength\footskip{30\p@}
%    \end{macrocode}
% \end{macro}
%
% \begin{macro}{\maxdepth}
% \changes{v1.2k}{1994/05/06}{Added setting of \cs{maxdepth} and
%    \cs{@maxdepth}}
% \changes{v1.3j}{1995/08/16}{Take setting of
%    \cs{@maxdepth} out again}
%    \TeX\ 原语寄存器(primitive register) |\maxdepth| 有一个类似于 |\topskip| 的函数。
%    |\@maxdepth| 寄存器应该始终包含 |\maxdepth| 的副本。这是通过在 |\begin{document}| 处
%    的内部设置它来实现的。在plain \TeX\ 和 \LaTeX~2.09\ 中,|\maxdepth| 都有一个固定的值\texttt{4pt};
%    在原生 \LaTeX2e\ 模式(native \LaTeX2e\ mode)中,我们让值取决于字体尺寸(typesize)。我们将其设置
%    为|\maxdepth| $+$ |\topskip| $=$ typesize $\times 1.5$。碰巧,在这些类中,|\topskip| 等于 typesize,
%    因此我们将 |\maxdepth| 设置为 |\topskip| 值的一半。
%    \begin{macrocode}
\if@compatibility \setlength\maxdepth{4\p@} \else
  \setlength\maxdepth{.5\topskip} \fi
%    \end{macrocode}
% \end{macro}
%
% \subsubsection[文本的尺寸]{\heiti 文本的尺寸}
%
% \begin{macro}{\textwidth}
%    当我们处于兼容模式(compatibility mode)时,我们必须确保打印区域的尺寸(dimensions of the printed area)
%    与用户习惯看到的尺寸没有差异。
%
%    \begin{macrocode}
\if@compatibility
  \if@twocolumn
    \setlength\textwidth{410\p@}
  \else
%<10pt&!bk>    \setlength\textwidth{345\p@}
%<11pt&!bk>    \setlength\textwidth{360\p@}
%<12pt&!bk>    \setlength\textwidth{390\p@}
%<10pt&bk>    \setlength\textwidth{4.5in}
%<11pt&bk>    \setlength\textwidth{5in}
%<12pt&bk>    \setlength\textwidth{5in}
  \fi
%    \end{macrocode}
%    当我们不是处于兼容模式时,我们可以设置一些不同的尺寸(dimensions),例如考虑纸张尺寸(paper size)。
%    \begin{macrocode}
\else
%    \end{macrocode}
%    首先,我们计算所选纸张上允许的最大 |\textwidth|,并将其存储在 |\@tempdima| 中。然后,
%    我们在 |\@tempdimb| 中存储大约 60--70 个字符的行长度(length of a line)。当使用
%    计算机现代字体(Computer Modern fonts)时,给出的值或多或少是合适的。
% \changes{v1.1a}{1994/03/12}{Have old values for width in native mode}
%    \begin{macrocode}
  \setlength\@tempdima{\paperwidth}
  \addtolength\@tempdima{-2in}
%<10pt>  \setlength\@tempdimb{345\p@}
%<11pt>  \setlength\@tempdimb{360\p@}
%<12pt>  \setlength\@tempdimb{390\p@}
%    \end{macrocode}
%
%    现在我们可以设置 |\textwidth|,这取决于我们将设置一栏还是两栏。
%
%    在双栏模式(two-column mode)中,每一{\kaiti 栏}的宽度不应
%    超过 |\@tempdimb|(例如,这可能发生在 \textsc{a3}\ 纸张上)。
%    \begin{macrocode}
  \if@twocolumn
    \ifdim\@tempdima>2\@tempdimb\relax
      \setlength\textwidth{2\@tempdimb}
    \else
      \setlength\textwidth{\@tempdima}
    \fi
%    \end{macrocode}
%
%    在单栏模式(one-column mode)中,文本的宽度不应超过最小纸张宽度(paperwidth)(页边空白为-2英寸)
%    和由字符数(number of characters)定义的一行的最大长度。
%    \begin{macrocode}
  \else
    \ifdim\@tempdima>\@tempdimb\relax
      \setlength\textwidth{\@tempdimb}
    \else
      \setlength\textwidth{\@tempdima}
    \fi
  \fi
\fi
%    \end{macrocode}
%
%    这里我们稍微修改一下文本的宽度(width of the text),使之成为一个完整的点数(number of points)。
%    \begin{macrocode}
\if@compatibility\else
  \@settopoint\textwidth
\fi
%    \end{macrocode}
% \end{macro}
%
% \begin{macro}{\textheight}
%    现在我们已经计算了文本的宽度,我们必须考虑高度(height)。|\textheight| 是文本的高度,包括
%    脚注(footnotes)和图形(figures),不包括栏外标题(running head)和栏外脚注(running foot)。
%
%    首先确保兼容模式(compatibility mode)的尺寸(dimensions)与 \LaTeX2.09\ 的尺寸相同。
%    行数的计算方法是旧的 |\textheight| 减去 |\topskip| 的下限(floor),再除以 |\baselineskip| 得到
%    的 |\normalsize|。旧的 |\textheight| 值是 528pt
%
%    \begin{macrocode}
\if@compatibility
%<10pt&!bk>  \setlength\textheight{43\baselineskip}
%<10pt&bk>  \setlength\textheight{41\baselineskip}
%<11pt>  \setlength\textheight{38\baselineskip}
%<12pt>  \setlength\textheight{36\baselineskip}
%    \end{macrocode}
%
%    同样,我们根据纸张尺寸(papersize)和使用的基线间距(baselineskip)来计算这个值,以便在页面
%    上有完整的行数(whole number of lines)。
%    \begin{macrocode}
\else
  \setlength\@tempdima{\paperheight}
%    \end{macrocode}
%
%    我们在页面的顶部和底部留有至少 1 英寸的边距(margin)。
%    \begin{macrocode}
  \addtolength\@tempdima{-2in}
%    \end{macrocode}
%
%    我们还必须为栏外标题(running headers)和栏外页脚(running footers)留出空间。
%    \begin{macrocode}
  \addtolength\@tempdima{-1.5in}
%    \end{macrocode}
%
%    然后,我们将结果除以当前的 |\baselineskip|,并将其存储在计数寄存器(count register) |\@tempcnta| 中,
%    该寄存器随后包含适合此页面的行数。
%    \begin{macrocode}
  \divide\@tempdima\baselineskip
  \@tempcnta=\@tempdima
%    \end{macrocode}
%
%    由此我们可以计算文本的高度(height)。
%    \begin{macrocode}
  \setlength\textheight{\@tempcnta\baselineskip}
\fi
%    \end{macrocode}
%
%    页面上的第一行的高度为 |\topskip|。
%    \begin{macrocode}
\addtolength\textheight{\topskip}
%    \end{macrocode}
% \end{macro}
%
%
%
% \subsubsection[边距]{\heiti 边距}
%
%    现在,这些参数的大多数值都是根据所使用的纸张尺寸(papersize)进行计算的。在计算中,需要考虑 |\marginparsep|,
%    因此我们首先给它赋值。
%
% \begin{macro}{\marginparsep}
% \begin{macro}{\marginparpush}
%    主文本(main text)和边距注释(marginal notes)之间的水平间距(horizontal space)由 |\marginparsep| 决定,
%    两个边距注释间的最小垂直间距(minimum vertical separation)由 |\marginparpush| 控制。
%    \begin{macrocode}
\if@twocolumn
  \setlength\marginparsep {10\p@}
\else
%<10pt&!bk>  \setlength\marginparsep{11\p@}
%<11pt&!bk>  \setlength\marginparsep{10\p@}
%<12pt&!bk>  \setlength\marginparsep{10\p@}
%<bk>  \setlength\marginparsep{7\p@}
\fi
%<10pt|11pt>\setlength\marginparpush{5\p@}
%<12pt>\setlength\marginparpush{7\p@}
%    \end{macrocode}
% \end{macro}
% \end{macro}
%
%    现在我们可以给出其他边距参数(margin parameters)的值。对于原生(native) \LaTeXe,
%    计算这些值。
% \begin{macro}{\oddsidemargin}
% \begin{macro}{\evensidemargin}
% \begin{macro}{\marginparwidth}
%    首先,我们给出兼容模式(compatibility mode)的值。
%
%    双面打印(two-sided printing)的值:
%    \begin{macrocode}
\if@compatibility
%<*bk>
%<10pt>   \setlength\oddsidemargin   {.5in}
%<11pt>   \setlength\oddsidemargin   {.25in}
%<12pt>   \setlength\oddsidemargin   {.25in}
%<10pt>   \setlength\evensidemargin  {1.5in}
%<11pt>   \setlength\evensidemargin  {1.25in}
%<12pt>   \setlength\evensidemargin  {1.25in}
%<10pt>   \setlength\marginparwidth {.75in}
%<11pt>   \setlength\marginparwidth {1in}
%<12pt>   \setlength\marginparwidth {1in}
%</bk>
%<*!bk>
  \if@twoside
%<10pt>     \setlength\oddsidemargin   {44\p@}
%<11pt>     \setlength\oddsidemargin   {36\p@}
%<12pt>     \setlength\oddsidemargin   {21\p@}
%<10pt>     \setlength\evensidemargin  {82\p@}
%<11pt>     \setlength\evensidemargin  {74\p@}
%<12pt>     \setlength\evensidemargin  {59\p@}
%<10pt>     \setlength\marginparwidth {107\p@}
%<11pt>     \setlength\marginparwidth {100\p@}
%<12pt>     \setlength\marginparwidth {85\p@}
%    \end{macrocode}
%    单面打印(one-sided printing)的值:
%    \begin{macrocode}
  \else
%<10pt>     \setlength\oddsidemargin   {63\p@}
%<11pt>     \setlength\oddsidemargin   {54\p@}
%<12pt>     \setlength\oddsidemargin   {39.5\p@}
%<10pt>     \setlength\evensidemargin  {63\p@}
%<11pt>     \setlength\evensidemargin  {54\p@}
%<12pt>     \setlength\evensidemargin  {39.5\p@}
%<10pt>     \setlength\marginparwidth  {90\p@}
%<11pt>     \setlength\marginparwidth  {83\p@}
%<12pt>     \setlength\marginparwidth  {68\p@}
  \fi
%</!bk>
%    \end{macrocode}
%    两栏模式(two-column mode)的值:
%    \begin{macrocode}
  \if@twocolumn
     \setlength\oddsidemargin  {30\p@}
     \setlength\evensidemargin {30\p@}
     \setlength\marginparwidth {48\p@}
  \fi
%    \end{macrocode}
%
%    当我们不处于兼容模式时,我们可以考虑所选纸张的尺寸。
%
%    |\oddsidemargin| 和 |\marginparwidth| 的值将根据 |\if@twoside| 的状态设置。
%
%    如果 |@twoside| 为 true(对 于book 总是如此),我们将使内页边距(inner margin)小于外页边距(outer margin)。
%    \begin{macrocode}
\else
  \if@twoside
    \setlength\@tempdima        {\paperwidth}
    \addtolength\@tempdima      {-\textwidth}
    \setlength\oddsidemargin    {.4\@tempdima}
    \addtolength\oddsidemargin  {-1in}
%    \end{macrocode}
%    文本的边距宽度(width of the margin)设置为宽度的余数(remainder),但宽度为 0.4in 英寸的
%    空白的“实际边距(real margin)”除外。也许应该内置一个检查(check),以确保(文本)边距宽度不会太小!
%
% \changes{v1.1a}{1994/03/12}{New algorithm for \cs{oddsidemargin}}
% \changes{v1.1a}{1994/03/12}{New algorithm for \cs{marginparwidth}}
% \changes{v1.2z}{1995/04/14}{Also take \cs{marginparsep} into account
%    here}
%    \begin{macrocode}
    \setlength\marginparwidth   {.6\@tempdima}
    \addtolength\marginparwidth {-\marginparsep}
    \addtolength\marginparwidth {-0.4in}
%    \end{macrocode}
%    对于单面打印(one-sided printing),我们通过计算 |\textwidth| 和 |\paperwidth| 之间的
%    差值将文本居中放置在页面上。该差值的一半用于边距(margin)(因此 |\oddsidemargin| 小于 1 英寸)。
%    \begin{macrocode}
  \else
    \setlength\@tempdima        {\paperwidth}
    \addtolength\@tempdima      {-\textwidth}
    \setlength\oddsidemargin    {.5\@tempdima}
    \addtolength\oddsidemargin  {-1in}
    \setlength\marginparwidth   {.5\@tempdima}
    \addtolength\marginparwidth {-\marginparsep}
    \addtolength\marginparwidth {-0.4in}
    \addtolength\marginparwidth {-.4in}
  \fi
%    \end{macrocode}
%    使用上述算法,|\marginparwidth| 可以变得相当大,这是我们不希望看到的。
%    \begin{macrocode}
  \ifdim \marginparwidth >2in
     \setlength\marginparwidth{2in}
  \fi
%    \end{macrocode}
%    完成这些计算之后,我们将它们设置为  pt 值。
%    \begin{macrocode}
  \@settopoint\oddsidemargin
  \@settopoint\marginparwidth
%    \end{macrocode}
%
%    现在可以根据上面设置的值计算 |\evensidemargin|。
% \changes{v1.0l}{1994/01/11}{Computing of \cs{evensidemargin}
%    should only occur in compatibility mode}
%    \begin{macrocode}
  \setlength\evensidemargin  {\paperwidth}
  \addtolength\evensidemargin{-2in}
  \addtolength\evensidemargin{-\textwidth}
  \addtolength\evensidemargin{-\oddsidemargin}
%    \end{macrocode}
%    将 |\evensidemargin| 设置为整点值(full point value)可能会产生一个小的误差(small error)。
%    然而,该误差在当今技术的双面打印机(doublesided printer)能够精确打印的误差范围(error range)内。
%    \begin{macrocode}
  \@settopoint\evensidemargin
\fi
%    \end{macrocode}
% \end{macro}
% \end{macro}
% \end{macro}
%
% \begin{macro}{\topmargin}
%    |\topmargin|(上边距)是“可打印区域(printable area)”顶部(低于纸张顶部 1 英寸)与
%    栏外标题(running head)所在的盒子顶部之间的距离。
%
%    现在可以根据上面设置的值计算它。
%    \begin{macrocode}
\if@compatibility
%<!bk>  \setlength\topmargin{27pt}
%<10pt&bk>  \setlength\topmargin{.75in}
%<11pt&bk>  \setlength\topmargin{.73in}
%<12pt&bk>  \setlength\topmargin{.73in}
\else
  \setlength\topmargin{\paperheight}
  \addtolength\topmargin{-2in}
  \addtolength\topmargin{-\headheight}
  \addtolength\topmargin{-\headsep}
  \addtolength\topmargin{-\textheight}
  \addtolength\topmargin{-\footskip}     % 这可能是错误的!
%    \end{macrocode}
%    通过改变下一行中的因子(factor),整个页面可以垂直移动(shifted vertically)。
% \changes{v1.2u}{1994/07/13}{Moved rounding of \cs{topmargin} to
%    native mode}
%    \begin{macrocode}
  \addtolength\topmargin{-.5\topmargin}
  \@settopoint\topmargin
\fi
%    \end{macrocode}
% \end{macro}
%
%
% \subsubsection[脚注]{\heiti 脚注}
%
% \begin{macro}{\footnotesep}
%    |\footnotesep| 是每个脚注开头的支柱(strut)的高度。它等于此类中 |\footnotesize| 的支柱的高度,
%    因此脚注之间没有额外的空间(extra space)。
%    \begin{macrocode}
%<10pt>\setlength\footnotesep{6.65\p@}
%<11pt>\setlength\footnotesep{7.7\p@}
%<12pt>\setlength\footnotesep{8.4\p@}
%    \end{macrocode}
% \end{macro}
%
% \begin{macro}{\footins}
%    |\skip\footins| 是主文本(main text)最后一行和第一个脚注顶部之间的空间(space)。
%    \begin{macrocode}
%<10pt>\setlength{\skip\footins}{9\p@ \@plus 4\p@ \@minus 2\p@}
%<11pt>\setlength{\skip\footins}{10\p@ \@plus 4\p@ \@minus 2\p@}
%<12pt>\setlength{\skip\footins}{10.8\p@ \@plus 4\p@ \@minus 2\p@}
%</10pt|11pt|12pt>
%    \end{macrocode}
% \end{macro}
%
% \subsubsection[浮动放置参数]{\heiti 浮动放置参数}
%
% 在 \LaTeXe{}\ 内核中,所有的浮动参数(float parameters)都被赋予默认值。因此,需要
% 使用 |\renewcommand| 设置非计数器(not counters)的参数。
%
% \paragraph{放置浮动对象的限制}
%
% \begin{macro}{\c@topnumber}
%    \Lcount{topnumber}\ 计数器(counter)保存可出现在文本页面(text page)顶部的最大浮动数(maximum number of floats)。
%    \begin{macrocode}
%<*article|report|book>
\setcounter{topnumber}{2}
%    \end{macrocode}
% \end{macro}
%
% \begin{macro}{\topfraction}
%    这表示文本页面(text page)顶部的浮动可以占据的最大部分。
% \changes{v1.0h}{1993/12/18}{Replaced \cs{newcommand} with
%    \cs{renewcommand}.  ASAJ.}
%    \begin{macrocode}
\renewcommand\topfraction{.7}
%    \end{macrocode}
% \end{macro}
%
% \begin{macro}{\c@bottomnumber}
%    \Lcount{bottomnumber}\ 计数器(counter)保存可出现在文本页底部的最大浮动数(maximum number of floats)。
%    \begin{macrocode}
\setcounter{bottomnumber}{1}
%    \end{macrocode}
% \end{macro}
%
% \begin{macro}{\bottomfraction}
%    这表示文本页面(text page)底部的浮动可以占据的最大部分。
% \changes{v1.0h}{1993/12/18}{Replaced \cs{newcommand} with
%    \cs{renewcommand}.  ASAJ.}
%    \begin{macrocode}
\renewcommand\bottomfraction{.3}
%    \end{macrocode}
% \end{macro}
%
% \begin{macro}{\c@totalnumber}
%    这表示可以出现在任何文本页面上的最大浮动数。
%    \begin{macrocode}
\setcounter{totalnumber}{3}
%    \end{macrocode}
% \end{macro}
%
% \begin{macro}{\textfraction}
%    这表示文本页面必须被文本占据的最小部分。
% \changes{v1.0h}{1993/12/18}{Replaced \cs{newcommand} with
%    \cs{renewcommand}.  ASAJ.}
%    \begin{macrocode}
\renewcommand\textfraction{.2}
%    \end{macrocode}
% \end{macro}
%
% \begin{macro}{\floatpagefraction}
%    这表示在生成“浮动页面(float page)”之前必须被浮动对象(floating objects)占据的页面的
%    最小部分(minimum part of a page)。
% \changes{v1.0h}{1993/12/18}{Replaced \cs{newcommand} with
%    \cs{renewcommand}.  ASAJ.}
%    \begin{macrocode}
\renewcommand\floatpagefraction{.5}
%    \end{macrocode}
% \end{macro}
%
% \begin{macro}{\c@dbltopnumber}
%    \Lcount{dbltopnumber}\ 计数器(counter)保存可以出现在两栏文本页面(two-column text page)
%    顶部的最大两栏浮点数(maximum number of two-column floats)。
%    \begin{macrocode}
\setcounter{dbltopnumber}{2}
%    \end{macrocode}
% \end{macro}
%
% \begin{macro}{\dbltopfraction}
%    这表示两栏文本页面(two-column text page)的最大部分可以被顶部的两栏浮动(two-column floats)所占据。
% \changes{v1.0h}{1993/12/18}{Replaced \cs{newcommand} with
%    \cs{renewcommand}.  ASAJ.}
%    \begin{macrocode}
\renewcommand\dbltopfraction{.7}
%    \end{macrocode}
% \end{macro}
%
% \begin{macro}{\dblfloatpagefraction}
%    这表示在生成“浮动页面(float page)”之前,页面的最小部分(minimum part of a page)必须由
%    两栏宽浮动对象(two-column wide floating objects)占据。
% \changes{v1.0h}{1993/12/18}{Replaced \cs{newcommand} with
%    \cs{renewcommand}.  ASAJ.}
%    \begin{macrocode}
\renewcommand\dblfloatpagefraction{.5}
%</article|report|book>
%    \end{macrocode}
% \end{macro}
%
% \paragraph{文本页面上的浮动}
%
% \begin{macro}{\floatsep}
% \begin{macro}{\textfloatsep}
% \begin{macro}{\intextsep}
%    当浮动对象(floating object)放置在带有文本的页面上时,这些参数控制浮动对象和页面上其他
%    对象之间的分隔(separation)。这些参数用于单栏模式(one-column mode)和双栏模式(two-column mode)
%    中的单栏浮点(single-column floats)。
%
%    |\floatsep| 是移动到文本页面顶部或底部的相邻浮动(adjacent floats)之间的间距(space)。
%
%    |\textfloatsep| 是主文本(main text)和页面顶部或底部浮动之间的间距(space)。
%
%    |\intextsep| 是文本内浮动(|\intextsep|)和文本之间的间距。
%    \begin{macrocode}
%<*10pt>
\setlength\floatsep    {12\p@ \@plus 2\p@ \@minus 2\p@}
\setlength\textfloatsep{20\p@ \@plus 2\p@ \@minus 4\p@}
\setlength\intextsep   {12\p@ \@plus 2\p@ \@minus 2\p@}
%</10pt>
%<*11pt>
\setlength\floatsep    {12\p@ \@plus 2\p@ \@minus 2\p@}
\setlength\textfloatsep{20\p@ \@plus 2\p@ \@minus 4\p@}
\setlength\intextsep   {12\p@ \@plus 2\p@ \@minus 2\p@}
%</11pt>
%<*12pt>
\setlength\floatsep    {12\p@ \@plus 2\p@ \@minus 4\p@}
\setlength\textfloatsep{20\p@ \@plus 2\p@ \@minus 4\p@}
\setlength\intextsep   {14\p@ \@plus 4\p@ \@minus 4\p@}
%</12pt>
%    \end{macrocode}
% \end{macro}
% \end{macro}
% \end{macro}
%
% \begin{macro}{\dblfloatsep}
% \begin{macro}{\dbltextfloatsep}
%    当处于双栏模式(two-column mode)时,当跨越整个 |\textwidth| (文本宽度)的浮动对
%    象(floating objects)放置在文本页面上时,浮动和文本之间的分隔由 |\dblfloatsep| 和 |\dbltextfloatsep| 控制。
%
%    |\dblfloatsep| 是移动到文本页面(text page)顶部或底部的相邻浮动(adjacent floats)之间间距(space)。
%
%    |\dbltextfloatsep| 是主文本(main text)和页面顶部或底部浮动之间的间距(space)。
%
%    \begin{macrocode}
%<*10pt>
\setlength\dblfloatsep    {12\p@ \@plus 2\p@ \@minus 2\p@}
\setlength\dbltextfloatsep{20\p@ \@plus 2\p@ \@minus 4\p@}
%</10pt>
%<*11pt>
\setlength\dblfloatsep    {12\p@ \@plus 2\p@ \@minus 2\p@}
\setlength\dbltextfloatsep{20\p@ \@plus 2\p@ \@minus 4\p@}
%</11pt>
%<*12pt>
\setlength\dblfloatsep    {14\p@ \@plus 2\p@ \@minus 4\p@}
\setlength\dbltextfloatsep{20\p@ \@plus 2\p@ \@minus 4\p@}
%</12pt>
%    \end{macrocode}
% \end{macro}
% \end{macro}
%
% \paragraph{浮动在自己的页面或栏上}
%
% \begin{macro}{\@fptop}
% \begin{macro}{\@fpsep}
% \begin{macro}{\@fpbot}
%    当浮动对象(floating objects)放置在单独的页面上时,这些页面的布局由这些参数控制。在页面
%    顶部插入了 |\@fptop| 的可伸缩空白(stretchable whitespace),在页面底部插入了 |\@fpbot| 的可伸缩空白。
%    在相邻的浮动(adjacent floats)之间插入 |\@fpsep|。
%
%    这些参数用于在单栏模式(one-column mode)中放置浮动对象,或在两栏模式(two-column mode)中
%    在单栏浮动(single-column floats)中放置浮动对象。
%
%    请注意,|\@fptop| 和 |\@fpbot| 两个参数中至少有一个应该包含 |plus ...fil|,以便填充
%    剩余的空白(remaining empty space)。
%    \begin{macrocode}
%<*10pt>
\setlength\@fptop{0\p@ \@plus 1fil}
\setlength\@fpsep{8\p@ \@plus 2fil}
\setlength\@fpbot{0\p@ \@plus 1fil}
%</10pt>
%<*11pt>
\setlength\@fptop{0\p@ \@plus 1fil}
\setlength\@fpsep{8\p@ \@plus 2fil}
\setlength\@fpbot{0\p@ \@plus 1fil}
%</11pt>
%<*12pt>
\setlength\@fptop{0\p@ \@plus 1fil}
\setlength\@fpsep{10\p@ \@plus 2fil}
\setlength\@fpbot{0\p@ \@plus 1fil}
%</12pt>
%    \end{macrocode}
% \end{macro}
% \end{macro}
% \end{macro}
%
% \begin{macro}{\@dblfptop}
% \begin{macro}{\@dblfpsep}
% \begin{macro}{\@dblfpbot}
%    双栏模式(two-column mode)中的双栏浮动(double-column floats)使用类似的参数处理。
%    \begin{macrocode}
%<*10pt>
\setlength\@dblfptop{0\p@ \@plus 1fil}
\setlength\@dblfpsep{8\p@ \@plus 2fil}
\setlength\@dblfpbot{0\p@ \@plus 1fil}
%</10pt>
%<*11pt>
\setlength\@dblfptop{0\p@ \@plus 1fil}
\setlength\@dblfpsep{8\p@ \@plus 2fil}
\setlength\@dblfpbot{0\p@ \@plus 1fil}
%</11pt>
%<*12pt>
\setlength\@dblfptop{0\p@ \@plus 1fil}
\setlength\@dblfpsep{10\p@ \@plus 2fil}
\setlength\@dblfpbot{0\p@ \@plus 1fil}
%</12pt>
%<*article|report|book>
%    \end{macrocode}
% \end{macro}
% \end{macro}
% \end{macro}
%
% \subsection[页面样式]{\heiti 页面样式}
%
%    页面样式(page style) \pstyle{foo}\ 是通过定义 |\ps@foo| 命令来定义的。此命令应仅进行
%    局部定义(local definitions)。定义中不应该夹杂空格(stray spaces),因为它们可能导致
%    输出中出现神秘的额外空格(mysterious extra spaces)(这是应该始终避免的)。
%
% \begin{macro}{\@evenhead}
% \begin{macro}{\@oddhead}
% \begin{macro}{\@evenfoot}
% \begin{macro}{\@oddfoot}
%    |\ps@...| 命令定义 |\@oddhead|、|\@oddfoot|、|\@evenhead|等宏,|\@evenfoot| 用于定义
%    栏外标题(running heads)和流水脚注(running feet)---例如,|\@oddhead| 是生成奇数页(odd-numbered pages)
%    标题盒子(heading box)内容的宏。它在一个宽度为 |\textwidth| 的 |\hbox| 中被调用。
% \end{macro}
% \end{macro}
% \end{macro}
% \end{macro}
%
% \subsubsection[标记约定]{\heiti 标记约定}
%
%    要使标题(headings)由分节命令(sectioning commands)确定,页面样式(page style)定义了
%    命令 |\chaptermark|、|\sectionmark|、\ldots,其中 |\chaptermark{|\meta{TEXT}|}|
%    被 |\chapter| 调用以设置标记(mark),依此类推
%
%    在下列宏的帮助下定义了 |\...mark| 命令和 |\...head| 宏。(所有 |\...mark| 命令都应该初始化为 no-ops。)
%
%    \LaTeX{}\ 通过使用以下命令生成“left(左)”和“right(右)”两种标记(mark),扩展了 \TeX\ 的
%    标记功能(|\mark| facility):
%    \begin{flushleft}
%     |\markboth{|\meta{LEFT}|}{|\meta{RIGHT}|}|:添加两个标记。
%
%     |\markright{|\meta{RIGHT}|}|:添加“right(右)”标记。
%
%     |\leftmark|:在 |\@oddhead|、|\@oddfoot|、|\@evenhead| 或 |\@evenfoot| 宏中使用时,
%                  它获得当前的“left(左)”标记。|\leftmark| 的工作方式类似于 \TeX\ 的 |\botmark| 命令。
%
%     |\rightmark|: 在 |\@oddhead|、|\@oddfoot|、|\@evenhead| 或 |\@evenfoot| 宏中使用时,
%                   它获得当前的“right(右)”标记。|\rightmark| 的工作方式类似于 \TeX\ 的 |\firstmark|命令。
%    \end{flushleft}
%
%    标记命令(marking commands)对于“编号在”左标记内(“numbered within”left marks)的右标记(right marks)
%    工作得相当好---例如,左标记由 |\chapter| 命令更改,右标记由 |\section| 命令更改。但是,如果
%    两个 |\markboth| 同时出现在同一个页面上时,它确实会产生一些异常的结果。
%
%
%    像 |\tableofcontents| 这样的命令应该使用 |\@mkboth| 命令在某些页面样式(page styles)中设置标记(set the marks),
%    页面样式命令(pagestyle command)(|\ps@...|) |\let|(让) |\@mkboth| 变成 |\markboth| 用于设置标题(heading),或者
%    |\let|(让) |\@mkboth| 变成 |\@gobbletwo| 设置为不做任何事情。
%
%    [译者注]感觉前一段可能翻译错误!前一段的原文是:
%    Commands like |\tableofcontents| that should set the marks in some
%    page styles use a |\@mkboth| command, which is |\let| by the
%    pagestyle command (|\ps@...|)  to |\markboth| for setting the
%    heading or to |\@gobbletwo| to do nothing.
%
%
%
% \subsubsection[定义页面样式]{\heiti 定义页面样式}
% \label{sec:classes:pagestyle}
%
%    页面样式(pagestyles)的 \pstyle{empty}(空)和 \pstyle{plain}(普通)是在 \LaTeX{}\ 格式中定义的。
%
% \begin{macro}{\ps@headings}
%    双面打印(two sided printing)与单面打印(one sided printing)的页面样式(page style) \pstyle{headings}\ 的定义必须不同。
%
%    \begin{macrocode}
\if@twoside
  \def\ps@headings{%
%    \end{macrocode}
%    在这种页面样式中,流水脚注(running feet)是空的(empty),栏外标题(running head,即短标题)
%    包含页码(page number)和其中一个标记(marks)。
%    \begin{macrocode}
      \let\@oddfoot\@empty\let\@evenfoot\@empty
      \def\@evenhead{\thepage\hfil\slshape\leftmark}%
      \def\@oddhead{{\slshape\rightmark}\hfil\thepage}%
%    \end{macrocode}
%
%    使用此页面样式时,栏外标题(running head,即短标题)的内容由章标题(chapter titles)
%    和节标题(section titles)决定。因此,我们 |\let|(让) |\@mkboth| 变成 |\markboth|。
%    \begin{macrocode}
      \let\@mkboth\markboth
%    \end{macrocode}
%
%    对于 article 文档类,我们定义 |\sectionmark| 来清除右标记(right mark),并将节
%    的编号(当节被编号时)及其标题放在左标记(left mark)中。右标记由 |\subsectionmark| 设置,
%    以包含小节标题(subsection titles)。
%
%    请注意,|\sectionmark| 命令的参数使用 |##1|,该参数将在执行 |\ps@headings| 时定义。
%
% \changes{v1.2z}{1995/04/03}{Removed extra dot after \cs{thesection}
%    (PR 1519)}
% \changes{v1.3c}{1995/05/25}{Replace \cs{hskip}
%    \texttt{1em}\cs{relax} with \cs{quad}}
%    \begin{macrocode}
%<*article>
    \def\sectionmark##1{%
      \markboth {\MakeUppercase{%
        \ifnum \c@secnumdepth >\z@
          \thesection\quad
        \fi
        ##1}}{}}%
    \def\subsectionmark##1{%
      \markright {%
        \ifnum \c@secnumdepth >\@ne
          \thesubsection\quad
        \fi
        ##1}}}
%</article>
%    \end{macrocode}
%
%    在 report 和 book 文档类中,我们使用 |\chaptermark| 和 |\sectionmark| 宏来填充
%    栏外标题(running head,即短标题)。
%
%    请注意,|\chaptermark| 命令的参数使用 |##1|,该参数将在执行 |\ps@headings| 时定义。
%
%    \begin{macrocode}
%<*report|book>
    \def\chaptermark##1{%
      \markboth {\MakeUppercase{%
        \ifnum \c@secnumdepth >\m@ne
%<book>          \if@mainmatter
            \@chapapp\ \thechapter. \ %
%<book>          \fi
        \fi
        ##1}}{}}%
    \def\sectionmark##1{%
      \markright {\MakeUppercase{%
        \ifnum \c@secnumdepth >\z@
          \thesection. \ %
        \fi
        ##1}}}}
%</report|book>
%    \end{macrocode}
%
%    对于单面打印(one sided printing),|\ps@headings| 的定义可以简单得多,因为我们
%    对偶数页和奇数页的处理是一样的。因此,我们不需要定义 |\@even...|。
%    \begin{macrocode}
\else
  \def\ps@headings{%
    \let\@oddfoot\@empty
    \def\@oddhead{{\slshape\rightmark}\hfil\thepage}%
    \let\@mkboth\markboth
%    \end{macrocode}
%    我们现在使用 |\markright|,而不是像双面打印那样使用 |\markboth|。
%    \begin{macrocode}
%<*article>
    \def\sectionmark##1{%
      \markright {\MakeUppercase{%
        \ifnum \c@secnumdepth >\m@ne
          \thesection\quad
        \fi
        ##1}}}}
%</article>
%    \end{macrocode}
%
%    \begin{macrocode}
%<*report|book>
    \def\chaptermark##1{%
      \markright {\MakeUppercase{%
        \ifnum \c@secnumdepth >\m@ne
%<book>          \if@mainmatter
            \@chapapp\ \thechapter. \ %
%<book>          \fi
        \fi
        ##1}}}}
%</report|book>
\fi
%    \end{macrocode}
% \end{macro}
%
% \begin{macro}{\ps@myheadings}
%    页面样式(page style) \pstyle{myheadings}\ 的定义相当简单,因为用户可以使用
%    |\markboth| 和 |\markright| 命令自行确定栏外标题(running head,即短标题)。
%
%    \begin{macrocode}
\def\ps@myheadings{%
    \let\@oddfoot\@empty\let\@evenfoot\@empty
    \def\@evenhead{\thepage\hfil\slshape\leftmark}%
    \def\@oddhead{{\slshape\rightmark}\hfil\thepage}%
%    \end{macrocode}
%
%    我们必须确保禁用章节标题所使用的标记命令(marking commands)。我们这样做,|\let|(让)
%    它们进入一个吞噬其参数的宏。
%    \begin{macrocode}
    \let\@mkboth\@gobbletwo
%<!article>    \let\chaptermark\@gobble
    \let\sectionmark\@gobble
%<article>    \let\subsectionmark\@gobble
    }
%    \end{macrocode}
% \end{macro}
%
%\newpage
%
% \section{\heiti 文档标记}
%
% \subsection[标题]{\heiti 标题}
%
% \begin{macro}{\title}
% \begin{macro}{\author}
% \begin{macro}{\date}
%    这三个宏(macros)由 \LaTeX{}\ 格式提供,用于提供有关文档的标题(title)、作者(authors)
%    和日期(date)的信息。信息存储在内部控制序列(internal control sequences)中。|\maketitle| 命令
%    的任务是使用提供的信息。这里显示了这些宏的定义以供参考。
%    \begin{macrocode}
% \DeclareRobustCommand*{\title}[1]{\gdef\@title{#1}}
% \DeclareRobustCommand*{\author}[1]{\gdef\@author{#1}}
% \DeclareRobustCommand*{\date}[1]{\gdef\@date{#1}}
%    \end{macrocode}
%    默认情况下, |\date| 宏获取今天的日期。
%    \begin{macrocode}
% \date{\today}
%    \end{macrocode}
% \end{macro}
% \end{macro}
% \end{macro}
%
% \begin{macro}{\maketitle}
%    |\maketitle| 的定义取决于是否创建单独的标题页(title page)。这是 report 和
%     book 文档类的默认值,但是对于 article 类,它是可选的。
%
%    制作标题页时,我们在局部重新定义 |\footnotesize| 和 |footnoterule|,以更改
%    |\thanks| 命令生成的脚注的外观;这些变化影响到所有脚注。
% \changes{v1.3o}{1995/11/02}{(CAR) Make \cs{footnote} always work in
%      title, etc}
%    \begin{macrocode}
\if@titlepage
  \newcommand\maketitle{\begin{titlepage}%
  \let\footnotesize\small
  \let\footnoterule\relax
  \let \footnote \thanks
%    \end{macrocode}
%    我们将整个标题垂直居中;通过添加一个 |\vskip|,可以稍微调整中心位置。
%    (在兼容模式下,标题页环境将页码设置为 0,以保持 \LaTeX\ 2.09 样式文件的行为。)
% \changes{v1.0g}{1993/12/09}{Removed the setting of the page number,
%    when not in compatibility mode}
% \changes{v1.2c}{1994/03/17}{Removed setting of page number, now done
%    in titlepage environment}
%    \begin{macrocode}
  \null\vfil
  \vskip 60\p@
%    \end{macrocode}
%    然后我们用 |\LARGE| 字体设置标题(title);留出一点空间(space),用 |\large| 字体
%    设置作者(author)。我们在表格环境(tabular environment)中执行此操作,以便将它们
%    放在单栏(single column)中。在 date(日期)之前,我们又留了一点空白(whitespace)。
%    \begin{macrocode}
  \begin{center}%
    {\LARGE \@title \par}%
    \vskip 3em%
    {\large
     \lineskip .75em%
      \begin{tabular}[t]{c}%
        \@author
      \end{tabular}\par}%
      \vskip 1.5em%
    {\large \@date \par}%       % Set date in \large size.
  \end{center}\par
%    \end{macrocode}
%    然后,我们调用 |\@thanks| 打印进入脚注中的信息并完成页面。
%    \begin{macrocode}
  \@thanks
  \vfil\null
  \end{titlepage}%
%    \end{macrocode}
%    我们重置 \Lcount{footnote}\ 计数器(counter),禁用 |\thanks| 和 |\maketitle|,
%    并通过清空内部信息宏(internal information macros)来节省一些存储空间(storage space)。
% \changes{v1.3j}{1995/08/16}{use \cs{let} to save space}
% \changes{v1.3n}{1995/10/29}{Empty \cs{@date} as well}
%    \begin{macrocode}
  \setcounter{footnote}{0}%
  \global\let\thanks\relax
  \global\let\maketitle\relax
  \global\let\@thanks\@empty
  \global\let\@author\@empty
  \global\let\@date\@empty
  \global\let\@title\@empty
%    \end{macrocode}
%    设置标题后,声明命令 |\title| 等可能会消失(vanish)。|\and| 的定义只有在 |\author| 的参数中才有意义,
%    所以也可以这样做。
% \changes{v1.3k}{1995/08/27}{Disable \cs{title} and similar decls}
%    \begin{macrocode}
  \global\let\title\relax
  \global\let\author\relax
  \global\let\date\relax
  \global\let\and\relax
  }
%    \end{macrocode}
%    当标题不在自己的页面上时,标题的布局略有不同。我们用符号(symbols)来标记脚注(footnotes),
%    我们必须处理两栏文档(two-column documents)。
%
%    因此,我们首先启动一个新组(new group),以保持局部更改(changes local)。然后我们
%    重新定义 |\thefootnote| 以使用 |\fnsymbol|;并更改 |\@makefnmark| 使脚注标记(footnotemarks)的
%    宽度为零(以使作者姓名居中看起来更好)。
% \changes{v1.2s}{1994/06/02}{Reset \cs{@makefntext}}
% \changes{v1.3a}{1995/05/17}{Use \cs{@makefnmark} in definition of
%    \cs{@makefntext}}
% \changes{v1.3g}{1995/06/26}{Fix definition of \cs{@makefnmark} and
%    \cs{@makefntext} to a) work and b) without using math}
%    \begin{macrocode}
\else
  \newcommand\maketitle{\par
    \begingroup
      \renewcommand\thefootnote{\@fnsymbol\c@footnote}%
      \def\@makefnmark{\rlap{\@textsuperscript{\normalfont\@thefnmark}}}%
      \long\def\@makefntext##1{\parindent 1em\noindent
              \hb@xt@1.8em{%
                \hss\@textsuperscript{\normalfont\@thefnmark}}##1}%
%    \end{macrocode}
%    如果这是一个两栏文档(two-column document),我们将以两栏模式(two-column mode)开始一个新页面,
%    标题设置为文本的全宽。标题信息的实际打印由 |\@maketitle| 完成。
% \changes{v1.2k}{1994/05/06}{Added check on number of columns in use
%    locally}
%    \begin{macrocode}
      \if@twocolumn
        \ifnum \col@number=\@ne
          \@maketitle
        \else
          \twocolumn[\@maketitle]%
        \fi
      \else
%    \end{macrocode}
%    当这不是一个两栏文档时,我们只需开始一个新页面,防止浮动对象(floating objects)出现
%    在此页面顶部并打印标题信息(title information)。
%    \begin{macrocode}
      \newpage
        \global\@topnum\z@   % Prevents figures from going at top of page.
        \@maketitle
      \fi
%    \end{macrocode}
%    此页面有一个 \pstyle{plain}\ 的布局。我们称 |\@thanks| 生成脚注。
%    \begin{macrocode}
      \thispagestyle{plain}\@thanks
%    \end{macrocode}
%    现在,我们可以关闭组(group),重置 \Lcount{footnote}\ 计数器,禁用 |\thanks|、|\maketitle| 和
%    |\@maketitle|,并通过清空内部信息宏(internal information macros)来节省一些存储空间(storage space)。
% \changes{v1.3j}{1995/08/16}{use \cs{let} to save space}
% \changes{v1.3k}{1995/08/27}{Disable \cs{title} and similar decls}
% \changes{v1.3n}{1995/10/29}{Empty \cs{@date} as well}
%    \begin{macrocode}
    \endgroup
    \setcounter{footnote}{0}%
    \global\let\thanks\relax
    \global\let\maketitle\relax
    \global\let\@maketitle\relax
    \global\let\@thanks\@empty
    \global\let\@author\@empty
    \global\let\@date\@empty
    \global\let\@title\@empty
    \global\let\title\relax
    \global\let\author\relax
    \global\let\date\relax
    \global\let\and\relax
  }
%    \end{macrocode}
% \end{macro}
%
% \begin{macro}{\@maketitle}
%    当我们没有单独的标题页时,这个宏负责格式化标题信息。
%
%    我们总是开始一个新的页面,留下一些空白(white space)并将信息居中。标题以 |\LARGE| 字体设置,
%    作者姓名(author names)和日期(date)以 |\large| 字体设置。
% \changes{v1.3o}{1995/11/02}{(CAR) Make \cs{footnote} always work in
%      title, etc}
%    \begin{macrocode}
\def\@maketitle{%
  \newpage
  \null
  \vskip 2em%
  \begin{center}%
  \let \footnote \thanks
    {\LARGE \@title \par}%
    \vskip 1.5em%
    {\large
      \lineskip .5em%
      \begin{tabular}[t]{c}%
        \@author
      \end{tabular}\par}%
    \vskip 1em%
    {\large \@date}%
  \end{center}%
  \par
  \vskip 1.5em}
\fi
%    \end{macrocode}
% \end{macro}
%
% \subsection[章和节]{\heiti 章和节}
%
% \subsubsection[构建块]{\heiti 构建块} 类文件(class file)这一部分中的定义使用了两个
%    内部宏(internal macros) |\@startsection| 和 |\secdef|。
%    为了理解这里发生了什么,我们描述了它们的语法。
%
%    |\@startsection| 宏有 6 个必需参数(required arguments),可以选择后跟一个$*$、
%    可选参数(optional argument)和必需参数:
%
%    |\@startsection|\meta{name}\meta{level}\meta{indent}^^A
%                    \meta{beforeskip}\meta{afterskip}\meta{style}
%            optional *\\
%    \null\hphantom{\bslash @startsection}^^A
%            |[|\meta{altheading}|]|\meta{heading}
%
%    这是一个用于启动节(section)的通用命令(generic command),参数具有以下含义:
%
%    \begin{description}
%    \item[\meta{name}] 用户级命令的名称,例如“section”。
%    \item[\meta{level}] 表示节深度的数字--例如,chapter=1、section = 2等。当且仅
%    当 \meta{level} $<=$ \Lcount{secnumdepth}\ 计数器的值时,将打印节编号(section number)。
%    \item[\meta{indent}] 标题从左边距的缩进。
%    \item[\meta{beforeskip}] 此参数的绝对值(absolute value)允许在标题(heading)上方放置垂直
%    空间(skip)。如果为负值(negative),则紧跟标题后面的文本段落缩进(indent)将被抑制。
%    \item[\meta{afterskip}] 如果为正值(positive),则允许在标题(heading)下方放置垂直空间(skip)。
%    否则允许在参选标题(run-in heading)的右侧放置垂直空间(skip)。
%    \item[\meta{style}] 设置标题样式(style of the heading)的命令
%    \item[$*$] 当此项缺失时,将对标题进行编号,并递增相应的计数器(counter)。
%    \item[\meta{altheading}] 提供一个可在目录(table of contents)和栏外标题(running heads)中
%    使用的替代标题(alternative heading)。使用 $*$ 形式时,不应出现这种情况。
%    \item[\meta{heading}] 新节(new section)的标题(heading)。
%    \end{description}
%  分段命令(sectioning command)通常定义为 |\@startsection| 及其前六个参数。
%
%    当不使用 |\@startsection| 来定义分节命令(sectioning command)时,可以使用宏 |\secdef|。
%    它有两个参数:
%
%    |\secdef|\meta{unstarcmds}\meta{starcmds}
%
%    \begin{description}
%    \item[\meta{unstarcmds}] 用于分节命令(sectioning command)的正常形式(normal form)。
%    \item[\meta{starcmds}] 用于分节命令(sectioning command)的 $*$ 形式。
%    \end{description}
%
%    您可以按如下方式使用 |\secdef|:
% \begin{verbatim}
%       \def\chapter { ... \secdef \CMDA \CMDB }
%       \def\CMDA    [#1]#2{ ... }  % Command to define
%                                   % \chapter[...]{...}
%       \def\CMDB    #1{ ... }      % Command to define
%                                   % \chapter*{...}
% \end{verbatim}
%
% \subsubsection[标记命令]{\heiti 标记命令}
%
% \begin{macro}{\chaptermark}
% \begin{macro}{\sectionmark}
% \begin{macro}{\subsectionmark}
% \begin{macro}{\subsubsectionmark}
% \begin{macro}{\paragraphmark}
% \begin{macro}{\subparagraphmark}
%    |\...mark| 命令的默认初始化(default initializations)。这些命令用于定义页面样
%    式(page styles)(见第~\ref{sec:classes:pagestyle}~节),其中大多数已经由 \LaTeX{}\ 格式定义,
%    因此仅在此处显示。
%
%    \begin{macrocode}
%<!article>\newcommand*\chaptermark[1]{}
% \newcommand*\sectionmark[1]{}
% \newcommand*\subsectionmark[1]{}
% \newcommand*\subsubsectionmark[1]{}
% \newcommand*\paragraphmark[1]{}
% \newcommand*\subparagraphmark[1]{}
%    \end{macrocode}
% \end{macro}
% \end{macro}
% \end{macro}
% \end{macro}
% \end{macro}
% \end{macro}
%
% \subsubsection[定义计数器]{\heiti 定义计数器}
%
% \begin{macro}{\c@secnumdepth}
%    计数器 \Lcount{secnumdepth}\ 的值给出了用于生成节号(section numbers)的最高级别
%    的分节命令(sectioning command)的深度(depth)。
%    \begin{macrocode}
%<article>\setcounter{secnumdepth}{3}
%<!article>\setcounter{secnumdepth}{2}
%    \end{macrocode}
% \end{macro}
%
% \begin{macro}{\c@part}
% \begin{macro}{\c@chapter}
% \begin{macro}{\c@section}
% \begin{macro}{\c@subsection}
% \begin{macro}{\c@subsubsection}
% \begin{macro}{\c@paragraph}
% \begin{macro}{\c@subparagraph}
%    这些计数器(counters)用于节编号(section numbers)。宏 |\newcounter{|\meta{newctr}|}[|\meta{oldctr}|]|
%    定义 \meta{newctr}\ 为计数器(counter),当计数器 \meta{oldctr}\ 被步进时(stepped),\meta{newctr}\ 重置
%    为零(reset to zero)。\meta{oldctr}必须已定义。
%
%    \begin{macrocode}
\newcounter {part}
%<article>\newcounter {section}
%<*report|book>
\newcounter {chapter}
\newcounter {section}[chapter]
%</report|book>
\newcounter {subsection}[section]
\newcounter {subsubsection}[subsection]
\newcounter {paragraph}[subsubsection]
\newcounter {subparagraph}[paragraph]
%    \end{macrocode}
% \end{macro}
% \end{macro}
% \end{macro}
% \end{macro}
% \end{macro}
% \end{macro}
% \end{macro}
%
% \begin{macro}{\thepart}
% \begin{macro}{\thechapter}
% \begin{macro}{\thesection}
% \begin{macro}{\thesubsection}
% \begin{macro}{\thesubsubsection}
% \begin{macro}{\theparagraph}
% \begin{macro}{\thesubparagraph}
%    对于任何计数器 \Lcount{CTR},|\theCTR| 是定义计数器 \Lcount{CTR}\ 的打印版本
%    (printed version)的宏。它是根据以下宏来定义的:
%
%    |\arabic{|\Lcount{COUNTER}|}| 将 \Lcount{COUNTER}\ 的值打印为阿拉伯数字(arabic numeral)。
%
%    |\roman{|\Lcount{COUNTER}|}| 将 \Lcount{COUNTER}\ 的值打印为小写罗马数字(lowercase roman numeral)。
%
%    |\Roman{|\Lcount{COUNTER}|}| 将 \Lcount{COUNTER}\ 的值打印为大写罗马数字(uppercase roman numeral)。
%
%    |\alph{|\Lcount{COUNTER}|}| 将 \Lcount{COUNTER}\ 的值打印为小写字母(lowercase letter):$1 =$~a、$2 =$~ b等。
%
%    |\Alph{|\Lcount{COUNTER}|}| 将 \Lcount{COUNTER}\ 的值打印为大写字母(uppercase letter):$1 =$~A、$2 =$~B等。
%
%    实际上,为了节省空间,我们使用了内部计数器表示法(internal counter representations)和操作这些表示法的命令。
%    \begin{macrocode}
\renewcommand \thepart {\@Roman\c@part}
%<article>\renewcommand \thesection {\@arabic\c@section}
%<*report|book>
\renewcommand \thechapter {\@arabic\c@chapter}
\renewcommand \thesection {\thechapter.\@arabic\c@section}
%</report|book>
\renewcommand\thesubsection   {\thesection.\@arabic\c@subsection}
\renewcommand\thesubsubsection{\thesubsection.\@arabic\c@subsubsection}
\renewcommand\theparagraph    {\thesubsubsection.\@arabic\c@paragraph}
\renewcommand\thesubparagraph {\theparagraph.\@arabic\c@subparagraph}
%    \end{macrocode}
% \end{macro}
% \end{macro}
% \end{macro}
% \end{macro}
% \end{macro}
% \end{macro}
% \end{macro}
%
% \begin{macro}{\@chapapp}
%    |\@chapapp| 最初定义为“|\chaptername|”。|\appendix| 命令将其重新定义为“|\appendixname|”。
%
%    \begin{macrocode}
%<report|book>\newcommand\@chapapp{\chaptername}
%    \end{macrocode}
% \end{macro}
%
%  \subsubsection[前言、正文和附属资料]{\heiti 前言、正文和附属资料}

%    一本书(book)包含这三个(逻辑)节(sections)\footnote{[译者注]这三个逻辑节(logical sections)就是:
%    前言(front Matter)、正文(main Matter)、附属资料(back Matter)。前言包括诸如标题页、
%    摘要、目录、序言、注释清单、数字清单和表格清单,其中一些前题页如标题页传统上是
%    不编号的;附属资料可能包括词汇表、注释、参考书目和索引等。}。如果我们正在处理
%    正文(main Matter),则开关 |\@mainmatter| 为 true。 当此开关为 false 时,
%    |\chapter| 命令不打印章编号(chapter numbers)。
%
%    在这里,我们定义了启动这些节(sections)的命令。
%  \begin{macro}{\frontmatter}
%    此命令启动罗马页码编号(Roman page numbering)并关闭章编号(chapter numbering)。
%    由于这会从 1 开始重新启动页面编号,因此还应确保使用的是右页(recto page)。
% \changes{v1.3r}{1996/05/26}{Make this command react to the option
%    \texttt{openany}}
% \changes{v1.3y}{1998/05/05}{Two years on: Make this command not
%    react to the option \texttt{openany} as this makes the
%    verso/recto numbering wrong: see pr/2754 for discussion}
%    \begin{macrocode}
%<*book>
\newcommand\frontmatter{%
%   \if@openright
    \cleardoublepage
%   \else
%     \clearpage
%   \fi
  \@mainmatterfalse
  \pagenumbering{roman}}
%    \end{macrocode}
%  \end{macro}
%
%  \begin{macro}{\mainmatter}
%    此命令清除页面,开始阿拉伯数字页码编号(arabic page numbering),并打开章编号(chapter numbering)。
%    由于这会从 1 开始重新启动页面编号,因此还应确保使用的是右页(recto page)。
% \changes{v1.3r}{1996/05/26}{Make this command react to the option
%    \texttt{openany}}
% \changes{v1.3y}{1998/05/05}{Two years on: Make this command not
%    react to the option \texttt{openany} as this makes the
%    verso/recto numbering wrong: see pr/2754 for discussion}
%    \begin{macrocode}
\newcommand\mainmatter{%
%   \if@openright
    \cleardoublepage
%   \else
%     \clearpage
%   \fi
  \@mainmattertrue
  \pagenumbering{arabic}}
%    \end{macrocode}
%  \end{macro}
%
%  \begin{macro}{\backmatter}
%    这将清除页面,关闭章编号(chapter numbering)并保持页面编号不变。
%    \begin{macrocode}
\newcommand\backmatter{%
  \if@openright
    \cleardoublepage
  \else
    \clearpage
  \fi
  \@mainmatterfalse}
%</book>
%    \end{macrocode}
%  \end{macro}
%
% \subsubsection[部分]{\heiti 部分}
%
% \begin{macro}{\part}
%    启动文档新部分(new part)的命令。
%
%    在 article 类中,|\part| 的定义相当简单;我们开始一个新段落(new paragraph),
%    添加一点空白(white space),取消第一段的缩进,并使用 |\secdef|。与其他分节命令
%    (sectioning commands)({\LaTeXe}\ 内核中的 cf.\ |\@startsection|)一样,
%    我们需要检查 |@noskipsec| 开关并强制水平模式(horizontal mode)(如果设置了)。
% \changes{v1.4a}{1999/01/07}{Check \texttt{@noskipsec} switch and
%      possibly force horizontal mode; see PR/2889.}
%    \begin{macrocode}
%<*article>
\newcommand\part{%
   \if@noskipsec \leavevmode \fi
   \par
   \addvspace{4ex}%
   \@afterindentfalse
   \secdef\@part\@spart}
%</article>
%    \end{macrocode}
%
%    对于 report 和 book 类,我们的做法有些不同。
%
%    我们启动一个新的(右边的)页面并使用 \pstyle{plain}\ 页面样式(pagestyle)。
% \changes{v1.3r}{1996/05/26}{Make this command react to the option
%    \texttt{openany}}
%    \begin{macrocode}
%<*report|book>
\newcommand\part{%
  \if@openright
    \cleardoublepage
  \else
    \clearpage
  \fi
  \thispagestyle{plain}%
%    \end{macrocode}
%    当我们制作两栏文档(two-column document)时,这将是一个单栏页面(one column page)。
%    我们使用 |@tempswa| 来记住切换回两栏。
%    \begin{macrocode}
  \if@twocolumn
    \onecolumn
    \@tempswatrue
  \else
    \@tempswafalse
  \fi
%    \end{macrocode}
%    We need an empty box to prevent the fil glue from disappearing.
%    我们需要一个空盒子来防止空白粘连(fil glue)消失。
% \changes{v1.3j}{1995/08/16}{Replace \cs{hbox} by \cs{null}}
%    \begin{macrocode}
  \null\vfil
%    \end{macrocode}
%    在这里,我们使用 |\secdef| 来指示使用哪些命令来创建实际的标题(actual heading)。
%    \begin{macrocode}
  \secdef\@part\@spart}
%</report|book>
%    \end{macrocode}
%
% \begin{macro}{\@part}
%    此宏执行部分的标题(title of the part)的实际格式(actual formatting)。同样,
%    article 文档类的宏定义与 report 和 book 文档类的宏定义不同。

%    当 article 文档类的 \Lcount{secnumdepth}\ 大于 $-1$ 时,我们有一个已编号的部分(numbered part),
%    否则它就是未编号的(unnumbered)。
%    \begin{macrocode}
%<*article>
\def\@part[#1]#2{%
    \ifnum \c@secnumdepth >\m@ne
      \refstepcounter{part}%
      \addcontentsline{toc}{part}{\thepart\hspace{1em}#1}%
    \else
      \addcontentsline{toc}{part}{#1}%
    \fi
%    \end{macrocode}
%    我们打印 article 类中左对齐的标题(title flush left)。此外,我们还防止断行和重新设置字体。
% \changes{v1.3c}{1995/05/25}{replace \cs{reset@font} with
%    \cs{normalfont}}
%    \begin{macrocode}
    {\parindent \z@ \raggedright
     \interlinepenalty \@M
     \normalfont
%    \end{macrocode}
%    当这是一个已编号的部分(numbered part)时,我们必须打印编号(number)number和标题(title)。
%    此处的 |\nobreak| 应防止分页。
% \changes{v1.4e}{2001/05/24}{Replaced tilde with \cs{nobreakspace}
%                             (pr/3310)}
%    \begin{macrocode}
     \ifnum \c@secnumdepth >\m@ne
       \Large\bfseries \partname\nobreakspace\thepart
       \par\nobreak
     \fi
     \huge \bfseries #2%
%    \end{macrocode}
%    现在我们清空标记寄存器(mark registers),留下一些空白(white space),并让 |\@afterheading| 负责抑制缩进。
%    \begin{macrocode}
     \markboth{}{}\par}%
    \nobreak
    \vskip 3ex
    \@afterheading}
%</article>
%    \end{macrocode}
%
%    对于 report 和 book 文档类,当 \Lcount{secnumdepth}\ 大于 $-2$,我们有一个
%    已编号的部分(numbered part),否则就没有编号(unnumbered)。
%    \begin{macrocode}
%<*report|book>
\def\@part[#1]#2{%
    \ifnum \c@secnumdepth >-2\relax
      \refstepcounter{part}%
      \addcontentsline{toc}{part}{\thepart\hspace{1em}#1}%
    \else
      \addcontentsline{toc}{part}{#1}%
    \fi
%    \end{macrocode}
%    我们清空标记寄存器(mark registers),并将标题放在 report 和 book 文档类的页面中央。
%    此外,我们还防止断行和重新设置字体。
% \changes{v1.3c}{1995/05/25}{replace \cs{reset@font} with
%    \cs{normalfont}}
% \changes{v1.3j}{1995/08/16}{add missing percent}
%    \begin{macrocode}
    \markboth{}{}%
    {\centering
     \interlinepenalty \@M
     \normalfont
%    \end{macrocode}
%    当这是一个已编号的部分(numbered part)时,我们必须打印编号(number)。
% \changes{v1.4e}{2001/05/24}{Replaced tilde with \cs{nobreakspace}
%                             (pr/3310)}
%    \begin{macrocode}
     \ifnum \c@secnumdepth >-2\relax
       \huge\bfseries \partname\nobreakspace\thepart
       \par
%    \end{macrocode}
%    在打印标题之前,我们留出一些空间(space),并将收尾工作(finishing)留给 |\@endpart|。
%    \begin{macrocode}
       \vskip 20\p@
     \fi
     \Huge \bfseries #2\par}%
    \@endpart}
%</report|book>
%    \end{macrocode}
% \end{macro}
%
% \begin{macro}{\@spart}
%    当使用用户命令的星形形式(star form)时,此宏执行部分标题(title of the part)的
%    实际格式(actual formatting)。在这种情况下,我们{\kaiti 从不}打印编号(number)。
%    否则格式相同。
%
%    该宏在 article 文档类中的定义与在 report 和 book 文档类中的定义的差异与
%    它们在 |\@part| 中定义的差异相同。
% \changes{v1.3c}{1995/05/25}{replace \cs{reset@font} with
%    \cs{normalfont}}
%    \begin{macrocode}
%<*article>
\def\@spart#1{%
    {\parindent \z@ \raggedright
     \interlinepenalty \@M
     \normalfont
     \huge \bfseries #1\par}%
     \nobreak
     \vskip 3ex
     \@afterheading}
%</article>
%<*report|book>
\def\@spart#1{%
    {\centering
     \interlinepenalty \@M
     \normalfont
     \Huge \bfseries #1\par}%
    \@endpart}
%</report|book>
%    \end{macrocode}
% \end{macro}
%
% \begin{macro}{\@endpart}
% \changes{v1.3j}{1995/08/16}{move docstrip guard to avoid defining
%    \cs{@endpart} in article}
%    此宏完成了 |\@part| 和 |\@spart| 的部分页(part page)。
%
%    首先我们填充当前页面(current page)。
%    \begin{macrocode}
%<*report|book>
\def\@endpart{\vfil\newpage
%    \end{macrocode}
%    然后,当我们处于双面模式(twosided mode)并且章(chapters)应该位于右侧时,我们会
%    生成一个完全空白的页面(completely blank page)。
% \changes{v1.4b}{2000/05/19}{Only add empty page after part if
%    twoside and openright (pr/3155)}
%    \begin{macrocode}
              \if@twoside
               \if@openright
                \null
                \thispagestyle{empty}%
                \newpage
               \fi
              \fi
%    \end{macrocode}
%    当这是一个两栏文档(two-column document)时,我们必须切换回两栏模式(two-column mode)。
%    \begin{macrocode}
              \if@tempswa
                \twocolumn
              \fi}
%</report|book>
%    \end{macrocode}
% \end{macro}
% \end{macro}
%
% \subsubsection[章]{\heiti 章}
%
% \begin{macro}{\chapter}
%    章(chapter)应该总是从新页面开始,因此我们首先调用 |\clearpage| 并将此页面的页
%    面样式(pagestyle)设置为 \pstyle{plain}。
%    \begin{macrocode}
%<*report|book>
\newcommand\chapter{\if@openright\cleardoublepage\else\clearpage\fi
                    \thispagestyle{plain}%
%    \end{macrocode}
%    然后我们防止浮动(floats)出现在页面顶部,因为在章标题上方有浮动对象(floating object)看起来很奇怪。
%    \begin{macrocode}
                    \global\@topnum\z@
%    \end{macrocode}
%    然后,我们通过将 |\@afterindent| 开关设置为 |false| 来抑制第一段的缩进。
%    我们使用 |\secdef| 指定用于实际设置章标题(chapter title)的宏。
%    \begin{macrocode}
                    \@afterindentfalse
                    \secdef\@chapter\@schapter}
%    \end{macrocode}
%
% \begin{macro}{\@chapter}
%    当我们有一个已编号的章(numbered chapter)时,就会调用这个宏。当 \Lcount{secnumdepth}\ 大
%    于 $-1$ 时,并且在 boo k类中, |\@mainmatter| 为 true,我们显示章编号(chapter number)。
%    我们还通过向终端(terminal)写入消息来通知用户即将排版新章。
%    \begin{macrocode}
\def\@chapter[#1]#2{\ifnum \c@secnumdepth >\m@ne
%<book>                       \if@mainmatter
                         \refstepcounter{chapter}%
                         \typeout{\@chapapp\space\thechapter.}%
                         \addcontentsline{toc}{chapter}%
                                   {\protect\numberline{\thechapter}#1}%
%<*book>
                       \else
                         \addcontentsline{toc}{chapter}{#1}%
                       \fi
%</book>
                    \else
                      \addcontentsline{toc}{chapter}{#1}%
                    \fi
%    \end{macrocode}
%    在将条目(entry)写入目录(table of contents)后,我们将本章(可选)标题存储
%    为 |\chaptermark|,并在图(figures)和表(tables)列表中添加一些空白。
%    \begin{macrocode}
                    \chaptermark{#1}%
                    \addtocontents{lof}{\protect\addvspace{10\p@}}%
                    \addtocontents{lot}{\protect\addvspace{10\p@}}%
%    \end{macrocode}
%    然后我们调用 |\@makechapterhead| 来格式化实际的章标题(actual chapter title)。
%    当我们处于两栏模式(two-column mode)时,为了让章标题使用整个 |\textwidth| 宽度,
%    我们必须以一种特殊的方式进行此操作。在单栏模式(one-column mode)中,我们调
%    用 |\@afterheading| 来抑制缩进
%    \begin{macrocode}
                    \if@twocolumn
                      \@topnewpage[\@makechapterhead{#2}]%
                    \else
                      \@makechapterhead{#2}%
                      \@afterheading
                    \fi}
%    \end{macrocode}
%
% \begin{macro}{\@makechapterhead}
%    上面的宏使用 |\@makechapterhead|\meta{text}\ 来格式化章标题(heading of the chapter)。
%
%    我们首先留一些空白(white space)。我们打开了一个组(group),其中段落缩进为 0pt,
%    文本设置为右侧不对齐(ragged)。我们还重置了字体。
% \changes{v1.3c}{1995/05/25}{replace \cs{reset@font} with
%    \cs{normalfont}}
%    \begin{macrocode}
\def\@makechapterhead#1{%
  \vspace*{50\p@}%
  {\parindent \z@ \raggedright \normalfont
%    \end{macrocode}
%    然后我们检查是否必须打印章编号。如果是这样,我们在章编号(chapternumber)和标题之间留一些空白(whitespace)。
% \changes{v1.2v}{1994/11/30}{Added a \cs{nobreak} to prevent a
%    pagebreak between the chapternumber and the chaptertitle}
% \changes{v1.3j}{1995/08/16}{replace braces by \cs{space}}
%    \begin{macrocode}
    \ifnum \c@secnumdepth >\m@ne
%<book>      \if@mainmatter
        \huge\bfseries \@chapapp\space \thechapter
        \par\nobreak
        \vskip 20\p@
%<book>      \fi
    \fi
%    \end{macrocode}
%    现在我们将标题设置为大号粗体(large bold)字体。我们防止分页符(pagebreak)出现在
%    标题的中间或后面。最后,在文本开始之前,我们留下一些空白。
% \changes{v1.2v}{1994/11/30}{Added \cs{interlinepenalty}\cs{@M} to
%    prevent a pagebreak in the middle of a title}
%    \begin{macrocode}
    \interlinepenalty\@M
    \Huge \bfseries #1\par\nobreak
    \vskip 40\p@
  }}
%    \end{macrocode}
% \end{macro}
% \end{macro}
%
% \begin{macro}{\@schapter}
%    当我们有一个未编号的章(unnumbered chapter)时,就会调用此宏。它比 |\@chapter| 简单得多,
%    因为它只需要排版章标题(chapter title)。
%    \begin{macrocode}
\def\@schapter#1{\if@twocolumn
                   \@topnewpage[\@makeschapterhead{#1}]%
                 \else
                   \@makeschapterhead{#1}%
                   \@afterheading
                 \fi}
%    \end{macrocode}
%
% \begin{macro}{\@makeschapterhead}
%    上面的宏使用 |\@makeschapterhead|\meta{text}\ 格式化章标题。它与 |\@makechapterhead| 类似,
%    只是它不必打印章编号(chapter number)。
%
% \changes{v1.2v}{1994/11/30}{Added \cs{interlinepenalty}\cs{@M} to
%    prevent a pagebreak in the middle of a title}
% \changes{v1.3c}{1995/05/25}{replace \cs{reset@font} with
%    \cs{normalfont}}
%    \begin{macrocode}
\def\@makeschapterhead#1{%
  \vspace*{50\p@}%
  {\parindent \z@ \raggedright
    \normalfont
    \interlinepenalty\@M
    \Huge \bfseries  #1\par\nobreak
    \vskip 40\p@
  }}
%</report|book>
%    \end{macrocode}
% \end{macro}
% \end{macro}
% \end{macro}
%
%
% \subsubsection[下级标题]{\heiti 下级标题}
%
%    这些命令都使用 |\@startsection|。
% \begin{macro}{\section}
%    这给出了一个普通的标题(normal heading),标题上下都有空白(space),标题设置
%    为 |\Large\bfseries|,第一段没有缩进。
% \changes{v1.3c}{1995/05/25}{replace \cs{reset@font} with
%    \cs{normalfont}}
%    \begin{macrocode}
\newcommand\section{\@startsection {section}{1}{\z@}%
                                   {-3.5ex \@plus -1ex \@minus -.2ex}%
                                   {2.3ex \@plus.2ex}%
                                   {\normalfont\Large\bfseries}}
%    \end{macrocode}
% \end{macro}
%
% \begin{macro}{\subsection}
%    这给出了一个普通的标题(normal heading),标题上下都有空白(space),标题设置
%    为 |\large\bfseries|,第一段没有缩进。
%    \begin{macrocode}
\newcommand\subsection{\@startsection{subsection}{2}{\z@}%
                                     {-3.25ex\@plus -1ex \@minus -.2ex}%
                                     {1.5ex \@plus .2ex}%
                                     {\normalfont\large\bfseries}}
%    \end{macrocode}
% \end{macro}
%
% \begin{macro}{\subsubsection}
%    这将给出一个普通的标题(normal heading),标题上方和下方都有空白(space),标题设置
%    为 |\normalsize\bfseries|,第一段没有缩进。
%    \begin{macrocode}
\newcommand\subsubsection{\@startsection{subsubsection}{3}{\z@}%
                                     {-3.25ex\@plus -1ex \@minus -.2ex}%
                                     {1.5ex \@plus .2ex}%
                                     {\normalfont\normalsize\bfseries}}
%    \end{macrocode}
% \end{macro}
%
% \begin{macro}{\paragraph}
%    这将在标题上方和右侧提供一个带有空白的栏外标题(run-in heading),
%    标题设置为 |\normalsize\bfseries|。
%    \begin{macrocode}
\newcommand\paragraph{\@startsection{paragraph}{4}{\z@}%
                                    {3.25ex \@plus1ex \@minus.2ex}%
                                    {-1em}%
                                    {\normalfont\normalsize\bfseries}}
%    \end{macrocode}
% \end{macro}
%
% \begin{macro}{\subparagraph}
%    这将在标题上方和右侧显示缩进的带有空白的栏外标题(run-in heading),标题设置
%    为 |\normalsize\bfseries|。
%    \begin{macrocode}
\newcommand\subparagraph{\@startsection{subparagraph}{5}{\parindent}%
                                       {3.25ex \@plus1ex \@minus .2ex}%
                                       {-1em}%
                                      {\normalfont\normalsize\bfseries}}
%    \end{macrocode}
% \end{macro}
%
% \subsection[列表]{\heiti 列表}
%
% \subsubsection[普通的列表参数]{\heiti 普通的列表参数}
%
% 以下命令用于设置列表环境(list environment)参数的默认值。有关参数含义的解释,请参阅 \LaTeX{}\ 手册。
% 列表环境的默认设置如下。首先,将 |\rightmargin|、|\listparindent| 和 |\itemindent| 设置为0pt。
% 然后,对于第 K 级列表(Kth level list),调用命令 |\@listK|,其中“K”表示“i”、“ii”、...、“vi”。
% (即,第三级列表调用 |\@listiii|。)按照惯例,|\@listK| 应将 |\leftmargin| 设置为 |\leftmarginK|。
%
% \begin{macro}{\leftmargin}
% \begin{macro}{\leftmargini}
% \begin{macro}{\leftmarginii}
% \begin{macro}{\leftmarginiii}
% \begin{macro}{\leftmarginiv}
% \begin{macro}{\leftmarginv}
% \begin{macro}{\leftmarginvi}
% \changes{v1.0m}{1994/01/12}{Use em instead of pt to remain
%    compatible with old styles}
% \changes{v1.3q}{1995/12/20}{Temporary(?) fix: revert to setting
%    \cs{leftmargin} at outer level}
%
% 当我们处于两栏模式(two-column mode)时,有些页边距设置得稍微小一些。
%    \begin{macrocode}
\if@twocolumn
  \setlength\leftmargini  {2em}
\else
  \setlength\leftmargini  {2.5em}
\fi
%    \end{macrocode}
%    在这些文件中的所有参数设置合理化之前,我们需要在此外部级别(outer level)设置 |\leftmargin| 的值。
%    \begin{macrocode}
\leftmargin  \leftmargini
%    \end{macrocode}
%    计算以下三个值,使其大于 |\labelsep| 和默认标签(default labels)宽度之和
%    默认标签即“(m)”、“vii.”和“M.”。
%    \begin{macrocode}
\setlength\leftmarginii  {2.2em}
\setlength\leftmarginiii {1.87em}
\setlength\leftmarginiv  {1.7em}
\if@twocolumn
  \setlength\leftmarginv  {.5em}
  \setlength\leftmarginvi {.5em}
\else
  \setlength\leftmarginv  {1em}
  \setlength\leftmarginvi {1em}
\fi
%    \end{macrocode}
% \end{macro}
% \end{macro}
% \end{macro}
% \end{macro}
% \end{macro}
% \end{macro}
% \end{macro}
%
% \begin{macro}{\labelsep}
% \begin{macro}{\labelwidth}
% \changes{v1.0m}{1994/01/12}{Use em instead of pt to remain
%    compatible with old styles}
%    |\labelsep| 是标签(label)和项目文本(text of an item)之间的距离;|\labelwidth| 是标签的宽度。
%    \begin{macrocode}
\setlength  \labelsep  {.5em}
\setlength  \labelwidth{\leftmargini}
\addtolength\labelwidth{-\labelsep}
%    \end{macrocode}
% \end{macro}
% \end{macro}
%
% \begin{macro}{\partopsep}
%    当用户在环境前留下空行(blank line)时,除了 |\parskip| 和 |\topsep| 之外,还会
%    插入一个额外的垂直空间 |\partopsep|。
% \changes{v1.0m}{1994/01/12}{\cs{partopsep} should be different,
%    depending on the pointsize}
%    \begin{macrocode}
%</article|report|book>
%<10pt>\setlength\partopsep{2\p@ \@plus 1\p@ \@minus 1\p@}
%<11pt>\setlength\partopsep{3\p@ \@plus 1\p@ \@minus 1\p@}
%<12pt>\setlength\partopsep{3\p@ \@plus 2\p@ \@minus 2\p@}
%    \end{macrocode}
% \end{macro}
%
% \begin{macro}{\@beginparpenalty}
% \begin{macro}{\@endparpenalty}
%    这些惩罚(penalties)在列表(list)或段落(paragraph)环境前后插入。
%    它们被设置为奖金值(bonus value),以鼓励在这些点上分页。
% \begin{macro}{\@itempenalty}
%    在列表项(list items)之间插入此惩罚(penalty)。
%    \begin{macrocode}
%<*article|report|book>
\@beginparpenalty -\@lowpenalty
\@endparpenalty   -\@lowpenalty
\@itempenalty     -\@lowpenalty
%</article|report|book>
%    \end{macrocode}
% \end{macro}
% \end{macro}
% \end{macro}
%
% \begin{macro}{\@listi}
% \begin{macro}{\@listI}
% |\@listi| 为显示在顶层(top-level)的列表定义了 |\leftmargin|、|\parsep|、|\topsep|、|\itemsep| 等的值。
% 它的定义由字体尺寸命令(font-size commands)修改,例如,在 |\small| 中,列表参数的值“更小(smaller)”。
%
% 由于这个原因,\@listI\ 被定义为保存 \@listi\ 的一个副本,以便 |\normalsize| 可以将所有参数切换回来。
%
%    \begin{macrocode}
%<*10pt|11pt|12pt>
\def\@listi{\leftmargin\leftmargini
%<*10pt>
            \parsep 4\p@ \@plus2\p@ \@minus\p@
            \topsep 8\p@ \@plus2\p@ \@minus4\p@
            \itemsep4\p@ \@plus2\p@ \@minus\p@}
%</10pt>
%<*11pt>
            \parsep 4.5\p@ \@plus2\p@ \@minus\p@
            \topsep 9\p@   \@plus3\p@ \@minus5\p@
            \itemsep4.5\p@ \@plus2\p@ \@minus\p@}
%</11pt>
%<*12pt>
            \parsep 5\p@  \@plus2.5\p@ \@minus\p@
            \topsep 10\p@ \@plus4\p@   \@minus6\p@
            \itemsep5\p@  \@plus2.5\p@ \@minus\p@}
%</12pt>
\let\@listI\@listi
%    \end{macrocode}
%    我们对参数进行了初始化(initialise),尽管严格来说这是不必要的。
%    \begin{macrocode}
\@listi
%    \end{macrocode}
% \end{macro}
% \end{macro}
%
% \begin{macro}{\@listii}
% \begin{macro}{\@listiii}
% \begin{macro}{\@listiv}
% \begin{macro}{\@listv}
% \begin{macro}{\@listvi}
%    下面是用于更高层级列表的相同宏。请注意,它们没有保存的版本(saved versions),
%    也不会被字体尺寸命令(font-size commands)修改。换句话说,此类假定嵌套列表(nested lists)
%    仅以 |\normalsize| 显示,即主文档尺寸(document size)。
%    \begin{macrocode}
\def\@listii {\leftmargin\leftmarginii
              \labelwidth\leftmarginii
              \advance\labelwidth-\labelsep
%<*10pt>
              \topsep    4\p@ \@plus2\p@ \@minus\p@
              \parsep    2\p@ \@plus\p@  \@minus\p@
%</10pt>
%<*11pt>
              \topsep    4.5\p@ \@plus2\p@ \@minus\p@
              \parsep    2\p@   \@plus\p@  \@minus\p@
%</11pt>
%<*12pt>
              \topsep    5\p@   \@plus2.5\p@ \@minus\p@
              \parsep    2.5\p@ \@plus\p@    \@minus\p@
%</12pt>
              \itemsep   \parsep}
\def\@listiii{\leftmargin\leftmarginiii
              \labelwidth\leftmarginiii
              \advance\labelwidth-\labelsep
%<10pt>              \topsep    2\p@ \@plus\p@\@minus\p@
%<11pt>              \topsep    2\p@ \@plus\p@\@minus\p@
%<12pt>              \topsep    2.5\p@\@plus\p@\@minus\p@
              \parsep    \z@
              \partopsep \p@ \@plus\z@ \@minus\p@
              \itemsep   \topsep}
\def\@listiv {\leftmargin\leftmarginiv
              \labelwidth\leftmarginiv
              \advance\labelwidth-\labelsep}
\def\@listv  {\leftmargin\leftmarginv
              \labelwidth\leftmarginv
              \advance\labelwidth-\labelsep}
\def\@listvi {\leftmargin\leftmarginvi
              \labelwidth\leftmarginvi
              \advance\labelwidth-\labelsep}
%</10pt|11pt|12pt>
%    \end{macrocode}
% \end{macro}
% \end{macro}
% \end{macro}
% \end{macro}
% \end{macro}
%
% \subsubsection[排序列表]{\heiti 排序列表}
%
%    排序列表环境(enumerate environment)使用四个计数器(counters):\Lcount{enumi}、
%    \Lcount{enumii}、\Lcount{enumiii} 和 \Lcount{enumiv},其中 \Lcount{enumN}\ 控制
%    第 N 级排序(enumeration)的编号(numbering)。
%
% \begin{macro}{\theenumi}
% \begin{macro}{\theenumii}
% \begin{macro}{\theenumiii}
% \begin{macro}{\theenumiv}
%    计数器已定义为 \LaTeX{}\ 格式,但此处更改了它们的表示形式(representation)。
%
%    \begin{macrocode}
%<*article|report|book>
\renewcommand\theenumi{\@arabic\c@enumi}
\renewcommand\theenumii{\@alph\c@enumii}
\renewcommand\theenumiii{\@roman\c@enumiii}
\renewcommand\theenumiv{\@Alph\c@enumiv}
%    \end{macrocode}
% \end{macro}
% \end{macro}
% \end{macro}
% \end{macro}
%
% \begin{macro}{\labelenumi}
% \begin{macro}{\labelenumii}
% \begin{macro}{\labelenumiii}
% \begin{macro}{\labelenumiv}
%    每个项目(item)的标签(label)由 |\labelenumi| \ldots\ |\labelenumiv| 命令生成。
%    \begin{macrocode}
\newcommand\labelenumi{\theenumi.}
\newcommand\labelenumii{(\theenumii)}
\newcommand\labelenumiii{\theenumiii.}
\newcommand\labelenumiv{\theenumiv.}
%    \end{macrocode}
% \end{macro}
% \end{macro}
% \end{macro}
% \end{macro}
%
% \begin{macro}{\p@enumii}
% \begin{macro}{\p@enumiii}
% \begin{macro}{\p@enumiv}
%    在引用排序列表的第 N 层级的项目时,|\p@enumN||\theenumN| 的展开(expansion)定义
%     |\ref| 命令的输出。
%    \begin{macrocode}
\renewcommand\p@enumii{\theenumi}
\renewcommand\p@enumiii{\theenumi(\theenumii)}
\renewcommand\p@enumiv{\p@enumiii\theenumiii}
%    \end{macrocode}
% \end{macro}
% \end{macro}
% \end{macro}
%
% \subsubsection[常规列表]{\heiti 常规列表}
%
% \begin{macro}{\labelitemi}
% \begin{macro}{\labelitemii}
% \changes{v1.2k}{1994/05/06}{Inserted \cs{normalfont}}
% \changes{v1.3s}{1996/08/24}{Replaced -{}- with \cs{textendash}}
% \changes{v1.3u}{1996/10/31}{Changed to \cs{textbullet},
%                 \cs{textasteriskcentered} and \cs{textperiodcentered}}
% \begin{macro}{\labelitemiii}
% \begin{macro}{\labelitemiv}
%    项目(itemization)由四个命令控制:|\labelitemi|、|\labelitemii|、|\labelitemiii| 和 |\labelitemiv|,
%    它们定义了各种项目级别的标签:使用的符号有黑点(bullet)、粗体短划线(bold en-dash)、
%    居中星号(centered asterisk)和居中的点(centred dot)。
%
% \changes{v1.4n}{2021/06/11}{Drop incorrect space in \cs{labelitemiv} (gh/496)}
%    \begin{macrocode}
\newcommand\labelitemi  {\labelitemfont \textbullet}
\newcommand\labelitemii {\labelitemfont \bfseries \textendash}
\newcommand\labelitemiii{\labelitemfont \textasteriskcentered}
\newcommand\labelitemiv {\labelitemfont \textperiodcentered}
%    \end{macrocode}
%
% \begin{macro}{\labelitemfont}
%    \cs{labelitemfont}\ 的默认定义(default definition)是将字体重置为 \cs{normalfont},
%    以便不管周围的条件如何,始终生成相同的符号(same symbol)。
%
% \changes{v1.4l}{2019/12/20}{Normalize label fonts}
%    可能的替代方案(alternatives)是,例如:
%\begin{verbatim}
%  \renewcommand\labelitemfont
%     {\normalfont\fontfamily{lmss}\selectfont}
%  \renewcommand\labelitemfont
%     {\rmfamily\normalshape}
%\end{verbatim}
%    第一种方法将使用 Latin Modern Sans 的符号,第二种方法将只允许更改字体序列(font series),
%    以便在粗体上下文中的 \texttt{itemize}\ 产成生更粗体的符号(bolder symbols)。
%    \begin{macrocode}
\newcommand\labelitemfont{\normalfont}
%    \end{macrocode}
% \end{macro}
% \end{macro}
% \end{macro}
% \end{macro}
% \end{macro}
%
% \subsubsection[解说列表]{\heiti 解说列表}
%
% \begin{environment}{description}
%    这里定义了解说列表环境(description environment) -- 而 itemize 和 enumerate 环境
%    是以 \LaTeX{}\ 格式定义的。
%
%    \begin{macrocode}
\newenvironment{description}
               {\list{}{\labelwidth\z@ \itemindent-\leftmargin
                        \let\makelabel\descriptionlabel}}
               {\endlist}
%    \end{macrocode}
% \end{environment}
%
% \begin{macro}{\descriptionlabel}
%    要更改标签(label)的格式,必须重新定义 |\descriptionlabel|。
%
% \changes{v1.2k}{1994/05/06}{Inserted \cs{normalfont}}
% \changes{v1.2y}{1995/01/31}{made command short}
%    \begin{macrocode}
\newcommand*\descriptionlabel[1]{\hspace\labelsep
                                \normalfont\bfseries #1}
%    \end{macrocode}
% \end{macro}
%
% \subsection[定义新环境]{\heiti 定义新环境}
%
% \subsubsection[摘要]{\heiti 摘要}
%
% \begin{environment}{abstract}
%    当我们制作单独的标题页(separate titlepage)时,我们也会将摘要(abstract)放在它
%    自己的页面上。摘要在页面上垂直居中放置。
%
%    请注意,此环境不是为 books 定义的。
% \changes{v1.3e}{1995/06/19}{Added setting of \cs{@endparpenalty}
%         to avoid page break after abstract heading.}
%    \begin{macrocode}
% \changes{v1.3m}{1995/10/23}{Added setting of \cs{beginparpenalty} to
%    discourage page break before abstract heading.}
%<*article|report>
\if@titlepage
  \newenvironment{abstract}{%
      \titlepage
      \null\vfil
      \@beginparpenalty\@lowpenalty
      \begin{center}%
        \bfseries \abstractname
        \@endparpenalty\@M
      \end{center}}%
     {\par\vfil\null\endtitlepage}
%    \end{macrocode}
%    当我们没有创建单独的标题页--article 文档类的默认标题页,我们必须检查是否处于
%    两栏模式(two-column mode)。在这种情况下,摘要将作为一个|\section*|,否则将使
%    用引用环境(quotation environment)对摘要进行排版。
%    \begin{macrocode}
\else
  \newenvironment{abstract}{%
      \if@twocolumn
        \section*{\abstractname}%
      \else
        \small
        \begin{center}%
          {\bfseries \abstractname\vspace{-.5em}\vspace{\z@}}%
        \end{center}%
        \quotation
      \fi}
      {\if@twocolumn\else\endquotation\fi}
\fi
%</article|report>
%    \end{macrocode}
% \end{environment}
%
% \subsubsection[诗歌]{\heiti 诗歌}
%
% \begin{environment}{verse}
%   诗歌环境(verse environment)是通过巧妙地使用列表环境(list environment)的参数来定义的。
%   用户键入 |\\| 结束一行。这是通过 |\let| |\\| 等于 |\@centercr| 来实现的。
%
% \changes{v1.3j}{1995/08/16}{stop \cs{item} scanning for [ with
%    \cs{relax}}
%    \begin{macrocode}
\newenvironment{verse}
               {\let\\\@centercr
                \list{}{\itemsep      \z@
                        \itemindent   -1.5em%
                        \listparindent\itemindent
                        \rightmargin  \leftmargin
                        \advance\leftmargin 1.5em}%
                \item\relax}
               {\endlist}
%    \end{macrocode}
% \end{environment}
%
% \subsubsection[引用(quotation)]{\heiti 引用(quotation)}
%
% \begin{environment}{quotation}
%   引用环境环境(quotation environment)也可以通过巧妙地使用列表环境(list environment)的
%   参数来定义。环境中的线条设置小于 |\textwidth|。此环境中段落的第一行是缩进的。
%
% \changes{v1.3j}{1995/08/16}{stop \cs{item} scanning for [ with
%    \cs{relax}}
%    \begin{macrocode}
\newenvironment{quotation}
               {\list{}{\listparindent 1.5em%
                        \itemindent    \listparindent
                        \rightmargin   \leftmargin
                        \parsep        \z@ \@plus\p@}%
                \item\relax}
               {\endlist}
%    \end{macrocode}
% \end{environment}
%
% \subsubsection[引用(quote)]{引用(quote)}
%
% \begin{environment}{quote}
%   引用(quote)环境与引用(quotation)环境类似,只是段落没有缩进。
%
% \changes{v1.3j}{1995/08/16}{stop \cs{item} scanning for [ with
%    \cs{relax}}
%    \begin{macrocode}
\newenvironment{quote}
               {\list{}{\rightmargin\leftmargin}%
                \item\relax}
               {\endlist}
%    \end{macrocode}
% \end{environment}
%
% \subsubsection[定理]{\heiti 定理}
%
%    这个文档类(document class)没有定义自己的定理环境(theorem environments),可以
%    使用 \LaTeX{}\ 格式提供的默认值。
%
% \subsubsection[标题页]{\heiti 标题页}
%
% \begin{environment}{titlepage}
%  在普通环境(normal environments)中,titlepage 环境只会开始和结束页面,并禁止页码(page numbers)。
%  当 \LaTeX\ 处于两栏模式(two-column mode)时,环境会暂时切换到单栏模式(one-column mode)。在 report 类中,
%  它还将页码重置为 1,然后在两栏模式中,在末尾将其重置为 1。对于 book 类,环境通过
%  发出 \cs{cleardouplepage}\ 命令来确保标题页位于右侧页(recto page)上。在兼容模式(compatibility mode)中,
%  它将页码设置为零。这是不正确的,因为它的结果是使用右侧页面(right-hand page)的页面参数(page parameters),但它是以前的方式。
%
% \changes{v1.0g}{1993/12/09}{Moved the setting of
%    \cs{@restonecolfalse}}
% \changes{v1.2c}{1994/03/17}{page :!= 0 only in compatibility mode
%    (LL)}
% \changes{v1.2d}{1994/04/11}{Moved \cs{cleardoublepage} inside
%    definition of titlepage environment}
% \changes{v1.3i}{1995/08/08}{New implementation with support for
%      twoside and openright option}
%
%    首先,我们给出了兼容模式(compatibility mode)的定义。
%    \begin{macrocode}
\if@compatibility
  \newenvironment{titlepage}
    {%
%<book>      \cleardoublepage
      \if@twocolumn
        \@restonecoltrue\onecolumn
      \else
        \@restonecolfalse\newpage
      \fi
      \thispagestyle{empty}%
      \setcounter{page}\z@
    }%
    {\if@restonecol\twocolumn \else \newpage \fi
    }
%    \end{macrocode}
%
%    这是原始的(native) \LaTeXe{}。
%    \begin{macrocode}
\else
  \newenvironment{titlepage}
    {%
%<book>      \cleardoublepage
      \if@twocolumn
        \@restonecoltrue\onecolumn
      \else
        \@restonecolfalse\newpage
      \fi
      \thispagestyle{empty}%
      \setcounter{page}\@ne
    }%
    {\if@restonecol\twocolumn \else \newpage \fi
%    \end{macrocode}
%    如果我们不是在双面模式(two-side mode)中,标题页(title page)后的第一页也应该是第 1 页。
%    \begin{macrocode}
     \if@twoside\else
        \setcounter{page}\@ne
     \fi
    }
\fi
%    \end{macrocode}
% \end{environment}
%
% \subsubsection[附录]{\heiti 附录}
%
% \begin{macro}{\appendix}
%
%    |\appendix| 命令并不是真正的环境,它是一个宏,它对事情的处理方式做了一些改变。
%
%    在 article 文档类中,|\appendix| 命令必须执行以下操作:
%    \begin{itemize}
%    \item 将节(section)和子节(subsection)计数器重置为零,
%    \item 重新定义 |\thesection| 以产生按字母顺序排列的附录编号(alphabetic appendix numbers)。
%    这种重新定义是全局进行的,以确保即使在诸如 \texttt{multicols}\ 之类的环境中发出 |\appendix|,
%    它也能够继续存在。
%    \end{itemize}
%
% \changes{1.3z}{1998/09/19}{Redefine \cs{thesection} globally (pr/2862)}
%    \begin{macrocode}
%<*article>
\newcommand\appendix{\par
  \setcounter{section}{0}%
  \setcounter{subsection}{0}%
  \gdef\thesection{\@Alph\c@section}}
%</article>
%    \end{macrocode}
%
%    在 report 和 book 文档类中, |\appendix| 命令必须执行以下操作:
%    \begin{itemize}
%    \item 将章(chapter)和节(section)计数器重置为零
%    \item 将 |\@chapapp| 设置为|\appendixname|(用于消息)
%    \item 重新定义章计数器(chapter counter)以生成附录编号(appendix numbers)
%    \item 如果附录名称(appendix titles)和附录标题(appendix headings)看起来与章名
%    称(chapter titles)和章标题(chapter headings)不同,可能需要重新定义 |\chapter| 命令。
%    这种重新定义是全局进行的,以确保即使在诸如 \texttt{multicols}\ 之类的环境中发
%    出 |\appendix|,它也能继续存在。
%    \end{itemize}
%
% \changes{1.3z}{1998/09/19}{Redefine \cs{thechapter} and
%                            \cs{@chapapp} globally (pr/2862)}
%    \begin{macrocode}
%<*report|book>
\newcommand\appendix{\par
  \setcounter{chapter}{0}%
  \setcounter{section}{0}%
  \gdef\@chapapp{\appendixname}%
  \gdef\thechapter{\@Alph\c@chapter}}
%</report|book>
%    \end{macrocode}
% \end{macro}
%
% \subsection[为现有环境设置参数]{\heiti 为现有环境设置参数}
%
% \subsubsection[数组(array)和表格(tabular)]{\heiti 数组(array)和表格(tabular)}
%
% \begin{macro}{\arraycolsep}
%    数组环境(array environment)中的列(columns)由 2 个 |\arraycolsep| 分隔。
%    \begin{macrocode}
\setlength\arraycolsep{5\p@}
%    \end{macrocode}
% \end{macro}
%
% \begin{macro}{\tabcolsep}
%    表格环境(tabular environment)中的列(columns)由 2 个 |\tabcolsep| 分隔。
%    \begin{macrocode}
\setlength\tabcolsep{6\p@}
%    \end{macrocode}
% \end{macro}
%
% \begin{macro}{\arrayrulewidth}
%    数组(array)和表格(tabular)环境中线段(rule)的宽度由 |\arrayrulewidth| 给出。
%    \begin{macrocode}
\setlength\arrayrulewidth{.4\p@}
%    \end{macrocode}
% \end{macro}
%
% \begin{macro}{\doublerulesep}
%    数组(array)和表格(tabular)环境中相邻线段(rule)之间的间距由 |\doublerulesep| 给出。
%    \begin{macrocode}
\setlength\doublerulesep{2\p@}
%    \end{macrocode}
% \end{macro}
%
% \subsubsection{表格(tabbing)}
%
% \begin{macro}{\tabbingsep}
%    这将控制 |\'| 命令放入的空间(space)。(有关说明,请参阅 \LaTeX{}\ 手册。)
%    \begin{macrocode}
\setlength\tabbingsep{\labelsep}
%    \end{macrocode}
% \end{macro}
%
% \subsubsection[小页]{\heiti 小页}
%
% \begin{macro}{\@minipagerestore}
%    当 |\@minipagerestore| 宏进入到一个小页环境(minipage environmen)时,它被调用
%    来设置在小页环境中需要以不同方式处理的事情。在当前的类中,它什么也不做。
% \end{macro}
%
% \begin{macro}{\@mpfootins}
%    小页(minipages)有自己的脚注(footnotes);|\skip||\@mpfootins| 在迷你页面中对
%    脚注的作用与对普通脚注(ordinary footnotes)的相同。
%
%    \begin{macrocode}
\skip\@mpfootins = \skip\footins
%    \end{macrocode}
% \end{macro}
%
% \subsubsection[有框的盒子]{\heiti 有框的盒子}
%
% \begin{macro}{\fboxsep}
%    |\fbox| 和 |\framebox| 在框和其中的文本之间留下空白(space)。
% \begin{macro}{\fboxrule}
%    由 |\fbox| 和 |\framebox| 创建的盒子(box)中线段(rules)的宽度。
%    \begin{macrocode}
\setlength\fboxsep{3\p@}
\setlength\fboxrule{.4\p@}
%    \end{macrocode}
% \end{macro}
% \end{macro}
%
% \subsubsection[排序单行公式和排序公式组]{排序单行公式(equation)和排序公式组(eqnarray)}
%
% \begin{macro}{\theequation}
%    在章内时,排序单行公式(equation)计数器将在新章开始时重置,排序单行公式(equation)编号将以章编号为前缀。
% \changes{v1.3u}{1996/10/31}{Added test for non-zero chapter number}
%
%    此代码必须遵循 |\chapter| 的定义,或者更确切地说,遵循章计数器(chapter counter)的定义。
%    \begin{macrocode}
%<article>\renewcommand \theequation {\@arabic\c@equation}
%<*report|book>
\@addtoreset {equation}{chapter}
\renewcommand\theequation
  {\ifnum \c@chapter>\z@ \thechapter.\fi \@arabic\c@equation}
%</report|book>
%    \end{macrocode}
% \end{macro}
%
% \begin{macro}{\jot}
%    |\jot| 是排序公式组(eqnarray)环境行之间添加的额外空间(extra space)。使用默认值。
%    \begin{macrocode}
% \setlength\jot{3pt}
%    \end{macrocode}
% \end{macro}
%
% \begin{macro}{\@eqnnum}
%    |\@eqnnum| 宏定义排序单行公式(equation)编号在公式中的显示方式。再次使用默认值。
%
%    \begin{macrocode}
% \def\@eqnnum{(\theequation)}
%    \end{macrocode}
% \end{macro}
%
% \subsection[浮动对象]{\heiti 浮动对象}
%
%    \LaTeX{}\ 格式仅定义了一些可用于定义浮动对象(floating objects)的工具。这是在
%    文档类(document class)中完成的。它需要为 \texttt{TYPE}\ 类型(例如\texttt{TYPE}=figure)的
%    每个浮动对象定义以下宏。
%
%    \begin{description}
%    \item[\texttt{\bslash fps@TYPE}]
%        \texttt{TYPE}\ 类型的浮动的默认位置说明符(default placement specifier)。
%
%    \item[\texttt{\bslash ftype@TYPE}]
%        \texttt{TYPE}\ 类型的浮动的类型编号(type number)。每个 \texttt{TYPE}\ 都关
%        联了一个唯一的正 \texttt{TYPE}\ 编号,它是 2 的幂。例如,图形(figures)可能有
%        类型编号 1,表格(tables)可能有类型编号 2,程序(programs)可能有类型编号 4,等等。
%
%    \item[\texttt{\bslash ext@TYPE}]
%        指示存储浮动类型 \texttt{TYPE}\ 内容列表(contents list)的文件的扩展名。例如|\ext@figure|=`lof'。
%
%    \item[\texttt{\bslash fnum@TYPE}]
%        为标题(caption)生成图形编号(figure number)的宏。例如 |\fnum@TYPE| == `Figure |\thefigure|'。
%
%    \item[\texttt{\bslash @makecaption}{\meta{num}}{\meta{text}}]
%        用于制作标题(caption)的宏,\meta{num}\ 为标题文本 |\fnum@...| and \meta{text}\ 生成的值。
%        它可以假定它位于适当宽度的 |\parbox| 中。这将用于{\kaiti 所有}浮动对象。
%
%    \end{description}
%
%    实现浮动对象(如图形)的实际环境是使用 |\@float| 和 |\end@float| 宏定义的,
%    而这些宏又是在 \LaTeX{}\ 格式中定义好的。
%
%    实现单栏浮动对象(single-column floating object)的环境以 \texttt{TYPE}\ 类型
%    的 |\@float{|\texttt{TYPE}|}[|\meta{placement}|]| 开始,\texttt{TYPE}\ 类型带
%    有 \meta{placement}\ 作为位置说明符(placement specifier)。\meta{PLACEMENT}\ 的
%    默认值由 |\fps@TYPE| 定义。
%
%    环境结束于 |\end@float|,例如,|\figure| == |\@float|{figure}, |\endfigure| == |\end@float|。
%
% \subsubsection[图形]{\heiti 图形}
%
%    下面是图形环境(figure environment)的实现。
%
% \begin{macro}{\c@figure}
%    首先,我们必须分配一个计数器(counter)来对图形(figures)进行编号。
%
%    在 report 和 book 文档类中,章(chapters)内的图形按章编号。
% \changes{v1.3u}{1996/10/31}{Added test for non-zero chapter number}
%    \begin{macrocode}
%<*article>
\newcounter{figure}
\renewcommand \thefigure {\@arabic\c@figure}
%</article>
%<*report|book>
\newcounter{figure}[chapter]
\renewcommand \thefigure
     {\ifnum \c@chapter>\z@ \thechapter.\fi \@arabic\c@figure}
%</report|book>
%    \end{macrocode}
% \end{macro}
%
% \begin{macro}{\fps@figure}
% \begin{macro}{\ftype@figure}
% \begin{macro}{\ext@figure}
% \begin{macro}{\num@figure}
%    以下是“figure(图)”类型的浮动对象的参数。
% \changes{v1.4e}{2001/05/24}{Replaced tilde with \cs{nobreakspace}
%                             (pr/3310)}
%    \begin{macrocode}
\def\fps@figure{tbp}
\def\ftype@figure{1}
\def\ext@figure{lof}
\def\fnum@figure{\figurename\nobreakspace\thefigure}
%    \end{macrocode}
% \end{macro}
% \end{macro}
% \end{macro}
% \end{macro}
%
% \begin{environment}{figure}
% \begin{environment}{figure*}
%    以及实际环境(actual environment)的定义。带 |*| 的表单(form)用于双栏图形(double-column figures)。
%    \begin{macrocode}
\newenvironment{figure}
               {\@float{figure}}
               {\end@float}
\newenvironment{figure*}
               {\@dblfloat{figure}}
               {\end@dblfloat}
%    \end{macrocode}
% \end{environment}
% \end{environment}
%
% \subsubsection[表格]{\heiti 表格}
%
%    下面是表环境(table environment)的实现。它与图形环境(figure environment)非常相似。
%
% \begin{macro}{\c@table}
%    首先,我们必须分配一个计数器来对表进行编号。
%
%    在 report 和 book 文档类中,章内的表格按章编号。
% \changes{v1.3u}{1996/10/31}{Added test for non-zero chapter number}
%    \begin{macrocode}
%<*article>
\newcounter{table}
\renewcommand\thetable{\@arabic\c@table}
%</article>
%<*report|book>
\newcounter{table}[chapter]
\renewcommand \thetable
     {\ifnum \c@chapter>\z@ \thechapter.\fi \@arabic\c@table}
%</report|book>
%    \end{macrocode}
% \end{macro}
%
% \begin{macro}{\fps@table}
% \begin{macro}{\ftype@table}
% \begin{macro}{\ext@table}
% \begin{macro}{\num@table}
%    以下是“table(表格)”类型的浮动对象的参数。
% \changes{v1.4e}{2001/05/24}{Replaced tilde with \cs{nobreakspace}
%                             (pr/3310)}
%    \begin{macrocode}
\def\fps@table{tbp}
\def\ftype@table{2}
\def\ext@table{lot}
\def\fnum@table{\tablename\nobreakspace\thetable}
%    \end{macrocode}
% \end{macro}
% \end{macro}
% \end{macro}
% \end{macro}
%
% \begin{environment}{table}
% \begin{environment}{table*}
%    以及实际环境的定义。带 |*| 的表格用于双栏表格(double-column tables)。
%    \begin{macrocode}
\newenvironment{table}
               {\@float{table}}
               {\end@float}
\newenvironment{table*}
               {\@dblfloat{table}}
               {\end@dblfloat}
%    \end{macrocode}
% \end{environment}
% \end{environment}
%
% \subsubsection[标题]{\heiti 标题}
%
% \begin{macro}{\@makecaption}
%    |\caption| 命令调用 |\@makecaption| 来格式化浮动对象的标题(caption)。它有两个参数,
%    第一个 \meta{number},为浮动对象的编号(number),第二个 \meta{text},为标题的文本(text of the caption)。
%    通常 \meta{number}\ 包含一个字符串(string),如“图 3.2”。宏可以假定它是在一个宽度
%    为 |\normalsize| 的 |\parbox| 中调用的。
%
% \begin{macro}{\abovecaptionskip}
% \begin{macro}{\belowcaptionskip}
%    这些长度包含标题上方和下方要保留的空白(space)。
%    \begin{macrocode}
\newlength\abovecaptionskip
\newlength\belowcaptionskip
\setlength\abovecaptionskip{10\p@}
\setlength\belowcaptionskip{0\p@}
%    \end{macrocode}
% \end{macro}
% \end{macro}
%
%    此宏的定义是 |\long|,能在标题(caption)中允许多个段落(paragraph)。
%    \begin{macrocode}
\long\def\@makecaption#1#2{%
  \vskip\abovecaptionskip
%    \end{macrocode}
%    我们想看看标题(caption)是否适合页面上的一行,因此我们首先在一个临时盒子(temporary box)中排版。
% \changes{v1.2q}{1994/05/29}{Use \cs{sbox}\cs{@tempboxa} instead of
%    \cs{setbox}\cs{@tempboxa}\cs{hbox} to make this colour safe}
%    \begin{macrocode}
  \sbox\@tempboxa{#1: #2}%
%    \end{macrocode}
%    我们可以测量它的宽度。它比当前的 |\hsize| 大,我们将标题排版为普通段落(ordinary paragraph)。
%    \begin{macrocode}
  \ifdim \wd\@tempboxa >\hsize
    #1: #2\par
%    \end{macrocode}
%    如果标题合适,我们将其居中。因为这在垂直模式(vertical mode)中直接使用了一个 |\hbox|,
%    所以它不会执行 |\everypar| 令牌(tokens);这里唯一需要做的就是重置“小页标志(minipage flag)”,
%    所以我们显式地这么做。
% \changes{v1.2x}{1994/12/09}{Due to a change in the way floats are
%    handled we need to set the \cs{if@minipage} switch to false}
%    \begin{macrocode}
  \else
    \global \@minipagefalse
    \hb@xt@\hsize{\hfil\box\@tempboxa\hfil}%
  \fi
  \vskip\belowcaptionskip}
%    \end{macrocode}
% \end{macro}
%
% \subsection[字体更改]{\heiti 字体更改}
%
%    这里我们提供了\LaTeX\ 2.09 及更早版本中常见的声明性字体更改命令(declarative font changing commands)。
%    这些命令在文本模式(text mode){\kaiti 和}数学模式(math mode)中工作。它们是为了兼容性而提供的,
%    但使用时应以 |\text...| 和 |\math...| 命令开始。这些命令使用 |\DeclareTextFontCommand| 定义,
%    该命令有三个参数:要定义的用户命令(user command)、在文本模式中要执行 \LaTeX\ 命令、
%    在数学模式中要执行 \LaTeX\ 命令。
%
% \changes{v1.0g}{1993/12/12}{Distinguished between compatibility and
%    `normal' mode for the font changing commands.}
% \changes{v1.0h}{1993/12/18}{These are now defined in the kernel, so
%    use \cs{@renewfontswitch}.  Compatibility mode defines
%    \cs{@renewfontswitch} to do nothing, so we don't need to check
%    for compatibility mode any more.}
% \changes{v1.0j}{1993/12/20}{Added \cs{normalfont} back in the
%    definitions of \cs{rm} etc. as this should be the default
%    behaviour}
% \changes{v1.2e}{1994/04/14}{\cs{@renewfontswitch} has become
%    \cs{DeclareOldFontCommand}}
%
%  \begin{macro}{\rm}
% \changes{v1.0f}{1993/12/08}{Macro added}
%  \begin{macro}{\tt}
% \changes{v1.0f}{1993/12/08}{Macro added}
%  \begin{macro}{\sf}
% \changes{v1.0f}{1993/12/08}{Macro added}
%
%    修改族(family)的命令。在兼容模式下,我们首先选择“default”字体,以获得 \LaTeX2.09\ 行为(behaviour)。
%    \begin{macrocode}
\DeclareOldFontCommand{\rm}{\normalfont\rmfamily}{\mathrm}
\DeclareOldFontCommand{\sf}{\normalfont\sffamily}{\mathsf}
\DeclareOldFontCommand{\tt}{\normalfont\ttfamily}{\mathtt}
%    \end{macrocode}
%  \end{macro}
%  \end{macro}
%  \end{macro}
%
%  \begin{macro}{\bf}
% \changes{v1.0f}{1993/12/08}{Macro added}
%    更改为粗体序列(bold series)的命令。应该使用 |\mdseries| 显式切换回中等序列(medium series)。
%    \begin{macrocode}
\DeclareOldFontCommand{\bf}{\normalfont\bfseries}{\mathbf}
%    \end{macrocode}
%  \end{macro}
%
%  \begin{macro}{\sl}
% \changes{v1.0f}{1993/12/08}{Macro added}
% \changes{v1.2g}{1994/04/24}{Added warning if used in math mode}
%  \begin{macro}{\it}
% \changes{v1.0f}{1993/12/08}{Macro added}
%  \begin{macro}{\sc}
% \changes{v1.0f}{1993/12/08}{Macro added}
% \changes{v1.2g}{1994/04/24}{Added warning if used in math mode}
%
%    以及更改字体形状(shape)的命令。默认情况下,倾斜(slanted)和小型大写(small caps)形状
%    不可用作数学字母(math alphabets),因此这些更改在数学模式(math mode)中不起作用。
%    但是,我们确实警告用户,这样的选择不会产生任何影响。应该使用 |\upshape| 明确地
%    更改回直立形状(upright shape)。
%    \begin{macrocode}
\DeclareOldFontCommand{\it}{\normalfont\itshape}{\mathit}
\DeclareOldFontCommand{\sl}{\normalfont\slshape}{\@nomath\sl}
\DeclareOldFontCommand{\sc}{\normalfont\scshape}{\@nomath\sc}
%    \end{macrocode}
%  \end{macro}
%  \end{macro}
%  \end{macro}
%
% \begin{macro}{\cal}
% \changes{v1.0g}{1993/12/12}{Macro added}
% \begin{macro}{\mit}
% \changes{v1.0g}{1993/12/12}{Macro added}
%
%    命令 |\cal| 和 |\mit| 只能在数学模式中使用,在数学模式之外,它们没有任何作用。
%    目前,新字体选择方案(New Font Selection Scheme,NFSS)定义了这些命令来生成警告消息。
%    因此,我们必须“手工(by hand)”定义它们。
% \changes{v1.2w}{1994/12/01}{Now define \cs{cal} and \cs{mit} using
%    \cs{DeclareRobustCommand*}}
% \changes{v1.3j}{1995/08/16}{Remove surplus braces}
%    \begin{macrocode}
\DeclareRobustCommand*\cal{\@fontswitch\relax\mathcal}
\DeclareRobustCommand*\mit{\@fontswitch\relax\mathnormal}
%    \end{macrocode}
%  \end{macro}
%  \end{macro}
%
%\newpage
%
% \section{\heiti 交叉引用}
% \subsection[目录,等]{\heiti 目录,等}
%
%     |\section| 命令将 |\contentsline{section}{|\meta{title}|}{|\meta{page}|}| 命令
%     写入 \file{.toc}\ 文件中,在\file{.toc}\ 文件中,\meta{title}\ 包含条目(entry)的内容,
%     而 \meta{page}\ 是页码(page number)。如果正在对节进行编号,则 \meta{title}\  格式
%     为 |\numberline{|\meta{num}|}{|\meta{heading}|}|,这里的 \meta{num}\ 是
%     由 |\thesection| 生成的编号(number)。其他分节命令(sectioning commands)的工作方式类似。
%
%     “figure(图形)”环境中的 |\caption| 命令写入
%
%     |\contentsline{figure}{\numberline{|\meta{num}|}{|^^A
%                    \meta{caption}|}}{|\meta{page}|}|
%
%     在 .\file{lof}\ 文件中,其中的 \meta{num}\ 是 |\thefigure| 生成的编号(number),
%     而 \meta{caption}\ 是图形标题(figure caption)。它在“table(表格)”环境中的工作方式类似。
%
%    命令 |\contentsline{|\meta{name}|}| 展开为 |\l@|\meta{name}。因此,要指定目录(table of contents),
%    我们必须定义 |\l@chapter|、|\l@section|、|\l@subsection|、... ;要指定图表目录(list of figures),
%    我们必须定义 |\l@figure|;等等。其中大多数可以用 |\@dottedtocline| 命令来定义,其工作原理如下:
%
%    |\@dottedtocline{|\meta{level}|}{|\meta{indent}|}{|^^A
%                      \meta{numwidth}|}{|^^A
%                      \meta{title}|}{|\meta{page}|}|
%
%    \begin{description}
%    \item[\meta{level}] 只有当 \meta{level} $\leq$ \Lcount{tocdepth}\ 计数器的值时才
%    生成条目。注意,|\chapter| 是 0 级,|\section| 是 1 级,等等。
%    \item[\meta{indent}] 从内容行开始处的左外边距(outer left margin)的缩进。
%    \item[\meta{numwidth}] 节编号(section number)所在盒子的宽度,如果 \meta{title}\ 包括一个 |\numberline| 命令。
%    \end{description}
%
% \begin{macro}{\@pnumwidth}
% \begin{macro}{\@tocrmarg}
% \begin{macro}{\@dotsep}
%    此命令使用以下三个参数,这三个参数是用 |\newcommand| 命令设置的(因此可以使用 em 使它们依赖于字体)。
%    \begin{description}
%    \item[\texttt{\bslash @pnumwidth}] 放置页码的盒子的宽度。
% \changes{v1.2v}{1994/10/29}{Changed documentation from $!>$ or $!=$ to
%    $\ge$}
%    \item[\texttt{\bslash @tocrmarg}] 多行条目(multiple line entries)的右边距(right margin)。一个条目需要 |\@tocrmarg| $\ge$ |\@pnumwidth|。
%    \item[\texttt{\bslash @dotsep}] 点(dots)与点之间的距离,单位是 μ。应该定义为像 2 或 1.7 这样的数字
%    \end{description}
%
%    \begin{macrocode}
\newcommand\@pnumwidth{1.55em}
\newcommand\@tocrmarg{2.55em}
\newcommand\@dotsep{4.5}
%<article>\setcounter{tocdepth}{3}
%<!article>\setcounter{tocdepth}{2}
%    \end{macrocode}
% \end{macro}
% \end{macro}
% \end{macro}
%
% \subsubsection[目录]{\heiti 目录}
%
% \begin{macro}{\tableofcontents}
%    此宏用于请求 \LaTeX{}\ 生成目录。在 report 和 book 文档类中,目录、图形等始终
%    以单栏模式(single-column mode)设置。
%
% \changes{v1.0g}{1993/12/09}{Moved the setting of
%    \cs{@restonecolfalse}}
% \changes{v1.4h}{2007/10/19}{Explain why \cs{@mkboth} is inside the heading
%                         arg for \cs{tableofcontents} (pr/3285 and pr/3984)}
%    \begin{macrocode}
\newcommand\tableofcontents{%
%<*report|book>
    \if@twocolumn
      \@restonecoltrue\onecolumn
    \else
      \@restonecolfalse
    \fi
%    \end{macrocode}
%    使用 |\chapter*| 命令设置标题(title),确保栏外标题(running head)--如果需要--包含正确的信息。
%    \begin{macrocode}
    \chapter*{\contentsname
%</report|book>
%<article>    \section*{\contentsname
%    \end{macrocode}
%    |\@mkboth| 的代码放在标题(heading)内,以避免标题后的垂直间距(vertical spacing)
%    受到任何影响(在某些情况下)。对于其他命令,例如下面的 |\listoffigures|,
%    它与 \LaTeX{}2.09\ 版本相比有所改变,因为如果在双栏模式(two-column mode)中使用,
%    它将产生严重的错误(请参见 pr/3285)。然而,在这些类中,|\tableofcontents| 总是
%    以单栏模式(one-column mode)排版,因此出于兼容性的原因,保留了一些不一致的设置
%    (inconsistent setting)。
%    \begin{macrocode}
        \@mkboth{%
           \MakeUppercase\contentsname}{\MakeUppercase\contentsname}}%
%    \end{macrocode}
%    实际的目录是通过调用 |\@starttoc{toc}| 生成的。之后,如有必要,我们将恢复双栏
%    模式(two-column mode)。
%    \begin{macrocode}
    \@starttoc{toc}%
%<!article>    \if@restonecol\twocolumn\fi
    }
%    \end{macrocode}
% \end{macro}
%
% \begin{macro}{\l@part}
%    如上所述,每个分节命令(sectioning command)都需要一个附加宏(additional macro)
%    来格式化其在目录中的条目(entry)。部分条目(entry for parts)的宏是以特殊方式定义的。
%
%    首先,我们确保如果应该出现分页符(pagebreak),它会出现在此条目{\kaiti 之前}。此外,
%    还添加了一些空白(whitespace),并且一个组(group)开始保持局部的更改(keep changes local)。
% \changes{v1.0h}{1993/12/18}{Replaced -\cs{@secpenalty} by
%    \cs{@secpenalty}.  ASAJ.}
% \changes{v1.2i}{1994/04/28}{Don't print a toc line when the tocdepth
%    counter is less then -1}
% \changes{v1.3b}{1995/05/23}{Added missing braces around argument
%           to \cs{addpenalty}.}
% \changes{v1.3x}{1997/10/10}{Removed setting of \cs{@tempdima} as
%    this macro does not use \cs{numberline} to set the toc line.}
% \changes{v1.4a}{1998/10/12}{we should use \cs{@tocrmarg}; see PR/2881.}
%    \begin{macrocode}
\newcommand*\l@part[2]{%
  \ifnum \c@tocdepth >-2\relax
%<article>    \addpenalty\@secpenalty
%<!article>    \addpenalty{-\@highpenalty}%
    \addvspace{2.25em \@plus\p@}%
%    \end{macrocode}
%    宏 |\numberline| 要求保存部分编号(part number)的盒子的宽度存储在 \LaTeX\ 的暂
%    存寄存器(scratch register) |\@tempdima| 中。因此,即使我们没有在内部使
%    用 |\numberline|,我们也会在那里初始化它,因为使用的值非常大,所以
%    像 |\numberline{VIII}| 这样的值仍然可以工作
% \changes{v1.4d}{2001/04/21}{Initialize \cs{@tempdima} to some
%    sensible value (pr/3327)}
%    \begin{macrocode}
    \setlength\@tempdima{3em}%
    \begingroup
%    \end{macrocode}
%    我们将 |\parindent| 设置为 0pt,并使用 |\rightskip| 为页码(page numbers)留出
%    足够的空间(room)。\footnote{我们应该将 \cs{rightskip}\ 设置为 \cs{@tocrmarg}\ 而不是 \cs{@pnumwidth}
%    ({\LaTeX}\ 的任何版本都没有这样做),否则 \cs{rightskip}\ 会太小。不幸的是,这在\ {\LaTeXe} 中无法改变,
%    因为我们不想创建不同版本的\ {\LaTeXe},除非这是绝对必要的,否则会产生不同的排版输出(typeset output);
%    相反,我们为了\ \LaTeX3\ 而暂停它。}
%    为防止溢框信息(overfull box messages),|\parfillskip| 设置为负值。
%    \begin{macrocode}
      \parindent \z@ \rightskip \@pnumwidth
      \parfillskip -\@pnumwidth
%    \end{macrocode}
%    现在我们可以用大粗体字体(large bold font)设置条目。我们确保离开垂直模式(vertical mode),
%    设置部分标题(part title)并添加页码(page number),设置右侧对齐(flush right)。
%    \begin{macrocode}
      {\leavevmode
       \large \bfseries #1\hfil
       \hb@xt@\@pnumwidth{\hss #2%
%    \end{macrocode}
%
% \changes{v1.4i}{2018/09/03}{Prevent protrusion
%     (\url{https://tex.stackexchange.com/q/172785/10109})}
%    \begin{macrocode}
                          \kern-\p@\kern\p@}}\par
%    \end{macrocode}
%    在这个条目之后立即防止分页符(pagebreak),但是使用 |\everypar| 重置 |\if@nobreak| 开关。
%    最后,我们关闭组(close the group)。
% \changes{v1.3j}{1995/08/16}{Add missing percent}
%    \begin{macrocode}
       \nobreak
%<article>       \if@compatibility
         \global\@nobreaktrue
         \everypar{\global\@nobreakfalse\everypar{}}%
%<article>      \fi
    \endgroup
  \fi}
%    \end{macrocode}
% \end{macro}
%
% \begin{macro}{\l@chapter}
%    此宏设置章目录中的条目的格式。它与 |\l@part| 非常相似。
%
%    首先,我们确保如果出现分页符(pagebreak),它会出现在该条目{\kaiti 之前}。此外,
%    还添加了一点空白(whitespace),一个组(group)开始保持局部的更改(keep changes local)。
% \changes{v1.2i}{1994/04/28}{Don't print a toc line when the tocdepth
%    counter is less than 0}
% \changes{v1.3b}{1995/05/23}{Added missing braces around argument
%           to \cs{addpenalty}.}
% \changes{v1.4a}{1998/10/12}{we should use \cs{@tocrmarg}; see PR/2881.}
%    \begin{macrocode}
%<*report|book>
\newcommand*\l@chapter[2]{%
  \ifnum \c@tocdepth >\m@ne
    \addpenalty{-\@highpenalty}%
    \vskip 1.0em \@plus\p@
%    \end{macrocode}
%
%    宏 |\numberline| 要求保存部分编号(part number)的盒子的宽度存储在 \LaTeX\ 的暂
%    存寄存器(scratch register) |\@tempdima| 中。因此,即使我们没有在内部使用 |\numberline|,
%    我们也会在那里对其进行初始化(位置和值似乎有问题,但在不产生兼容性问题的情况下无法更改)。
%    我们开始一个组(group),并更改一些段落参数(paragraph parameters)
%    (另请参阅 \cs{l@part}\ 关于\ \cs{rightskip})。
%    \begin{macrocode}
    \setlength\@tempdima{1.5em}%
    \begingroup
      \parindent \z@ \rightskip \@pnumwidth
      \parfillskip -\@pnumwidth
%    \end{macrocode}
%    然后我们离开垂直模式(vertical mode)并切换到粗体字体(bold font)。
%    \begin{macrocode}
      \leavevmode \bfseries
%    \end{macrocode}
%    因为这里不使用 |\numberline|,所以在设置条目之前,我们需要“手工”进行一些微调。
%    我们不鼓励但不禁止在章条目(chapter entry)后立即使用分页符(pagebreak)。
%    \begin{macrocode}
      \advance\leftskip\@tempdima
      \hskip -\leftskip
      #1\nobreak\hfil
      \nobreak\hb@xt@\@pnumwidth{\hss #2%
%    \end{macrocode}
%
% \changes{v1.4i}{2018/09/03}{Prevent protrusion
%     (\url{https://tex.stackexchange.com/q/172785/10109})}
%    \begin{macrocode}
                                 \kern-\p@\kern\p@}\par
      \penalty\@highpenalty
    \endgroup
  \fi}
%</report|book>
%    \end{macrocode}
% \end{macro}
%
% \begin{macro}{\l@section}
%    在 article 文档类中,节(sections)目录中的条目看起来很像 report 和 book 文档类
%    的章条目(chapter entries)。
%
%    首先,我们确保如果出现分页符(pagebreak),它会出现在该条目{\kaiti 之前}。此外,
%    还添加了一点空白(whitespace),一个组(group)开始保持局部的更改(keep changes local)。
% \changes{v1.0h}{1993/12/18}{Replaced -\cs{@secpenalty} by
%    \cs{@secpenalty}.  ASAJ.}
% \changes{v1.2i}{1994/04/28}{Don't print a toc line when the tocdepth
%    counter is less than 1.}
% \changes{v1.4a}{1998/10/12}{we should use \cs{@tocrmarg}; see PR/2881.}
%    \begin{macrocode}
%<*article>
\newcommand*\l@section[2]{%
  \ifnum \c@tocdepth >\z@
    \addpenalty\@secpenalty
    \addvspace{1.0em \@plus\p@}%
%    \end{macrocode}
%
%    宏 |\numberline| 要求保存部分编号(part number)的盒子的宽度存储在 \LaTeX\ 的
%    暂存寄存器(scratch register) |\@tempdima| 中。因此,我们把它放在那里。我们
%    开始一个组(group),并更改一些段落参数(另请参阅 \cs{l@part}\ 关于\ \cs{rightskip})。
%    \begin{macrocode}
    \setlength\@tempdima{1.5em}%
    \begingroup
      \parindent \z@ \rightskip \@pnumwidth
      \parfillskip -\@pnumwidth
%    \end{macrocode}
%    然后我们离开垂直模式(vertical mode)并切换到粗体字体(bold font)。
%    \begin{macrocode}
      \leavevmode \bfseries
%    \end{macrocode}
%    因为这里不使用 |\numberline|,所以在设置条目之前,我们需要“手工”进行一些微调。
%    我们不鼓励但不禁止在章条目(chapter entry)后立即使用分页符(pagebreak)。
%    \begin{macrocode}
      \advance\leftskip\@tempdima
      \hskip -\leftskip
      #1\nobreak\hfil
      \nobreak\hb@xt@\@pnumwidth{\hss #2%
%    \end{macrocode}
%
% \changes{v1.4i}{2018/09/03}{Prevent protrusion
%     (\url{https://tex.stackexchange.com/q/172785/10109})}
%    \begin{macrocode}
                                 \kern-\p@\kern\p@}\par
    \endgroup
  \fi}
%</article>
%    \end{macrocode}
%    在 report 和 book 文档类中 |\l@section| 的定义要简单得多。
%    \begin{macrocode}
%<*report|book>
\newcommand*\l@section{\@dottedtocline{1}{1.5em}{2.3em}}
%</report|book>
%    \end{macrocode}
% \end{macro}
%
% \begin{macro}{\l@subsection}
% \begin{macro}{\l@subsubsection}
% \begin{macro}{\l@paragraph}
% \begin{macro}{\l@subparagraph}
%    所有较低层级的条目都是使用宏 |\@dottedtocline| 来定义的(请参见上文)。
%    \begin{macrocode}
%<*article>
\newcommand*\l@subsection{\@dottedtocline{2}{1.5em}{2.3em}}
\newcommand*\l@subsubsection{\@dottedtocline{3}{3.8em}{3.2em}}
\newcommand*\l@paragraph{\@dottedtocline{4}{7.0em}{4.1em}}
\newcommand*\l@subparagraph{\@dottedtocline{5}{10em}{5em}}
%</article>
%<*report|book>
\newcommand*\l@subsection{\@dottedtocline{2}{3.8em}{3.2em}}
\newcommand*\l@subsubsection{\@dottedtocline{3}{7.0em}{4.1em}}
\newcommand*\l@paragraph{\@dottedtocline{4}{10em}{5em}}
\newcommand*\l@subparagraph{\@dottedtocline{5}{12em}{6em}}
%</report|book>
%    \end{macrocode}
% \end{macro}
% \end{macro}
% \end{macro}
% \end{macro}
%
% \subsubsection[图形列表]{\heiti 图形列表}
%
% \begin{macro}{\listoffigures}
%    此宏用于请求 \LaTeX{}\ 生成图形列表(list of figures)。它与 |\tableofcontents| 非常相似。
%
% \changes{v1.0g}{1993/12/09}{Moved the setting of
%    \cs{@restonecolfalse}}
% \changes{v1.4c}{2001/01/06}{Moved \cs{@mkboth} out of heading
%                             arg (pr/3285)}
%    \begin{macrocode}
\newcommand\listoffigures{%
%<*report|book>
    \if@twocolumn
      \@restonecoltrue\onecolumn
    \else
      \@restonecolfalse
    \fi
    \chapter*{\listfigurename}%
%</report|book>
%<article>    \section*{\listfigurename}%
      \@mkboth{\MakeUppercase\listfigurename}%
              {\MakeUppercase\listfigurename}%
    \@starttoc{lof}%
%<report|book>    \if@restonecol\twocolumn\fi
    }
%    \end{macrocode}
% \end{macro}
%
% \begin{macro}{\l@figure}
%    此宏在图形列表(list of figures)中生成一个条目(entry)。
%    \begin{macrocode}
\newcommand*\l@figure{\@dottedtocline{1}{1.5em}{2.3em}}
%    \end{macrocode}
% \end{macro}
%
% \subsubsection[表格列表]{\heiti 表格列表}
%
% \begin{macro}{\listoftables}
%    此宏用于请求 \LaTeX{}\ 生成表格列表(list of tables)。它与 |\tableofcontents| 非常相似。
%
% \changes{v1.0g}{1993/12/09}{Moved the setting of
%    \cs{@restonecolfalse}}
% \changes{v1.4c}{2001/01/06}{Moved \cs{@mkboth} out of heading
%                             arg (pr/3285)}
%    \begin{macrocode}
\newcommand\listoftables{%
%<*report|book>
    \if@twocolumn
      \@restonecoltrue\onecolumn
    \else
      \@restonecolfalse
    \fi
    \chapter*{\listtablename}%
%</report|book>
%<article>    \section*{\listtablename}%
      \@mkboth{%
          \MakeUppercase\listtablename}%
         {\MakeUppercase\listtablename}%
    \@starttoc{lot}%
%<report|book>    \if@restonecol\twocolumn\fi
    }
%    \end{macrocode}
% \end{macro}
%
% \begin{macro}{\l@table}
%    此宏在表格列表(list of tables)中生成一个条目(entry)。
%    \begin{macrocode}
\let\l@table\l@figure
%    \end{macrocode}
% \end{macro}
%
% \subsection[参考文献]{\heiti 参考文献}
%
% \begin{macro}{\bibindent}
%    “开放(open)”参考文献格式(bibliography format)使用 |\bibindent| 缩进。
%    \begin{macrocode}
\newdimen\bibindent
\setlength\bibindent{1.5em}
%    \end{macrocode}
% \end{macro}
%
% \begin{environment}{thebibliography}
%    “theobjectory”环境执行以下命令:
%
%    |\renewcommand{\newblock}{\hskip.11em \@plus.33em \@minus.07em}|\\
%      --- 定义“封闭(closed)”格式,其中条目的块(blocks)(主要信息单元)一起运行。
%
%    |\sloppy|  --- 使用它是因为在参考文献中很难做断行(line breaks)处理,
%
%    |\sfcode`\.=1000\relax| --- 使“.”(句号)不产生句末空格。
%
%    此环境的实现(implementation)基于通用列表环境(generic list environment)。它在
%    内部使用 \Lcount{enumiv}\ 计数器生成列表的标签
%
%    当发现一个空的“thebibliography”环境时,会发出警告。
%
% \changes{v1.0i}{1993/12/19}{Corrected definition of thebibliography
%    for article}
% \changes{v1.2z}{1995/05/09}{added a missing percent character}
% \changes{v1.3b}{1995/05/23}{Added missing braces in definition
%    of thebibliography environment.}
% \changes{v1.3j}{1995/08/16}{remove surplus spaces}
% \changes{v1.3k}{1995/08/27}{Code for openbib changed}
% \changes{v1.3t}{1996/10/05}{Added setting value of \cs{@clubpenalty}}
%    \begin{macrocode}
\newenvironment{thebibliography}[1]
%<*article>
     {\section*{\refname}%
%    \end{macrocode}
%    由于至少在 report 和 book (双栏选项)中有 |\chapter| 的定义,否则会将其吞没,
%    所以 |\@mkboth| 被移出了标题参数(heading argument)。
% \changes{v1.4c}{2001/01/06}{Moved \cs{@mkboth} out of heading
%                             arg (pr/3285)}
%    \begin{macrocode}
      \@mkboth{\MakeUppercase\refname}{\MakeUppercase\refname}%
%</article>
%<*!article>
     {\chapter*{\bibname}%
      \@mkboth{\MakeUppercase\bibname}{\MakeUppercase\bibname}%
%</!article>
      \list{\@biblabel{\@arabic\c@enumiv}}%
           {\settowidth\labelwidth{\@biblabel{#1}}%
            \leftmargin\labelwidth
            \advance\leftmargin\labelsep
            \@openbib@code
            \usecounter{enumiv}%
            \let\p@enumiv\@empty
            \renewcommand\theenumiv{\@arabic\c@enumiv}}%
      \sloppy
%    \end{macrocode}
%    这是为整个环境设置 |\clubpenalty| 的正常(非无限)值,因此我们还必须重置其
%    已存储的值(stored value)。(为什么在下面第二个 4000 之后有 |%|?)
%    \begin{macrocode}
      \clubpenalty4000
      \@clubpenalty \clubpenalty
      \widowpenalty4000%
      \sfcode`\.\@m}
     {\def\@noitemerr
       {\@latex@warning{Empty `thebibliography' environment}}%
      \endlist}
%    \end{macrocode}
% \end{environment}
%
% \begin{macro}{\newblock}
%    |\newblock| 的默认定义是生成一个小空间(small space)。
% \changes{v1.3k}{1995/08/27}{Default changed.}
%    \begin{macrocode}
\newcommand\newblock{\hskip .11em\@plus.33em\@minus.07em}
%    \end{macrocode}
% \end{macro}
%
% \begin{macro}{\@openbib@code}
%    |\@openbib@code| 的默认定义就是什么都不做。它将由 \Lopt{openbib}\ 选项更改。
% \changes{v1.3k}{1995/08/27}{Macro added}
%    \begin{macrocode}
\let\@openbib@code\@empty
%    \end{macrocode}
% \end{macro}
%
% \begin{macro}{\@biblabel}
%    |\bibitem[...]| 命令的标签由该宏生成。使用 \LaTeX{}\ 格式的默认值。
%    \begin{macrocode}
% \renewcommand*{\@biblabel}[1]{[#1]\hfill}
%    \end{macrocode}
% \end{macro}
%
% \begin{macro}{\@cite}
%    |\cite| 命令的输出由该宏生成。使用 \LaTeX{}\ 格式的默认值。
%    \begin{macrocode}
% \renewcommand*{\@cite}[1]{[#1]}
%    \end{macrocode}
% \end{macro}
%
%  \subsection[索引]{\heiti 索引}
%
% \begin{environment}{theindex}
%    “theindex”环境可用于索引(index)。它创建的索引包含两栏,每个条目都是一个单独的段落。
%    在用户级别,命令|\item|、|\subitem| 和 |\subsubitem| 用于生成不同层级的索引条目(index entries)。
%    当遇到字母表(alphabet)中的新字母时,可以添加一定数量的 |\indexspace| 空白(white space)。
%
%
% \changes{v1.0g}{1993/12/09}{Moved the setting of
%    \cs{@restonecoltrue}}
%    \begin{macrocode}
\newenvironment{theindex}
               {\if@twocolumn
                  \@restonecolfalse
                \else
                  \@restonecoltrue
                \fi
%<article>                \twocolumn[\section*{\indexname}]%
%<!article>                \twocolumn[\@makeschapterhead{\indexname}]%
                \@mkboth{\MakeUppercase\indexname}%
                        {\MakeUppercase\indexname}%
                \thispagestyle{plain}\parindent\z@
%    \end{macrocode}
%    必须在 |\twocolumn| 操作之后对 |\columnseprule| 和 |\columnsep| 进行参数更改(parameter changes)。
%    否则,它们可能会影响索引前的最后一页。
% \changes{ v1.4f}{2004/02/16}{Moved setting of \cs{columnsep} and
%    \cs{columnseprule} later to avoid affecting the wrong page (pr/3616)}
%    \begin{macrocode}
                \parskip\z@ \@plus .3\p@\relax
                \columnseprule \z@
                \columnsep 35\p@
                \let\item\@idxitem}
%    \end{macrocode}
%    当文档在索引后继续,并且它是一个单栏文档(one-column document)时,我们必须
%    切换回索引之后的一栏。
%    \begin{macrocode}
               {\if@restonecol\onecolumn\else\clearpage\fi}
%    \end{macrocode}
% \end{environment}
%
% \begin{macro}{\@idxitem}
% \begin{macro}{\subitem}
% \begin{macro}{\subsubitem}
%    这些宏用于格式化索引中的条目。
% \changes{v1.3f}{1995/06/23}{Corrected error in definition of
%                         \cs{@idxitem}.}
% \changes{v1.3j}{1995/08/16}{use \cs{@idxitem} to save space}
%    \begin{macrocode}
\newcommand\@idxitem{\par\hangindent 40\p@}
\newcommand\subitem{\@idxitem \hspace*{20\p@}}
\newcommand\subsubitem{\@idxitem \hspace*{30\p@}}
%    \end{macrocode}
% \end{macro}
% \end{macro}
% \end{macro}
%
% \begin{macro}{\indexspace}
%    在索引中的“字母块(letter blocks)”之间插入的空白(white space)。
%    \begin{macrocode}
\newcommand\indexspace{\par \vskip 10\p@ \@plus5\p@ \@minus3\p@\relax}
%    \end{macrocode}
% \end{macro}
%
% \subsection[脚注]{\heiti 脚注}
%
% \begin{macro}{\footnoterule}
%    通常,脚注(footnotes)与正文(main body of the text)之间用一条小线段(rule)分开。
%    这条线段是由 |\footnoterule| 宏画出来的。我们必须确保该线段不占用垂直空
%    间(vertical space)(请参阅 \file{plain.tex}),因此我们通过添加适量的垂直间距
%    (vertical skip)来补偿 0.4pt 线段的自然高度(natural height)。
%
%    为了防止线段(rule)与脚注(footnote)冲突,我们首先添加了一点负的垂直间距(negative vertical skip),
%    然后我们放置线段,并确保我们在开始此操作的同一点结束。
% \changes{v1.3a}{1995/05/17}{use \cs{@width}}
%    \begin{macrocode}
\renewcommand\footnoterule{%
  \kern-3\p@
  \hrule\@width.4\columnwidth
  \kern2.6\p@}
%    \end{macrocode}
% \end{macro}
%
% \begin{macro}{\c@footnote}
%    脚注在报告(report)和书籍(book)文档类的章中编号。
%    \begin{macrocode}
%<!article>\@addtoreset{footnote}{chapter}
%    \end{macrocode}
% \end{macro}
%
% \begin{macro}{\@makefntext}
%    \LaTeX{}\ 的脚注机制(footnote mechanism)调用宏 |\@makefntext| 来生成实际的脚注。
%    宏获取脚注的文本作为其参数,并应使用 |\@thefnmark| 作为脚注的标记(mark)。
%    当 |\@makefntext| 宏有效地位于宽度为 |\columnwidth| (即使用 |\hsize| = |\columnwidth|)的 |\parbox| 内时,将调用该宏。
%
%   下面的 \TeX\ 代码给出了一个可以实现的示例。
% \begin{verbatim}
%          \newcommand\@makefntext[1]{%
%             \@setpar{\@@par
%                      \@tempdima = \hsize
%                      \advance\@tempdima-10pt
%                      \parshape \@ne 10pt \@tempdima}%
%             \par
%             \parindent 1em\noindent
%             \hbox to \z@{\hss\@makefnmark}#1}
% \end{verbatim}
%    这个定义的效果是脚注的所有行缩进 10pt,而新段落的第一行缩进 1em。要更改这些尺寸,
%    只需将“10pt”(在两个位置)或“1em”替换为所需的值(desired value)。这个记号(mark)刚好与脚注对齐。
%
%    在这些文档类中,我们使用了一个更简单的宏,其中脚注文本(footnote text)设置为普通文本段落(ordinary text paragraph),
%    除了段落的第一行和脚注的第一行外,没有缩进。因此,宏必须做的所有事情都是为后续段落设置 |\parindent| 为适当的值,
%    并在标记前放置适当的缩进(indentation)。
%
% \changes{v1.1a}{1994/03/13}{Use \cs{@makefnmark} to generate
%    footnote marker}
%    \begin{macrocode}
\newcommand\@makefntext[1]{%
    \parindent 1em%
    \noindent
    \hb@xt@1.8em{\hss\@makefnmark}#1}
%    \end{macrocode}
% \end{macro}
%
% \begin{macro}{\@makefnmark}
%    打印在文本中指向脚注的脚注标记(footnote markers)应由 |\@makefnmark| 宏生成。我们使用它的默认定义。
%    \begin{macrocode}
%\renewcommand\@makefnmark{\hbox{\@textsuperscript
%                                  {\normalfont\@thefnmark}}}
%    \end{macrocode}
% \end{macro}
%
%\newpage
%
% \section{\heiti 初始化}
%
% \subsection[单词]{\heiti 单词}
%
% 此文档类用于以英语编写的文档。为了准备另一种语言的版本,各种各样的英语单词(English words)必须被替换。
% 所有需要替换的英语单词在下面的命令名中定义。这些命令可以在用于非英语的 LATEX 定制的类或包中重新定义。
% \changes{v1.3h}{1995/07/20}{Split up to save on save stack /1742}
%
% \begin{macro}{\contentsname}
% \begin{macro}{\listfigurename}
% \begin{macro}{\listtablename}
%    \begin{macrocode}
\newcommand\contentsname{Contents}
\newcommand\listfigurename{List of Figures}
\newcommand\listtablename{List of Tables}
%    \end{macrocode}
% \end{macro}
% \end{macro}
% \end{macro}
%
% \begin{macro}{\refname}
% \begin{macro}{\bibname}
% \begin{macro}{\indexname}
%    \begin{macrocode}
%<article>\newcommand\refname{References}
%<report|book>\newcommand\bibname{Bibliography}
\newcommand\indexname{Index}
%    \end{macrocode}
% \end{macro}
% \end{macro}
% \end{macro}
%
% \begin{macro}{\figurename}
% \begin{macro}{\tablename}
%    \begin{macrocode}
\newcommand\figurename{Figure}
\newcommand\tablename{Table}
%    \end{macrocode}
% \end{macro}
% \end{macro}
%
% \begin{macro}{\partname}
% \begin{macro}{\chaptername}
% \begin{macro}{\appendixname}
% \begin{macro}{\abstractname}
%    \begin{macrocode}
\newcommand\partname{Part}
%<report|book>\newcommand\chaptername{Chapter}
\newcommand\appendixname{Appendix}
%<!book>\newcommand\abstractname{Abstract}
%    \end{macrocode}
% \end{macro}
% \end{macro}
% \end{macro}
% \end{macro}
%
% \subsection[日期]{\heiti 日期}
%
% \begin{macro}{\today}
%    该宏使用 \TeX\ 原语 |\month|、|\day| 和 |\year| 来提供 \LaTeX\ 运行的日期。
%
%    在 |\begin{document}|中此定义将得到优化,从而不会存储所有“错误(wrong)”月份的名称。
%    这种优化不会在这里完成,因为这将“冻结(freeze)” 在任何特殊用途的格式(special purpose format)中
%    的 |\today|,这种特殊用途的格式是通过将类文件(class file)加载到格式文件(format file)中而生成的格式。
% \changes{v1.3j}{1995/08/16}{use \cs{edef} to save a lot of space}
% \changes{v1.3w}{1997/10/06}{use \cs{def} again, latex/2620}
%    \begin{macrocode}
\def\today{\ifcase\month\or
  January\or February\or March\or April\or May\or June\or
  July\or August\or September\or October\or November\or December\fi
  \space\number\day, \number\year}
%    \end{macrocode}
% \end{macro}
%
% \subsection[两栏模式]{\heiti 两栏模式}
%
% \begin{macro}{\columnsep}
%    这给出了两栏模式(two-column mode)中两栏之间的距离。
%    \begin{macrocode}
\setlength\columnsep{10\p@}
%    \end{macrocode}
% \end{macro}
%
% \begin{macro}{\columnseprule}
%    这给出了两栏模式(two-column mode)中两栏之间的线段(rule)的宽度。我们没有
%    可见的线段(visible rule)。
%    \begin{macrocode}
\setlength\columnseprule{0\p@}
%    \end{macrocode}
% \end{macro}
%
% \subsection[页面样式]{\heiti 页面样式}
%    除非用户另有指定,否则文档类 article 和 report 中都有 \pstyle{plain}\ 页面。
%    在“book”文档类中,我们默认使用 \pstyle{headings}\ 页面样式。并使用阿拉伯数字
%    页码(arabic page numbers)。
%    \begin{macrocode}
%<!book>\pagestyle{plain}
%<book>\pagestyle{headings}
\pagenumbering{arabic}
%    \end{macrocode}
%
% \subsection[单面或双面打印]{\heiti 单面或双面打印}
%
%
% \changes{v1.2v}{1994/11/10}{removed typo}
%    当未指定 \Lopt{twoside}\ 选项时,我们不会试图使每个页面与其他页面一样长。
%    \begin{macrocode}
\if@twoside
\else
  \raggedbottom
\fi
%    \end{macrocode}
%    当指定了 \Lopt{twocolumn}\ 选项时,我们调用 |\twocolumn| 来激活此模式。
%    我们试图使每一栏一样长,只需调用 |sloppy| 即可。
%    \begin{macrocode}
\if@twocolumn
  \twocolumn
  \sloppy
  \flushbottom
%    \end{macrocode}
%    通常我们调用 |\onecolumn| 来启动单栏排版。
%    \begin{macrocode}
\else
  \onecolumn
\fi
%</article|report|book>
%    \end{macrocode}
%
% \changes{v1.3i}{1995/08/09}{Moved code for generic class options
% leqno and fleqn to kernel file}
%
% \Finale
%
\endinput
